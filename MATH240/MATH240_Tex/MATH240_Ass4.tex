\documentclass[12pt]{article}
\newcommand\hmmax{0}
\newcommand\bmmax{0}
%%%%--- PACKAGES ---%%%%%
% COLORS
\usepackage{xcolor}
\usepackage[many]{tcolorbox}

% MATH AND PHYSICS
\usepackage{mathtools, amssymb, amsfonts, amsthm, bm,amsmath , siunitx, xfrac, physics,breqn, undertilde,nccmath,cancel,nccmath,enumitem,venndiagram,thmtools} 

% FONTS
\usepackage{mathrsfs,bbm,bbold}

% TIKZ, FIGURES, TABLES AND CAPTIONS
\usepackage{tikz,float,wrapfig, caption,graphicx, graphics, fancyhdr, fancybox, tabularx, array,booktabs,mdframed, longtable,circuitikz}
\usepackage[labelfont=bf]{caption}
\usepackage[math]{cellspace}

% DIVERSE
\usepackage{xparse,lipsum,titling,titlesec,changepage,fullpage,listings}

%%%%--- COMMANDS ---%%%%
\newcommand{\la}{\langle} \newcommand{\ra}{\rangle}
\newcommand{\Rn}{\mathbb{R}^{n}} \newcommand{\R}{\mathbb{R}} \newcommand{\Rm}{\mathbb{R}^{m}}
\newcommand{\La}{\mathcal{L}}
\newcommand{\ep}{\epsilon} \newcommand{\de}{\delta}
\newcommand{\bs}{\backslash}
\newcommand{\vectorproj}[2][]{\textit{proj}_{\vect{#1}}\vect{#2}}

%% Spaces in tables for aesthetic arrangement
\setlength{\cellspacetoplimit}{3pt}
\setlength{\cellspacebottomlimit}{3pt}
\newcommand\numberthis{\addtocounter{equation}{1}\tag{\theequation}}
\allowdisplaybreaks

%%%--- MTPRO2 FONT ---%%%
\pdfmapfile{=mtpro2.map}
\usepackage[lite,]{mtpro2}

% Theorem environements
\newtheorem{theorem}{Theorem}[section] \theoremstyle{definition}
\newtheorem{corollary}{Corollary}[theorem] \theoremstyle{definition}
\newtheorem{lemma}[theorem]{Lemma} \theoremstyle{definition}
\newtheorem{definition}{Definition}[section] \theoremstyle{definition}
\newtheorem{Proposition}{Proposition}[section] \theoremstyle{definition}
\newtheorem*{example}{Example} \theoremstyle{example}
\newtheorem*{note}{Note} \theoremstyle{note}
\newtheorem*{remark}{Remark} \theoremstyle{remark}

%% Equation breaking
\relpenalty=9999
\binoppenalty=9999

%%%--- Command renewals ---%%%
\let\oldimplies\implies
\let\oldiff\iff

\renewcommand*{\implies}{
	\hspace{-0.05cm}\resizebox{.95\width}{\height}{$\oldimplies$}\hspace{-0.05cm}
}
\renewcommand*{\iff}{
	\hspace{-0.1cm}\oldiff\hspace{-0.1cm}
}
\renewcommand{\sectionmark}[1]{%
	\markboth{\thesection\quad #1}{}}

\renewcommand{\binom}{\mbinom}
\renewcommand{\bar}{\overline}

%%% --- DOCUMENT HEADER ---%%%
\title{MATH 240 Assignment 4}
\titleformat*{\section}{\LARGE\normalfont\fontsize{14}{14}\bfseries}
\titleformat*{\subsection}{\Large\normalfont\fontsize{12}{15}\bfseries}
\author{Mihail Anghelici 260928404 }
\date{\today}
\fancypagestyle{plain}{%
	\fancyhf{}
	\fancyhead[L]{\rule[0pt]{0pt}{0pt} Assignment 4} 
	\fancyhead[R]{\small Mihail Anghelici $260928404$} 
	\fancyfoot[C]{-- \thepage\ --}
	\renewcommand{\headrulewidth}{0.4pt}}
\pagestyle{plain}
\setlength{\headsep}{1cm}
\captionsetup{margin =1cm}
\AtBeginDocument{%
	\edef\Relbar{\mathord{\mathchar\the\numexpr\Relbar-"3000}}%
	
}
\begin{document}
	\maketitle
	\section*{Question 1}
		\subsection*{(a)}
			Given a set of $n$ elements, let us place the elements in subgroup $1$ or subgroup $2$. Choosing $0$ elements and sending in group $1$ that is $\binom{n}{0}$. Choosing $1$ element and sending it to subgroup $1$ while the $n-1$ rest in subgroup $2$, then that is $\binom{n}{1}$ combinations. Continuing this way we have a total of 
			$$\binom{n}{0}+\binom{n}{1}+\binom{n}{2}+\dots +\binom{n}{n}.$$
			This quantity is $2^{n}$ since there are two choices for every element chosen - subgroup $1$ or subgroup $2$. So $(2^{n})^{2} = 2^{2n}$.
			
			
			Then for the RHS, similarly to the LHS, we look for the number of ways of sending $0$ elements to subgroup $1$ and all $2n$ elements to subgroup $2$, then the number of ways to send $1$ in subgroup $1$ and the $2n-1$ rest in subgroup $2$, continuing in this way we get 
				$$ \binom{2n}{0}+\binom{2n}{1}+\binom{2n}{2}+\dots +\binom{2n}{2n}.$$
			This quantity is $2^{2n}$ since there are two choices for every element chosen- subroup $1$ or subgroup $2$.
			
			We conclude that the LHS and RHS are the same since both are $2^{2n}$.
		\subsection*{(b)}
			We solve this problem analytically by using the binomial theorem ; 
			$$ (x+y)^{n} = \sum_{k=0}^{n} \binom{n}{k}x^{n-k}y^{k} \xrightarrow{x=1, \ \ y =r }  \sum_{k=0}^{2n}\binom{2n}{k}(1)^{k} = (1+1)^{2n} = 2^{2n}.$$
			And for the LHS, 
			$$ (x+y)^{n} = \sum_{k=0}^{n} \binom{n}{k}x^{n-k}y^{k} \xrightarrow{x=1, \ \ y =r } \left(\sum_{k=0}^{n}\binom{n}{k}(1)^{k}\right)^{2} = (2^{n})^{2} = 2^{2n}.$$
	\section*{Question 2}
		Let us consider a set of $n$ elements. On the RHS, there are $\binom{n}{m}$ ways to chose $m$ elements from the set. If we then divide further these elements in two different subgroups, then there is an additional $2^{m}$ factor to the total possible number of ways, because each chosen $m$ has two possibilities. 
		
		
		On the LHS, we chose $k$ elements and sent them in the subgroup $1$ and the rest $m-k$ elements from the remaining $n-k$, in subgroup $2$, we do this for $1 \le k \le m$. 
		
		
		Both sides represent ways of dividing into $2$ subgroups $n$ elements of a set 
	\section*{Question 3}
		Let $A = \{a_i\}_{i=1}^{n}$. Then the possible congruence relationships are 
		$$ a_i \equiv s \mod n-1 \quad \text{for }\ s \in \{0,1, \dots ,n-1,n-2\}, \ \forall a_{i} \in A.$$
		So $\because |S|=n$, there is at least one duplicate congruence in the set, such that $\exists a_{i}, a_{j} , $ for $i\neq j$ for which $a_{i} \equiv s \mod n-1 = a_{j} \equiv s \mod n-1$ where $s \in \{0,1,\dots, n-1,n-2\}$. By properties of modular arithmetic, this implies that $a_{i} - a_{j} \equiv 0 \mod n-1 \square$.
	\section*{Question 4}
		Let $A,B$ and $C$ be the three sets of films watched by citizens respectively. There are $3$ possible cases \\
		
		\noindent \underline{\textbf{Case $1$ :}} \\
		Exists at least one citizen who watches only $A$, then every other citizen must watch at least $A$. 
		
		$\quad$ If $n$ people watch $A$ then $n \nless 2n/3 \implies$ someone sits on the floor. Same reasoning for $B$ and $C$. This case is proved.\\
		
		\noindent \underline{\textbf{Case $2$ :}} \\
		Exists at least one citizen who watches $A$ and $B$, then every other citizen must watch at least $A$ or $B$.
		
		$\quad$ Here $ 2n \ge |A \cup B| \ge n$  and $2n \ge |A \cap B| \ge 1$, consider the smallest possible value $|A \cup B|= n$ , then $|A \cap B|$ is at most $n$, so by \emph{inclusion-exclusion principle}, 
		$$ |A \cup B| = |A| + |B| - |A \cap B| \implies 2n = |A| + |B|,$$
		Impossible to divide $2n$ in two sets such that $n \nless 2n/3$ ,so at least one film is overcrowded. This case is proved.\\
		
		\noindent \underline{\textbf{Case $3$ :}} \\
		Exists at least one citizen who watches $A$,$B$ and $C$, then every other citizen must watch at least $A$ or $B$ or $C$.
		
		$\quad$ Here $ 3n \ge |A \cup B \cup C| \ge n$  and $3n \ge |A \cap B \cap C| \ge 1$. Consider the smallest possible value $|A \cup B \cup C| = n$ then $|A \cap B \cap C| $ is at most $n$, so by \emph{inclusion-exclusion principle}
		\begin{align*}  
		|A \cup B \cup C| &= |A| + |B| + |C| - |A \cap B \cap C| \\
		\implies 2n &= |A| + |B| + |C|,
		\end{align*}
		impossible to divide $2n$ in three sets such that each set has $n \nless 2n/3$, so at least one film will be overcrowded. This case is proved.\\
		
		\noindent All possible cases are verified so we conclude that in any case, someone shall sit on the floor on at least one of the films.
	\section*{Question 5}
		For $n = 1, 2, 3$ we have $2,5$ and $12$ combinations possible, respectively. We deduce by inspection that the recursive relationship is 
		$$ f_{n} = 2f_{n-1} + f_{n-2}.$$
		We proceed with solving this recursive relationship. Let $f_n = cx^{n}$ , then
		\begin{align*}
			cx^{n} = 2cx^{n-1} + cx^{n-2} \implies x^{2} = 2x +1 \xrightarrow{(-b\pm \sqrt{b^{2} - 4ac} )/2a} x = 1 \pm \sqrt{2}.
		\end{align*}
		Hence, the linear combination solution is $f_{n} = A\varphi^{n} + B \bar{\varphi}^{n},$ where $\varphi = 1 + \sqrt{2}$. Here we use the initial conditions 
		$$ \begin{rcases}
			&f_0 = 0 \implies 0 = A \varphi^{0} + B\bar{\varphi^{0}} \quad\\
			&f_1 = 1 \implies 2 = A \varphi^{1} + B\bar{\varphi^{1}} \quad 
		\end{rcases} \quad 
		\begin{align}
			0 &= A + B \\
			2 &= A \varphi^{1} + B \bar{\varphi^{1}}
		\end{align}.$$
		We deduce, 
		$$ B = -A \implies 2 = A (\varphi - \bar{\varphi}) = A((1+ \sqrt{2}) - (1 - \sqrt{2})) \implies A = 1/\sqrt{2}.$$
		Finally, the general solution to the given recursion relationship is 
		$$ f_n =\frac{1}{\sqrt{2}} ((1 + \sqrt{2})^{n} - (1 - \sqrt{2})^{n}).$$
	\section*{Question 6}
		Following the triangle inequality identity, it must be that the sum of any two sides of any triangle is strictly lager than the remaining third side. Now since 
		$$ l_i \ge l_j \quad \forall (i,j) \in (1,7) \ , i > j,$$
		then the condition for there to be a triangle is $l_i + l_{i+1} \ge l_{i+2}$, where we took the symbol greater or equal and not just strictly equal given that all elements in the set are strictly increasing. \\
		
		\noindent Let us assume the opposite of the condition, that is $l_i + l_{i+1} < l_{i+2}$. Then using the initial assumption that $l_2 \ge l_1 \ge 1$, we get 
		\begin{align*}
			l_3 &> l_2 + l_1 \ge 2 \qquad & \therefore l_3 > 2\\
			l_4 &> l_3 + l_2 \ge 3 \qquad & \therefore l_4 > 3 \\
			& \vdots & \\
			l_7 &> l_6 + l_5 \ge 13 \qquad & \therefore l_7 > 13
		\end{align*}
		The last inequality $l_7 > 13$ is a contradiction given the set as defined initially, so we conclude that the relationship $l_i + l_{i+1} \ge l_{i+2}$ holds for this given set, meaning that we can indeed construct triangles. 
\end{document}