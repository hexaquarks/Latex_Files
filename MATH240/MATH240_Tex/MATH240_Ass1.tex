\documentclass[12pt]{article}
\newcommand\hmmax{0}
\newcommand\bmmax{0}
%%%%--- PACKAGES ---%%%%%
% COLORS
\usepackage{xcolor}
\usepackage[many]{tcolorbox}

% MATH AND PHYSICS
\usepackage{mathtools, amssymb, amsfonts, amsthm, bm,amsmath , siunitx, xfrac, physics,breqn, undertilde,nccmath,cancel,nccmath,enumitem,venndiagram,thmtools} 

% FONTS
\usepackage{mathrsfs,bbm,bbold}

% TIKZ, FIGURES, TABLES AND CAPTIONS
\usepackage{tikz,float,wrapfig, caption,graphicx, graphics, fancyhdr, fancybox, tabularx, array,booktabs,mdframed, longtable,circuitikz}
\usepackage[labelfont=bf]{caption}
\usepackage[math]{cellspace}

% DIVERSE
\usepackage{xparse,lipsum,titling,titlesec,changepage,fullpage,listings}

%%%%--- COMMANDS ---%%%%
\newcommand{\la}{\langle}
\newcommand{\ra}{\rangle}
\newcommand{\Rn}{\mathbb{R}^{n}}
\newcommand{\R}{\mathbb{R}}
\newcommand{\Rm}{\mathbb{R}^{m}}
\newcommand{\La}{\mathcal{L}}
\newcommand{\ep}{\epsilon}
\newcommand{\de}{\delta}
\newcommand{\bs}{\backslash}
\newcommand{\vectorproj}[2][]{\textit{proj}_{\vect{#1}}\vect{#2}}

%% Spaces in tables for aesthetic arrangement
\setlength{\cellspacetoplimit}{3pt}
\setlength{\cellspacebottomlimit}{3pt}
\newcommand\numberthis{\addtocounter{equation}{1}\tag{\theequation}}
\allowdisplaybreaks

%%%--- MTPRO2 FONT ---%%%
\pdfmapfile{=mtpro2.map}
\usepackage[lite,]{mtpro2}

% Theorem environements
\newtheorem{theorem}{Theorem}[section]
\theoremstyle{definition}
\newtheorem{corollary}{Corollary}[theorem]
\theoremstyle{definition}
\newtheorem{lemma}[theorem]{Lemma}
\theoremstyle{definition}
\newtheorem{definition}{Definition}[section]
\theoremstyle{definition}
\newtheorem{Proposition}{Proposition}[section]
\theoremstyle{definition}
\newtheorem*{example}{Example}
\theoremstyle{example}
\newtheorem*{note}{Note}
\theoremstyle{note}
\newtheorem*{remark}{Remark}
\theoremstyle{remark}

%% Equation breaking
\relpenalty=9999
\binoppenalty=9999

%%%--- Command renewals ---%%%
\let\oldimplies\implies
\let\oldiff\iff

\renewcommand*{\implies}{
	\hspace{-0.05cm}\resizebox{.95\width}{\height}{$\oldimplies$}\hspace{-0.05cm}
}
\renewcommand*{\iff}{
	\hspace{-0.1cm}\oldiff\hspace{-0.1cm}
}
\renewcommand{\sectionmark}[1]{%
	\markboth{\thesection\quad #1}{}}

%%% --- DOCUMENT HEADER ---%%%
\title{MATH 240 Assignment 1}
\titleformat*{\section}{\LARGE\normalfont\fontsize{14}{14}\bfseries}
\titleformat*{\subsection}{\Large\normalfont\fontsize{12}{15}\bfseries}
\author{Mihail Anghelici 260928404 }
\date{\today}
\fancypagestyle{plain}{%
	\fancyhf{}
	\fancyhead[L]{\rule[0pt]{0pt}{0pt} Assignment 2} 
	\fancyhead[R]{\small Mihail Anghelici $260928404$} 
	\fancyfoot[C]{-- \thepage\ --}
	\renewcommand{\headrulewidth}{0.4pt}}
\pagestyle{plain}
\setlength{\headsep}{1cm}
\captionsetup{margin =1cm}
\AtBeginDocument{%
	\edef\Relbar{\mathord{\mathchar\the\numexpr\Relbar-"3000}}%
	
}
\begin{document}
	\maketitle
	\section*{Question 1}
	\subsection*{(a) }
		\begin{center}
			\begin{venndiagram3sets}[tikzoptions={scale=1.5, thick}, showframe={false}]
				\fillACapBCapC
				\fillOnlyC
			\end{venndiagram3sets}
		\end{center}
	\subsection*{(b) }
		The simplest description of the given Venn diagram is $$ B \oplus (A \cap C).$$
	\section*{Question 2}
		\subsection*{(a) }
		\[ 
			\begin{aligned}
				&\hspace{-2.8cm}(\subseteq) \\
				\text{Let } x \in \overline{A \cup B}. &\implies x \not\in A \cup B \\
				&\implies x \not\in A \text{ and } x \not\in B \\
				&\implies x \in \overline{A} \text{ and } x \in \overline{B} \\
				&\implies x \in \overline{A} \cap \overline{B} 
			\end{aligned}
			\qquad 
			\begin{aligned}
				&\hspace{-2.8cm}(\supseteq) \\
				\text{Let } x \in \overline{A} \cap \overline{B}. &\implies x \in \overline{A} \text{ and } x \in \overline{B} \\
				&\implies x \not\in A \text{ and } x \not\in B\\
				&\implies x \not\in A \cup B \\
				&\implies x \in \overline{A \cup B}
			\end{aligned}
		\]
		\subsection*{(b)}
		\begin{align*}
			(A \bs B) \cap (C \bs B) &= (A \cap \overline{B}) \cap (C \cap \overline{B}) \qquad&\textcolor{orange}{[\text{Set difference law}]}\\
			&=A \cap (\overline{B} \cap C) \cap \overline{B} \qquad&\textcolor{orange}{[\text{Associative law}]}\\
			&=A \cap (C \cap \overline{B}) \cap \overline{B} \qquad&\textcolor{orange}{[\text{Commutative law}]}\\
			&=(A \cap C) \cap (\overline{B} \cap \overline{B}) \qquad&\textcolor{orange}{[\text{Associative law}]}\\
			&=(A \cap C) \cap \overline{B} \qquad&\textcolor{orange}{[\text{Idenpotent law}]}\\
			&=(A \cap C) \bs B \qquad&\textcolor{orange}{[\text{Set difference law}]}
 		\end{align*}
		\section*{Question 3}
			\subsection*{(a) }
			\begin{enumerate}[label=(\roman*)]
				\item The statement is \textcolor{green}{true} since $\forall v \in \R$ we can chose $u=1$ such that the equality $uv = v$ holds. 
				\item $\neg(\exists u \in \R , \forall v \in \R, uv=v) = \forall u \in \R , \exists v \in \R, uv \neq v$.  Here we let $u=1$ then $\nexists v \in \R | uv \neq v$ . The statement is therefore \textcolor{red}{false}.
			\end{enumerate}
			\subsection*{(b) }
			\begin{enumerate}[label=(\roman*)]
				\item The given statement is \textcolor{red}{false}. Indeed, let $x=1, y=1$, then $z=1-1=0 \not\in \mathbb{N}$. 
				\item $\neg(\forall x \in \mathbb{N}, \forall y \in \mathbb{N} , \exists z \in \mathbb{N} , z =x-y) = \exists x \in \mathbb{N} , \exists y \in \mathbb{N} , \forall z \in \mathbb{N} , z \neq x -y$. Here to disprove the statement we have to show $\nexists z \in \mathbb{N}$ such that there is at least one $x$ and $y$ in $\mathbb{N}$ for which $z \neq x-y$. This is impossible since we can just pick $z=x=y$ which satisfies $z \neq x-y$. The statement is therefore \textcolor{green}{true}.
			\end{enumerate}
		\section*{Question 4}
			\subsection*{(a) }		
			\begin{displaymath}
			\begin{array}{c c c|c|c|c}
			p & q & r& p \implies r & q \implies r & (p \implies r) \iff (q \implies r)\\ % Use & to separate the columns
			\hline % Put a horizontal line between the table header and the rest.
			T & T & T & T & T & T \\
			T & T & F & F & F & T \\
			T & F & T & T & T & T \\
			T & F & F & F & T & F \\
			F & T & T & T & T & T \\
			F & T & F & T & F & F \\
			F & F & T & T & T & T \\
			F & F & F & T & T & T 
			\end{array}
			\end{displaymath}

			Not all values are true in the last column,therefore this is not a tautology.
			\subsection*{(b) }
			\begin{displaymath}
			\begin{array}{c c |c|c|c|c|c}
			p & q & \overline{p} & \overline{q} & \overline{p} \lor q  & p \land \overline{q} & (\overline{p} \lor q) \lor (p \land \overline{q})\\ % Use & to separate the columns
			\hline % Put a horizontal line between the table header and the rest.
			T & T & F & F & T & T & T\\
			T & F & F & T & F & T & T\\
			F & T & T & F & T & F & T\\
			F & F & T & T & T & T & T\\
			\end{array}
			\end{displaymath}

			All values are true in the last column, therefore this is a tautology.
		\subsection*{(c) }
			\begin{displaymath}
			\begin{array}{c c c|c|c|c|c|c|c}
			&\multicolumn{1}{c}{}&\multicolumn{1}{c}{}&\multicolumn{1}{c}{}&\multicolumn{1}{c}{\overbrace{\rule{2.0cm}{0pt}}^{:= P}}&\multicolumn{1}{c}{}&\multicolumn{1}{c}{} &\multicolumn{1}{c}{\overbrace{\rule{2.2cm}{0pt}}^{:= Q}}& & 
			\newline
			p & q & r& p \oplus q & (p \oplus q) \land r & p \land r  & q \land r &p \land r \oplus q \land r & P \iff Q\\ % Use & to separate the columns
			\hline % Put a horizontal line between the table header and the rest.
			T & T & T & F & F & T & T & F & T\\
			T & T & F & F & F & F & F & F & T\\
			T & F & T & T & T & T & F & T & T\\
			T & F & F & T & F & F & F & F & T\\
			F & T & T & T & T & F & T & T & T\\
			F & T & F & T & F & F & F & F & T\\
			F & F & T & F & F & F & F & F & T\\
			F & F & F & F & F & F & F & F & T
			\end{array}
			\end{displaymath}

			All values are true in the last column, therefore this is a tautology.
		\section*{Question 5}
			\subsection*{(a) }
				\begin{align*}
					p \implies (q \implies r) &\equiv \overline{p} \lor (q \implies r) \qquad\qquad &\textcolor{orange}{[\text{Definition of }\implies ]}\\
					&\equiv \overline{p} \lor (\overline{q} \lor r) &\textcolor{orange}{[\text{Definition of }\implies ]}\\
					&\equiv \overline{p} \lor \overline{q} \lor r &\textcolor{orange}{[\text{Commutativity of }\lor ]} \\
					&\equiv \overline{p \land q} \lor r &\textcolor{orange}{[\text{De Morgan }]}\\
					&\equiv (p \land q) \implies r &\textcolor{orange}{[\text{Definition of }\implies ]}
				\end{align*}
			\subsection*{(b) }	
				\begin{align*}
					\overline{p \implies q} &\equiv \underbrace{\overline{\overline{p} \lor q}}_{\mathclap{\textcolor{orange}{[\text{Definition of }\implies ]}}} \equiv\overbrace{ p \land \overline{q}}^{\mathclap{\textcolor{orange}{[\text{De Morgan }]}}}
				\end{align*}
			\subsection*{(c) }		
				\begin{align*}
					(\overline{p \lor \overline{q}}) \lor (\overline{p} \land \overline{q}) &\equiv (\overline{p} \land q) \lor (\overline{p} \land \overline{q}) \qquad\qquad&\textcolor{orange}{[\text{De Morgan}]} \\
					&\equiv \overline{p} \land (q \lor \overline{q})  &\textcolor{orange}{[\text{Distributive law}]} \\
					&\equiv \overline{p} \land 1 &\textcolor{orange}{[\text{Complement rule}]} \\
					&\equiv \overline{p} &\textcolor{orange}{[\text{Identity rule}]}
				\end{align*}
		\section*{Question 6}
			\subsection*{(a) }	
			
			\begin{center}
			\begin{tikzpicture}
			\node[nand port] (NAND1) {};
			\draw (NAND1.in 1) -- ++(-0.5,0) node[left ] {$p$};
			\draw (NAND1.in 2) -- ++(-0.5,0) node[left ] {$p$};
			\draw[->] (NAND1.out)  -- ++( 0.5,0) node[right] {$\overline{p \land p} \equiv \overline{p}$};
			\end{tikzpicture}
			\end{center}
		
			\subsection*{(b) }
			\begin{center}
			\begin{tikzpicture}
			\node[nand port] (NAND1) {};
			\node[nand port] at ($(NAND1) + (2,0)$) (NAND2) {};
			
			
			\draw (NAND1.in 1) -- ++(-0.5,0) node[left ] {$p$};
			\draw (NAND1.in 2) -- ++(-0.5,0) node[left ] {$q$};
			\draw (NAND2.in 1)  node[above,yshift=0mm,xshift=-3.5mm] {$\overline{p \land q}$};
			\draw (NAND2.in 2)  node[below,yshift=-1mm,xshift=-3.5mm] {$\overline{p \land q}$};

			
			\draw (NAND1.out)  -| (NAND2.in 1) -| (NAND2.in 2);
			\draw[->] (NAND2.out)  -- ++( 0.5,0) node[right] {$\overline{\overline{p \land q} \land \overline{p \land q}} \equiv \overline{\overline{p \land q}} \equiv p \land q$};
			\end{tikzpicture}
			\end{center}	
			\subsection*{(c) }
			\begin{center}
				\begin{tikzpicture}
				\node[nand port] (NAND1) {};
				\node[nand port] at ($(NAND1) + (0,-1.5)$) (NAND2) {};
				\node[nand port] at ($(NAND1) + (2,-0.75)$) (NAND3) {};
	
				
				\draw (NAND1.in 1) -- ++(-0.5,0) node[left ] {$p$};
				\draw (NAND1.in 2) -- ++(-0.5,0) node[left ] {$p$};
				\draw (NAND2.in 1) -- ++(-0.5,0) node[left ] {$q$};
				\draw (NAND2.in 2) -- ++(-0.5,0) node[left ] {$q$};
				\draw (NAND1.out)  node[above] {$\overline{p}$};
				\draw (NAND2.out)  node[below] {$\overline{q}$};
				
				\draw (NAND1.out)  -| (NAND3.in 1);
				\draw (NAND2.out)  -| (NAND3.in 2);
				\draw[->] (NAND3.out)  -- ++( 0.5,0) node[right] {$\overline{\overline{p} \land \overline{q}} \equiv \overline{\overline{p}} \lor \overline{\overline{q}} \equiv p \lor q$};
				\end{tikzpicture}
			\end{center}	
		
\end{document}