\documentclass[12pt]{article}
\newcommand\hmmax{0}
\newcommand\bmmax{0}
%%%%--- PACKAGES ---%%%%%
% COLORS
\usepackage{xcolor}
\usepackage[many]{tcolorbox}

% MATH AND PHYSICS
\usepackage{mathtools, amssymb, amsfonts, amsthm, bm,amsmath , siunitx, xfrac, physics,breqn, undertilde,nccmath,cancel,nccmath,enumitem,venndiagram,thmtools,nicematrix} 

% FONTS
\usepackage{mathrsfs,bbm,bbold}

% TIKZ, FIGURES, TABLES AND CAPTIONS
\usepackage{tikz,float,wrapfig, caption,graphicx, graphics, fancyhdr, fancybox, tabularx, array,booktabs,mdframed, longtable,circuitikz,tabu}
\usepackage[labelfont=bf]{caption}
\usepackage[math]{cellspace}

% DIVERSE
\usepackage{xparse,lipsum,titling,titlesec,changepage,fullpage,listings}

%%%%--- COMMANDS ---%%%%
\newcommand{\la}{\langle} \newcommand{\ra}{\rangle}
\newcommand{\Rn}{\mathbb{R}^{n}} \newcommand{\R}{\mathbb{R}} \newcommand{\Rm}{\mathbb{R}^{m}}
\newcommand{\La}{\mathcal{L}}
\newcommand{\ep}{\epsilon} \newcommand{\de}{\delta}
\newcommand{\bs}{\backslash}
\newcommand{\vectorproj}[2][]{\textit{proj}_{\vect{#1}}\vect{#2}}

%% Spaces in tables for aesthetic arrangement
\setlength{\cellspacetoplimit}{3pt}
\setlength{\cellspacebottomlimit}{3pt}
\newcommand\numberthis{\addtocounter{equation}{1}\tag{\theequation}}
\allowdisplaybreaks

%%%--- MTPRO2 FONT ---%%%
\pdfmapfile{=mtpro2.map}
\usepackage[lite,]{mtpro2}


% Theorem environements
\newtheorem{theorem}{Theorem}[section] \theoremstyle{definition}
\newtheorem{corollary}{Corollary}[theorem] \theoremstyle{definition}
\newtheorem{lemma}[theorem]{Lemma} \theoremstyle{definition}
\newtheorem{definition}{Definition}[section] \theoremstyle{definition}
\newtheorem{Proposition}{Proposition}[section] \theoremstyle{definition}
\newtheorem*{example}{Example} \theoremstyle{example}
\newtheorem*{note}{Note} \theoremstyle{note}
\newtheorem*{remark}{Remark} \theoremstyle{remark}

%% Equation breaking
\relpenalty=9999
\binoppenalty=9999

%%%--- Command renewals ---%%%
\let\oldimplies\implies
\let\oldiff\iff

\renewcommand*{\implies}{
	\hspace{-0.05cm}\resizebox{.95\width}{\height}{$\oldimplies$}\hspace{-0.05cm}
}
\renewcommand*{\iff}{
	\hspace{-0.1cm}\oldiff\hspace{-0.1cm}
}
\renewcommand{\sectionmark}[1]{%
	\markboth{\thesection\quad #1}{}}

\renewcommand{\binom}{\mbinom}
\renewcommand{\bar}{\overline}

%%% --- DOCUMENT HEADER ---%%%
\title{MATH 327 Assignment 3}
\titleformat*{\section}{\LARGE\normalfont\fontsize{14}{14}\bfseries}
\titleformat*{\subsection}{\Large\normalfont\fontsize{12}{15}\bfseries}
\author{Mihail Anghelici 260928404 }
\date{\today}
\fancypagestyle{plain}{%
	\fancyhf{}
	\fancyhead[L]{\rule[0pt]{0pt}{0pt} Assignment 3} 
	\fancyhead[R]{\small Mihail Anghelici $260928404$} 
	\fancyfoot[C]{-- \thepage\ --}
	\renewcommand{\headrulewidth}{0.4pt}}
\pagestyle{plain}
\setlength{\headsep}{1cm}
\captionsetup{margin =1cm}
\AtBeginDocument{%
	\edef\Relbar{\mathord{\mathchar\the\numexpr\Relbar-"3000}}%
	
}
\lstset{ %
	language=Matlab,
	backgroundcolor=\color{white},   % choose the background color
	breaklines=true,                 % automatic line breaking only at whitespace
	keywordstyle=\color{blue},
	escapeinside={(*}{*)},        % keyword style
	%identifierstyle=\color{blue},
	%stringstyle=\color{mymauve},     % string literal style
}
\begin{document}
	\maketitle
	\section*{Question 1}
		\subsection*{(a)}
			\begin{align*}
				\la Qx , Qy \ra &= (Qx)^{T}Qy \qquad\qquad& \norm{Qx}_{2} &=\sqrt{\sum_{i=1}^{n} (Qx)_{i}^{2}} \\ 
				&= x^{T}Q^{T}Qy \qquad\qquad& &=\la Qx, Qx \ra \\
				&= x^{T}Iy \qquad\qquad& &= \la x , x \ra \\
				&= x^{T}y \qquad\qquad& &= \norm{x}_{2}\\
				&= \la x , y \ra \qquad\qquad& &\implies \norm{Qx}_{2} = \norm{x}_{2}
			\end{align*}
		\subsection*{(b)}
			%TODO v vector
			We note that $(1 \ 0  \ 0\  0)^{T}$ is already in the desired form so we focus on $\hat{A} = (4 \  2  \ 4)^{T}$. We want
			\begin{gather*}
				\tilde{Q} \begin{pmatrix}
					4 \\ 2 \\4 
				\end{pmatrix} = \begin{pmatrix}
				-6 \\ 0 \\ 0
				\end{pmatrix}
				\qquad , \text{then }\ u = \begin{pmatrix}
					1 \\ 1/5 \\ 2 /5
				\end{pmatrix}
				 \implies \norm{u}_{2}^{2} = \frac{2 \tau}{\tau + x_1} = \frac65 \implies \gamma = \frac53.
			\end{gather*}
			Then we solve 
			\begin{align*}
				\tilde{Q} &= I - \gamma u u^{T} =\begin{pmatrix}
					1 & 0 & 0 \\
					0 & 1 & 0 \\
					0 & 0 & 1 
				\end{pmatrix} - \frac53 \begin{pmatrix}
					5 \\ 1 \\ 1
				\end{pmatrix}
				\begin{pmatrix}
				 5&1&1
				\end{pmatrix} = 
				\begin{pmatrix}
				1 & 0 & 0 \\
				0 & 1 & 0 \\
				0 & 0 & 1 
				\end{pmatrix} - \frac53 \begin{pmatrix}
				25 & 5 & 5 \\
				5 & 1 & 1 \\
				5 & 1 & 1
				\end{pmatrix}
				 \\ &= \begin{pmatrix}
				\frac{122}{3} & \frac{25}{3} & \frac{25}{3}\\
				\frac{25}{3} & \frac23 & \frac53 \\
				\frac{25}{3} & \frac53 & \frac23 
				\end{pmatrix} = \tilde{Q}^{T}
			\end{align*}
			Then we know that 
			$$ \hat{Q}_{2 \cross 4} = \begin{pmatrix}
			 1& 0 & 0 & 0 \\ 0 & x_{2} & x_{3} & x_{4} 
			\end{pmatrix} \qquad \text{and }\ \ R = \left( \begin{array}{cc}
					\mbox{\large$\hat{R}$ \hspace{-0.75cm}}}\\
					0& 0 \\
					0 & 0
			\end{array}\right) = \begin{pmatrix}
				1 & 2 \\ 0 & -6 \\ 0 & 0  \\ 0 & 0
			\end{pmatrix}.$$
			Here the vector $(x_{2}, x_{3} , x_{4})$ is the first row of $\tilde{Q}^{T}$ since their product gives $-\tau $, so then we conclude that
			$$ \hat{Q} = \begin{pmatrix}
				1 & 0 \\ 0 & \frac{122}{3} \\ 0 & \frac{25}{3} \\ 0 & \frac{25}{3}
			\end{pmatrix}, \qquad \hat{R} = \begin{pmatrix} 1 & 2 \\ 0 & -6 \end{pmatrix} ,\qquad u= \begin{pmatrix}
			1 \\ 1/5 \\ 2 /5
			\end{pmatrix} $$
		\subsection*{(c)}
				In the rank deficient case we have $\hat{c} = \hat{Q}^{T} b$ so by theorem there exists a decomposition $A=QR$. We apply $Q^{T}$ to the overdetermined system obtaining $Ax =b \implies Rx = Q^{T}b =c$, here $R$ is upper triangular so we can use $b$
				$$ \begin{pmatrix}
					R_{11} & R_{22} \\ 0 & 0
				\end{pmatrix}\begin{pmatrix}
					x_{1} \\ x_{2}
				\end{pmatrix} = \begin{pmatrix}
					c \\ d 
				\end{pmatrix} \implies \norm{s}_{2}= \sqrt{\norm{c - R_{11}x_{1} - R_{12}x_{2}}_{2}^{2} + \norm{d}_{2}^{2}}.$$
				With backward substitution we find $x_{1}$ then since $c - R_{11}x_{1}- R_{12}x_{2} =0$ we get $y= (x_{1} x_{2})^{T}$ which is the solution that minimizes $\norm{b - Ax}$ in $Ax = b$.
		\subsection*{(d)}
			First, we know $Rx = Q^{T} b = c = (1 \ -2/3  \ 5/3 \  -5/3)^{T} =c$. So 
			$$ \begin{pmatrix}
				1 & 2 \\ 0 & -6 \\ 0 & 0 \\ 0 & 0 
			\end{pmatrix} \begin{pmatrix}
			x_1 \\ x_2
			\end{pmatrix} = \begin{pmatrix}
			1 \\ -2/3 \\ 5/3 \\ -5/3
			\end{pmatrix} \implies 
			\begin{cases}
				x_{1} + 2x_{2} &=1 \\
				-6x_{2} &= -2/3 
			\end{cases} \implies x = \begin{pmatrix}
				7/9 \\ 1/9
			\end{pmatrix}.$$
			Then, we use $c - R_{11}x_{1}- R_{12}x_{2} =0$ ; 
			\begin{gather*}
				\begin{pmatrix}
					1 \\ -2/3 
				\end{pmatrix} - (1) \begin{pmatrix}
				7/9 \\ 1/9
				\end{pmatrix}
				- (2) x_2 = 0 \implies x_{2} = \left(\begin{pmatrix}
				-1 \\ 2/3
				\end{pmatrix} + \begin{pmatrix}
				7/9 \\ 1/9
				\end{pmatrix}\right)\left(-\frac12\right) = \begin{pmatrix}
					1/9 \\ -7/18
				\end{pmatrix}.
			\end{gather*}   
			We conclude that 
			$$ y = \begin{pmatrix}
				x_1 \\ x_2
			\end{pmatrix} = \begin{pmatrix}
				7/9 \\ 1/9 \\ 1/9 \\ -7/18
			\end{pmatrix}.$$
	\section*{Question 2}
		\subsection*{(a)}
			The forward routine is 
			\begin{lstlisting}[language=Matlab, xleftmargin=-15em, showstringspaces=true]
			% Forward substitution
			function [ y ] = fsub( R,b )
				y=nan(size(b));
				[m,n]=size(R);
				[mv,nv]=size(b);
			
				if m(*$\sim$*)=n 
					disp('matrix not square')
					return
				elseif mv(*$\sim$*)=m || nv (*$\sim$*)= 1
					disp('vector size not compatible with matrix')
					return
				end
			
				y(1)=b(1)/R(1,1);
			
				for i=2:n
					y(i)=b(i);
					for j=1:i-1
						y(i) = y(i)-R(i,j)*y(j);
					end
				y(i)=y(i)/R(i,i);
				end
			end
			\end{lstlisting}
		\subsection*{(b)}
		The coefficients found on Mathematica are 
		$$ x_{1} = \begin{pmatrix}
			1.6580 \\ 3.3360
		\end{pmatrix} , \qquad \qquad x_{2} = \begin{pmatrix}
			3.5523 \\ 0.6617 \\ 0.8914
		\end{pmatrix}.$$
			\begin{figure}[H]
				\centering
				\includegraphics[width = 1.0\linewidth]{MATH327_Ass3_Fig.jpg}
				\captionsetup{margin=1cm}
				\caption{}
			\end{figure}
	\section*{Question 3}
		\subsection*{(a)}
			Let $x \in \mathbb{R}^{m}$. Then 
			$$\norm{b- Ax}_{2} = \min\limits_{\hat{x} \in \mathbb{R}^{m}} \norm{b - A\hat{x}}_{2}  \iff b - Ax \in \mathcal{R}(A)^{\perp}.$$
			Since $\mathcal{R}(A)^{\perp} = \mathcal{N}(A^{T})$ by fact, then $x$ solves the LLSP if and only if $b - Ax \in \mathcal{N}(A^{T})$. By definition of the kernel, we write the equivalency
			$$ A^{T}(b - Ax) = 0 \implies A^{t}b = A^{T}Ax \implies A^{T}Ax = A^{T}b.$$
		\subsection*{(b)}
			\textcolor{orange}{If $A$ has full rank $(n \ge m)$} then $A^{T}A$ is positive definite. So $\exists (A^{T}A)^{-1} \implies \exists ! \ x \in \mathbb{R}^{m}$ that solves the LLSP. Indeed 
			$$ x^{T}(A^{T}A)x = 0 \implies (Ax)^{T} A^{x} = \la Ax, Ax \ra =0 \implies \mathcal{N}(A)=0,$$
			$\therefore A^{T}A$ is positive definite. In the rank deficient case, $(A^{T}A)^{-1}$ does not exist so the $x \in \mathbb{R}^{m}$ is not unique.
		\subsection*{(c)}
			\begin{gather*}
				A^{T}A = \begin{pmatrix}
					2 & 2& 1 \\ 4 & 1 & -1
				\end{pmatrix} \begin{pmatrix}
					2 &4 \\ 2 & 1 \\ 1 & -1
				\end{pmatrix} = \begin{pmatrix}
					9 & 9  \\ 9 & 18
				\end{pmatrix}, \qquad A^{T}b = \begin{pmatrix}
					2 & 2 &1 \\ 4 & 1 & -1
				\end{pmatrix} \begin{pmatrix}
					1 \\ 1 \\ 5
				\end{pmatrix} = \begin{pmatrix}
					9 \\ 0
				\end{pmatrix}.
			\end{gather*}
			We use Cholesky decomposition on $A^{T}A$ which is positive definite.
			\[ \begin{align}
				r_{11}^{2} &=9 &\implies  r_{11} &= 3\\
				r_{12}r_{11} &= 9 &\implies r_{12} &= 3\\
				r_{12}^{2} + r_{22}^{2} &= 18 &\implies r_{22} &= 3
			\end{align} \xrightarrow{} R = \begin{pmatrix}
				3 & 3 \\ 0 & 3
			\end{pmatrix}
			\]
			We apply forward substitution then backward substitution 
			\begin{align*}
				&R^{T}y = b \xrightarrow{} \begin{pmatrix}
					3 & 0 \\ 3  & 3
				\end{pmatrix}y = \begin{pmatrix}
					9 \\ 0 
				\end{pmatrix} 
				\ \implies \begin{cases}
					y_{1} & = 9/3 = 3 \\ 
					y_{2} &= (0-3(3))/3 = -3
				\end{cases}
				\qquad &\therefore y = \begin{pmatrix}
					3 \\ -3 
				\end{pmatrix} \\
				&Rx = y \xrightarrow{} \begin{pmatrix}
				3 & 3 \\ 0  & 3
				\end{pmatrix}x = \begin{pmatrix}
				3 \\ -3
				\end{pmatrix} 
				\ \implies \begin{cases}
					x_{2} &= -3/3 =-1 \\ 
					x_{1} &= (3-(3)(-3))/3 =2
				\end{cases}
				\qquad &\therefore x = \begin{pmatrix}
					2 \\ -1
				\end{pmatrix}
			\end{align*}
	\section*{Question 4}
		\subsection*{(a)}
		
		\begin{center}
			\tabulinesep=1.6mm
		\begin{tabu}{ccccc}
			\toprule
			\toprule
			\multicolumn{5}{c}{Comparing Algorithms}\\
			\multicolumn{1}{c}{$m$}&\multicolumn{1}{c}{\textit{cgs.m}}&\multicolumn{1}{c}{\textit{mgs.m}}&\multicolumn{1}{c}{ \textit{Householder}}&\multicolumn{1}{c}{\textit{QR}}\\
			\cmidrule(lr){2-5}
			$2$ & $1.6345$e$-16$   & $1.6345$e$-16$ & $7.2312$e$-16$ & $4.7108$e$-16$\\
			$4$ & $3.9218$e$-14$  & $5.7278$e$-15$ & $2.8507$e$-16$& $3.8886$e$-16$\\
			$8$ & $1.2777$e$-04$  & $8.0473$e$-12$ & $6.2207$e$-16$& $7.1078$e$-16$\\
			$16$ &$8.2721$e$+00$  & $1.2911$e$-04$ & $1.3689$e$-15$& $1.1569$e$-15$\\
			$32$ &$2.6692$e$+01$  & $1.0000$e$+00$ &$1.1423$e$-15$& $1.4161$e$-15$\\
			\bottomrule
			\bottomrule
		\end{tabu}
		\end{center}
		
		\subsection*{(b)}
			As $m$ grows the values become less accurate across all cases. This inaccuracy grows particularly fast for the \text{cgs} case around $m=8_{+}$. The \textit{mgs} case's inaccuracy increases slower but increases nevertheless, while the \textit{Householder triangularization} method seems to be very accurate and run on par with the MatLab built in \textit{QR} method. For this reason it is evident that MatLab uses the Householder triangularization method with some slight adjustments. 
		\subsection*{(c)}
			\begin{center}
				\tabulinesep=1.6mm
				\begin{tabu}{ccc}
					\toprule
					\toprule
					\multicolumn{3}{c}{Comparing Successive Algorithms}\\
					\multicolumn{1}{c}{Application}&\multicolumn{1}{c}{\textit{cgs.m}}&\multicolumn{1}{c}{\textit{mgs.m}}\\
					\cmidrule(lr){2-3}
					$1$ & $2.6692$e$+01$   & $1.0000$e$+00$\\
					$2$ & $2.1939$e$+01$  & $4.4033$e$-11$\\
					\bottomrule
					\bottomrule
				\end{tabu}
			\end{center}
		Indeed the accuracy increases and it increases substantially faster for the \textit{mgs.m} case, as it was expected.
		\subsection*{(d)}
			Computationally, the safest method would be the \textit{QR} built in method since it is an improved version of the \texit{Householder triangularization} and it is the most accurate across all matrix sizes. While, by hand, the \textit{CGS} method is sufficiently intuitive, easy, and accurate for low matrix size, so it is more appropriate. 
	\section*{Question 5}
		\subsection*{(a)}
			We first note that the rank of the matrix is $1$ since each column is linearly dependent on $u$. By the outer product formulation theorem, we have 
			$$ A = \sum_{j=1}^{r} \sigma_j u_j v_{j}^{T} = \sigma_1 u_1 v1j^{T} = \hat{U}\hat{\Sigma}\hat{V}^{T}.$$
			So then by the condensed SVD theorem we get 
			$$ A = \begin{pmatrix} u_1 \\ \vdots \\ u_m\end{pmatrix} \begin{pmatrix} \sigma_1 \end{pmatrix} \begin{pmatrix} v_1 \dots v_n\end{pmatrix}.$$
			Where $\sigma_1 =1$. But here $\hat{U}$ and $\hat{V}$ are not isometries as the theorem requires. So we normalize the vectors and conclude that 
			$$ A = \underbrace{\frac{u}{\norm{u}}}_{\hat{U}} \underbrace{(\norm{u}\norm{v})}_{\hat{\Sigma}} \underbrace{\frac{v}{\norm{v}}}_{\hat{V}^{T}}.$$
		\subsection*{(b)}
			We first find $V$ through $A^{T}A$
			$$ A^{T}A = \begin{pmatrix}
				14& 28 \\ 
				28 & 56
			\end{pmatrix} \xrightarrow{} \begin{align}
				\text{For } &\lambda =70 : & \begin{pmatrix}
					-56 & 28 \\ 28 & -14
				\end{pmatrix} \begin{pmatrix}
					x \\ y
				\end{pmatrix} = \begin{pmatrix}
					0 \\ 0
				\end{pmatrix} \impliex  &\begin{pmatrix}
					x \\ y
				\end{pmatrix} = \begin{pmatrix}
					1 \\ 2
				\end{pmatrix} := v_{1} \\
				\text{For } &\lambda =0 : & \begin{pmatrix}
				14 & 28 \\ 28 & 56
				\end{pmatrix} \begin{pmatrix}
				x \\ y
				\end{pmatrix} = \begin{pmatrix}
				0 \\ 0
				\end{pmatrix} \implies  &\begin{pmatrix}
				x \\ y
				\end{pmatrix} = \begin{pmatrix}
				-2 \\ 1
				\end{pmatrix} := v_{2}
			\end{align}$$			
			The vectors need to be normalized so we have 
			$$ V^{T} = \begin{pmatrix}
				\frac{1}{\sqrt{5}} & \frac{2}{\sqrt{5}} \\
				-\frac{2}{\sqrt{5}} & \frac{1}{\sqrt{5}}
 			\end{pmatrix} \implies \hat{V}^{T} = \begin{pmatrix}
 				\frac{1}{\sqrt{5}} &\frac{2}{\sqrt{5}}
 			\end{pmatrix}.$$
 			Similarly, we find $U$ through $AA^{T}$.
 			$$ AA^{T} = \begin{pmatrix}
 				5 & 10 & 15 \\ 
 				10& 20 & 30 \\
 				15 & 30 & 45
 			\end{pmatrix} \xrightarrow{} \begin{align*}
 				\text{For } &\lambda = 70 : &\begin{pmatrix}
	 				-65 & 10 & 15 \\
	 				10 & -50 & 30 \\
	 				15 & 30 & -25
 				\end{pmatrix}\begin{pmatrix}
 					x \\ y \\z
 				\end{pmatrix} = \begin{pmatrix}
 					0 \\ 0 \\0
 				\end{pmatrix} \implies \begin{pmatrix}
 					x \\ y \\ z
 				\end{pmatrix} = \begin{pmatrix}
 					1 \\ 2 \\3
 				\end{pmatrix} :=  u_1\\ 
 				\text{For } &\lambda = 0 : &\begin{pmatrix}
 				5 & 10 & 15 \\ 
 				10& 20 & 30 \\
 				15 & 30 & 45
 				\end{pmatrix}\begin{pmatrix}
 					x \\ y \\ z
 				\end{pmatrix} = \begin{pmatrix}
 					0 \\ 0 \\ 0
 				\end{pmatrix} \implies \begin{pmatrix}
 					x \\ y \\ z
 				\end{pmatrix} = \begin{pmatrix}
 					2 \\ -1 \\ 0 
 				\end{pmatrix}+ \begin{pmatrix}
 					3 \\ 0 \\ -1
 				\end{pmatrix} \\
 				& &  :=u_2 + u_3
 			\end{align*}$$
 			The vectors need to be normalized so we have
 			$$ U = \begin{pmatrix}
 				\frac{1}{\sqrt{14}} & \frac{2}{\sqrt{5 }} & \frac{3}{\sqrt{10}} \\
 				\frac{2}{\sqrt{14}} & -\frac{1}{\sqrt{15}} & 0 \\
 				\frac{3}{\sqrt{14}} & 0 & -\frac{1}{\sqrt{10}}
 			\end{pmatrix} \implies \hat{U} = \begin{pmatrix}
 				\frac{1}{\sqrt{14}} \\ \frac{2}{\sqrt{14}} \\ \frac{3}{\sqrt{14}}
 			\end{pmatrix}.$$
 			Also, from the eigenvalues of $A^{T}A$ we get $\Sigma$
 			$$ \Sigma = \begin{pmatrix}
 				\sqrt{70} & 0 \\0 & \sqrt{0} \\ 0 & 0
 			\end{pmatrix} \implies \hat{\Sigma} = \begin{pmatrix}
 				\sqrt{70} 
 			\end{pmatrix}.$$
 			Finally we conclude the condensed SVD decomposition is 
 			$$ \hat{U} = \begin{pmatrix}
 			\frac{1}{\sqrt{14}} \\ \frac{2}{\sqrt{14}} \\ \frac{3}{\sqrt{14}}
 			\end{pmatrix}, \qquad \hat{\Sigma} = \begin{pmatrix}
 				\sqrt{70} 
 			\end{pmatrix}, \qquad \hat{V}^{T} = \begin{pmatrix}
 				\frac{1}{\sqrt{5}} & \frac{2}{\sqrt{5}} 
 			\end{pmatrix}.$$
 			
 			\begin{align*} 
 			&a,b,\textcolor{red}{c}, d, e, f, \textcolor{red}{g},h ,i \\
 			&\underbrace{a, b}_{\text{deck1}}, \ \  \underbrace{\textcolor{red}{c}, d, e, f, \textcolor{red}{g}}_{\text{this deck}}, \ \  \underbrace{h, i}_{\text{deck3}}\\ 
 			\to & \underbrace{a, b}_{\text{deck1}}, \ \ \underbrace{h, i,\textcolor{red}{c}, d, e, f, \textcolor{red}{g}}_{\text{new deck}} \\
 			\to& \underbrace{ h, i,\textcolor{red}{c}, d, e, f, \textcolor{red}{g},a,b }_{\text{final deck}} 
 			\end{align*}
\end{document}