\documentclass[12pt]{article}
\newcommand\hmmax{0}
\newcommand\bmmax{0}
%%%%--- PACKAGES ---%%%%%
% COLORS
\usepackage{xcolor}
\usepackage[many]{tcolorbox}

% MATH AND PHYSICS
\usepackage{mathtools, amssymb, amsfonts, amsthm, bm,amsmath , siunitx, xfrac, physics,breqn, undertilde,nccmath,cancel,nccmath,enumitem,venndiagram,thmtools,nicematrix} 

% FONTS
\usepackage{mathrsfs,bbm,bbold}

% TIKZ, FIGURES, TABLES AND CAPTIONS
\usepackage{tikz,float,wrapfig, caption,graphicx, graphics, fancyhdr, fancybox, tabularx, array,booktabs,mdframed, longtable,circuitikz,tabu}
\usepackage[labelfont=bf]{caption}
\usepackage[math]{cellspace}

% DIVERSE
\usepackage{xparse,lipsum,titling,titlesec,changepage,fullpage,listings}

%%%%--- COMMANDS ---%%%%
\newcommand{\la}{\langle} \newcommand{\ra}{\rangle}
\newcommand{\Rn}{\mathbb{R}^{n}} \newcommand{\R}{\mathbb{R}} \newcommand{\Rm}{\mathbb{R}^{m}}
\newcommand{\La}{\mathcal{L}}
\newcommand{\ep}{\epsilon} \newcommand{\de}{\delta}
\newcommand{\bs}{\backslash}
\newcommand{\vectorproj}[2][]{\textit{proj}_{\vect{#1}}\vect{#2}}

%% Spaces in tables for aesthetic arrangement
\setlength{\cellspacetoplimit}{3pt}
\setlength{\cellspacebottomlimit}{3pt}
\newcommand\numberthis{\addtocounter{equation}{1}\tag{\theequation}}
\allowdisplaybreaks

%%%--- MTPRO2 FONT ---%%%
\pdfmapfile{=mtpro2.map}
\usepackage[lite,]{mtpro2}


% Theorem environements
\newtheorem{theorem}{Theorem}[section] \theoremstyle{definition}
\newtheorem{corollary}{Corollary}[theorem] \theoremstyle{definition}
\newtheorem{lemma}[theorem]{Lemma} \theoremstyle{definition}
\newtheorem{definition}{Definition}[section] \theoremstyle{definition}
\newtheorem{Proposition}{Proposition}[section] \theoremstyle{definition}
\newtheorem*{example}{Example} \theoremstyle{example}
\newtheorem*{note}{Note} \theoremstyle{note}
\newtheorem*{remark}{Remark} \theoremstyle{remark}

%% Equation breaking
\relpenalty=9999
\binoppenalty=9999

%%%--- Command renewals ---%%%
\let\oldimplies\implies
\let\oldiff\iff

\renewcommand*{\implies}{
	\hspace{-0.05cm}\resizebox{.95\width}{\height}{$\oldimplies$}\hspace{-0.05cm}
}
\renewcommand*{\iff}{
	\hspace{-0.1cm}\oldiff\hspace{-0.1cm}
}
\renewcommand{\sectionmark}[1]{%
	\markboth{\thesection\quad #1}{}}

\renewcommand{\binom}{\mbinom}
\renewcommand{\bar}{\overline}

%%% --- DOCUMENT HEADER ---%%%
\title{MATH 327 Assignment 4}
\titleformat*{\section}{\LARGE\normalfont\fontsize{14}{14}\bfseries}
\titleformat*{\subsection}{\Large\normalfont\fontsize{12}{15}\bfseries}
\author{Mihail Anghelici 260928404 }
\date{\today}
\fancypagestyle{plain}{%
	\fancyhf{}
	\fancyhead[L]{\rule[0pt]{0pt}{0pt} Assignment 4} 
	\fancyhead[R]{\small Mihail Anghelici $260928404$} 
	\fancyfoot[C]{-- \thepage\ --}
	\renewcommand{\headrulewidth}{0.4pt}}
\pagestyle{plain}
\setlength{\headsep}{1cm}
\captionsetup{margin =1cm}
\AtBeginDocument{%
	\edef\Relbar{\mathord{\mathchar\the\numexpr\Relbar-"3000}}%
	
}
\lstset{ %
	language=Matlab,
	backgroundcolor=\color{white},   % choose the background color
	breaklines=true,                 % automatic line breaking only at whitespace
	keywordstyle=\color{blue},
	escapeinside={(*}{*)},        % keyword style
	%identifierstyle=\color{blue},
	%stringstyle=\color{mymauve},     % string literal style
}
\begin{document}
	\maketitle
	\section*{Question 1}
	First we deduce 
	$$ \norm{A}_F^{2} = \sum_{i,j} \abs{A_{ij}}^{2} = \sum_i \left(\sum_j A_{ij}^{T} A_{ji}\right) = \sum_i (A^{T}A)_{ii} = \tr(A^{T}A) = \tr(AA^{T}).$$
	So then,
			\begin{align*}
				\norm{A}_F^{2} = \sum_i \sum_j \abs{a_{ij}}^{2} &= \tr(A A^{T}) \\ 
				&=\tr((U \Sigma V^{T} )(U \Sigma V^{T})) \\
				&=\tr((U \Sigma V^{T})(\Sigma V^{T})^{T} U^{T}) \\
				&=\tr((U \Sigma V^{T})V \Sigma^{T} U^{T}) \\
				&=\tr (\Sigma \Sigma^{T}) = \sum_i (\sigma_i)^{2}
			\end{align*}
			Thus indeed, 
			$$ \norm{A}_F = \sqrt{\sum_i(\sigma_i)^{2}}.$$
	\section*{Question 2}
		\subsection*{(a)}	
			For each row multiplication, there are $n-1$ additions and $n$ multiplications, so for $m$ rows in total that is $(n-1)m + nm = m(2n-1)$ operations in total.
			
		\subsection*{(b)}
			For $v_{i}^{T} x$ , we have $2n-1$ operations. Multiplying by $u_{i}$ from the left that is a total of $(2n -1)+m$ operations. And finally, multiplying by the constant $\sigma_i$ we conclude there are a total of $(2n-1)+2m$ operations.
			
		\subsection*{(c)}
		We start by $\sigma_i u_i$ which encompasses $m$ operations.Then,	$(\sigma_i u_i) v_i^{T}$ is an outer product and represents $m+mn$ operations. Summing summing over $k$ we get $(m+mn)(k-1)$ operations with an additional $(k-1)mn$ operations for summing up the matrices. Finally, from $(a)$ we know that $Ax$ requires $m(2n-1)$ operations so we conclude that 
			$$ A_k (x) \to (m+mn)(k-1) +(k-1)mn +m(2n-1)  \ \text{operations.}$$
		Evidently it is more efficient to compute $Ax$ instead of $A_k x$. There is an additional $(k-1)(m+mn) +(k-1)m$ operations within the latter case so for $k,m,n \ge 1$ the $Ax$ is more efficient.
	\section*{Question 3}
		\subsection*{(a)}
				\begin{figure}[H]
					\centering
					\includegraphics[width = 0.8\linewidth]{MATH327_Fig1.png}
					\captionsetup{margin=1cm}
					\caption{SVD decomposition for $k$ approximations of $1,2,4,16,64$, compared to the original picture defined by $A$.}
			\end{figure}
		\subsection*{(b)}
				\begin{figure}[H]
					\centering
					\includegraphics[width = 0.75\linewidth]{MATH327_Fig2.png}
					\captionsetup{margin=1cm}
					\caption{}
			\end{figure}
		The largest value of $k$ is $9$ and is found using the following code snippet : 
			\begin{lstlisting}[language=Matlab, xleftmargin=-15em, showstringspaces=true]
				A_k_list = [];
				[U,S,V] = svd(A);
				i=1;
				for k=1:64
					Ak = U(:,1:k)*S(1:k,1:k)*V(:,1:k)';
					A_k_list{i} = Ak;
					if (norm(A-A_k_list{i} ,2)/norm(A, 2) <=0.05)
						largest_k = k;
						break;
					end
					i=i+1;
				end
			\end{lstlisting}
	\section*{Question 4}
		\subsection*{(a)}	
			\begin{itemize}
				\item	$\norm{A}_2 = \max\limits_{x \neq 0} \frac{\norm{Ax}_2}{\norm{x}_2} = \max\limits_i  \sigma_{i}(A) = 4.$
				\item $\norm{A}_F = \sqrt{\sum_{i} \sigma_i^{2}}  = \sqrt{30}.$
				\item Since $\norm{QA} = \norm{A}$ for $Q \perp$. Then $\norm{U}_2 =1.$
				\item By inspection of  $\Sigma$, we deduce that $\rank(A) =4$. 
			\end{itemize}
		\subsection*{(b)}
			\begin{align*}
				&\mathcal{R} (A) = \text{Span}\{u_{1} , \dots , u_{r}\} \xrightarrow{\text{basis}}
				\begin{pmatrix}
					0 \\ \sfrac{1}{\sqrt{3}} \\ -\sfrac{1}{\sqrt{3}} \\ \sfrac{1}{\sqrt{3}} \\ 0 
				\end{pmatrix} , 
				\begin{pmatrix}
					1 \\ 0 \\ 0 \\ 0 \\ 0 
				\end{pmatrix} , 
				\begin{pmatrix}
					0 \\ -\sfrac{1}{\sqrt{2}} \\ -\sfrac{1}{\sqrt{2}} \\ 0 \\ 0 
				\end{pmatrix} , 
				\begin{pmatrix}
					0 \\ 0 \\ 0 \\ 0 \\ 1
				\end{pmatrix}
				\quad &\dim(\mathcal{R}(A)) = 4 \\
				&\mathcal{N}(A) = \text{Span}\{v_{r+1} , \dots , v_{n}\} \xrightarrow{}{} \ \text{no basis since } &\dim(\mathcal{N}(A)) = 0.
			\end{align*}
		\subsection*{(c)}
			Let $U = [U_1 \ U_2]$ and $\Sigma = [\hat{\Sigma} \ 0]^{T}$ for $U_{m \cross m} , U_{1, m\cross n}, U_{2, m \cross (m-n)}, \Sigma_{m\cross n}$ and $\hat{\Sigma}_{ n \cross n}$. Then we state ; 
			$$ A = \begin{pmatrix}
				0 & 1 & 0 & 0 \\
				\sfrac{1}{\sqrt{3}} & 0 & -\sfrac{1}{2} & 0 \\
				-\sfrac{1}{3} & 0 & - \sfrac{1}{2} & 0 \\
				-\sfrac{1}{3} &0 & - \sfrac{1}{2} & 0 \\
				\sfrac{1}{3} & 0 & 0 & 0 \\
				0 & 0 & 0 & 1 
			\end{pmatrix}\begin{pmatrix}
				4 & 0 & 0 & 0 \\0 & 3 & 0 & 0 \\0 & 0 & 2 & 0 \\0 & 0 & 0 & 1
			\end{pmatrix} \begin{pmatrix}
				0 & 1 & 0 & 0 \\
				 0 & 0 & -1 & 0 \\ 1 & 0 & 0 &0 \\0 & 0 & 0 & 1
			\end{pmatrix}.$$
		\subsection*{(d)}
			$$ A^{T}A = (U \Sigma V^{T})^{T} (U \Sigma V^{T}) = V \Sigma^{T} U^{T} U \Sigma V^{T} = V(\hat{\Sigma})^{2} V^{T}.$$
			Here $\hat{\Sigma}$ is diagonal so $\sigma_i (A^{T}A) = \sigma_i^{2}(\hat{\Sigma}) = \{16,9,4,1\}$. Moreover, the eigenvalues of $A^{T}A$ are the same as the singular values in the present case.		
		\subsection*{(e)}
			We use the definition ; 
			$$ \frac{\norm{A - A_2}_2}{\norm{A}_2} = \frac{1}{\kappa_2(A)} \implies \frac{\min\limits_{x \neq 0} \sfrac{\norm{Ax}_2}{\norm{x}_2}}{\max\limits_{x \neq 0} \sfrac{\norm{Ax}_2}{\norm{x}_2}}\norm{A}_2 = \frac{\sigma_{\min}(A)}{\sigma_{\max}(A)}\sigma_{\max}(A) =\frac{1}{4} 4= 1.$$
			Moreover, 
			$$ A_{2} = U(:, 1 : 2)\Sigma(1:2, 1:2)V(:,1:2)^{T} = \begin{pmatrix}
				0 & 1  \\ \sfrac{1}{\sqrt{3}} & 0 \\ - \sfrac{1}{\sqrt{3}} & 0\\ \sfrac{1}{\sqrt{3}} & 0 \\ 0 & 0 
			\end{pmatrix} \begin{pmatrix}
				4 & 0 \\ 0 & 3
			\end{pmatrix} \begin{pmatrix}
				0 & 1 & 0 & 0 \\ 0 & 0 & -1 & 0
			\end{pmatrix}.$$
	\section*{Question 5}
		\subsection*{(a)}
			$$ A = \begin{pmatrix}
				1 & 2 & 3 \\ 4 & 5 & 6 \\ 7 & 8 & 9
			\end{pmatrix} \sim \begin{pmatrix}
				1 & 2 & 3 \\
				0 & 1 & 2 \\
				0 & 1 & 2
			\end{pmatrix} \xrightarrow{} \text{Rank}(A) = 2.$$
		\subsection*{(b)}	
			It returns $1 \cross 10^{-15}$ which is actually the machine precision. This arises from the $\text{svd}(A)$ function called. The answer therefore does not contradict that from $(a)$. 
\end{document}