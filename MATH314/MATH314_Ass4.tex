\documentclass[
	12pt,
	]{article}
		\usepackage{xcolor}
			\usepackage[dvipsnames]{xcolor}
			\usepackage[many]{tcolorbox}
		\usepackage{changepage}
		\usepackage{titlesec}
		\usepackage{caption}
		\usepackage{mdframed, longtable}
		\usepackage{mathtools, amssymb, amsfonts, amsthm, bm,amsmath} 
		\usepackage{array, tabularx, booktabs}
		\usepackage{graphicx,wrapfig, float, caption}
		\usepackage{tikz,physics,cancel, siunitx, xfrac}
		\usepackage{graphics, fancyhdr}
		\usepackage{lipsum}
		\usepackage{xparse}
		\usepackage{thmtools}
		\usepackage{mathrsfs}
		\usepackage{undertilde}
		\usepackage{tikz}
		\usepackage{fullpage}
		\usepackage[labelfont=bf]{caption}
	\newcommand{\td}{\text{dim}}
	\newcommand{\tvw}{T : V\xrightarrow{} W }
	\newcommand{\ttt}{\widetilde{T}}
	\newcommand{\ex}{\textbf{Example}}
	\newcommand{\aR}{\alpha \in \mathbb{R}}
	\newcommand{\abR}{\alpha \beta \in \mathbb{R}}
	\newcommand{\un}{u_1 , u_2 , \dots , n}
	\newcommand{\an}{\alpha_1, \alpha_2, \dots, \alpha_2 }
	\newcommand{\sS}{\text{Span}(\mathcal{S})}
	\newcommand{\sSt}{($\mathcal{S}$)}
	\newcommand{\la}{\langle}
	\newcommand{\ra}{\rangle}
	\newcommand{\Rn}{\mathbb{R}^{n}}
	\newcommand{\R}{\mathbb{R}}
	\newcommand{\Rm}{\mathbb{R}^{m}}
	\usepackage{fullpage, fancyhdr}
	\newcommand{\La}{\mathcal{L}}


	\usepackage{mathtools}
	\DeclarePairedDelimiter{\norm}{\lVert}{\rVert}
	\newcommand{\vectorproj}[2][]{\textit{proj}_{\vect{#1}}\vect{#2}}
	\newcommand{\vect}{\mathbf}
	\newcommand{\uuuu}{\sum_{i=1}^{n}\frac{<u,u_i}{<u_i,u_i>} u_i}
	\newcommand{\B}{\mathcal{B}}
	\newcommand{\Ss}{\mathcal{S}}
	
	\newtheorem{theorem}{Theorem}[section]
	\theoremstyle{definition}
	\newtheorem{corollary}{Corollary}[theorem]
	\theoremstyle{definition}
	\newtheorem{lemma}[theorem]{Lemma}
	\theoremstyle{definition}
	\newtheorem{definition}{Definition}[section]
	\theoremstyle{definition}
	\newtheorem{Proposition}{Proposition}[section]
	\theoremstyle{definition}
	\newtheorem*{example}{Example}
	\theoremstyle{example}
	\newtheorem*{note}{Note}
	\theoremstyle{note}
	\newtheorem*{remark}{Remark}
	\theoremstyle{remark}
	\newtheorem*{example2}{External Example}
	\theoremstyle{example}
	
	\title{MATH 314 Ass 4.}
	\titleformat*{\section}{\LARGE\normalfont\fontsize{12}{12}\bfseries}
	\titleformat*{\subsection}{\Large\normalfont\fontsize{10}{15}\bfseries}
	\author{Mihail Anghelici 260928404 }
	\date{\today}
	
	\relpenalty=9999
			\binoppenalty=9999
		
			\renewcommand{\sectionmark}[1]{%
			\markboth{\thesection\quad #1}{}}
			
			\fancypagestyle{plain}{%
			  \fancyhf{}
			  \fancyhead[L]{\rule[0pt]{0pt}{0pt} Assignment 4} 
			  \fancyhead[R]{\small Mihail Anghelici $260928404$} 
			  \fancyfoot[C]{-- \thepage\ --}
			  \renewcommand{\headrulewidth}{0.4pt}}
			\pagestyle{plain}
			\setlength{\headsep}{1cm}
	\captionsetup{margin =1cm}
	\begin{document}
	\maketitle
		\section*{Question 1}
			The orientation is positive therefore we may use the standard form of the definition , i.e.,
			\begin{equation*}
				\int\limits_{C} P \ dx + Q \ dy = \iint\limits_{D} \pdv{Q}{x} - \pdv{P}{y} \ dA .
			\end{equation*}
			Setting $P(x,y) = y^{2} + x^{3}$ and $Q(x,y) = x^{4}$ we get
			\begin{align*}
				\int\limits_{C} y^{2} + x^{3}dx + x^{4} dy &= \int_{0}^{1} \int_{0}^{1} \left(\pdv{Q}{x} - \pdv{P}{y}\right) \ dydx \\
				&= \int_{0}^{1} \int_{0}^{1} (4x^{3}y - 2y) \ dy dx\\
				&= \int_{0}^{1} \left(\eval{4x^{3}y}_{0}^{1} - \eval{y^{2}}_{0}^{1}\right) \ dx 
				= \int_{0}^{1} 4x^{3} - 1 \ dx = \eval{x^{4}}_{0}^{1} - \eval{x}_{0}^{1} = 0.
			\end{align*}
		\section*{Question 2}
			\begin{equation} 
			\text{Area}(D) = \int\limits_{C}x \ dy = - \int\limits_{C} y \ dx.
			\end{equation}
			First assume $(1)$ is true. Since the contour integral is linear,
			\begin{gather*}
			\int\limits_{C} x \ dy + \int\limits_{C} y \ dx = 0 \implies \int\limits_{C} x\ dy + y \ dx =0.
			\end{gather*}
			Converting the last integral using Green's theorem and show it is equal to $0$ should imply that the RHS equality in $(1)$ is correct.
			\begin{gather*}
			\int\limits_{C} x \ dy +y \ dx = \int\limits_{C} Q \ dy + P \ dx \implies \iint\limits_{D} \frac{\partial Q}{\partial x} - \frac{\partial P}{\partial y} \ dA = \iint\limits_{D}(1-1)\ dA = 0.
			\end{gather*}
			We now show that Area$(D) = \int\limits_{C} x \ dy$.
			\begin{equation*}
				\int\limits_{C} x \ dy = \iint\limits_{D} \left(\pdv{Q}{x} - \pdv{P}{y}\right) \ dA = \iint\limits_{D} (1-0) \ dA = \iint\limits_{D} \ dA
 			\end{equation*}
 			By definition, $\iint\limits_{D} \ dA = \text{Area}(D)$, which completes the proof.
 			
 			\section*{Question 3}
 				\begin{align*}
 					&\implies \vec{T}_{u} = (1,1,v) \quad, \vec{T}_{v} = (-1,1,u)\quad, \vec{T}_{u} \cross \vec{T}_{v} = \begin{vmatrix}
 					\vec{i} & \vec{j} & \vec{k} \\ 1 & 1 & v \\ -1 & 1& u
 					\end{vmatrix} 
 					= (u-v, u+v, 2). \\
 					&\implies \norm{\vec{T}_{u} \cross \vec{T}_{v}} = \sqrt{2u^{2} + 2v^{2} + 4} = \sqrt{2} \sqrt{u^{2} + v^{2} + 2}
 				\end{align*}
 				We may now use the surface area definition 
 				\begin{align*}
 					\text{S.A} &= \iint\limits_{D} \norm{\vec{T}_{u} \cross \vec{T}_{v}} \ dudv \\
 					&= \sqrt{2}\iint\limits_{D} \sqrt{u^{2} + v^{2} +2}\ dudv
 					\intertext{Let us convert to polar coordinates , $u = r\cos\theta$ and $v = r\sin \theta$, yielding}
 					&= \sqrt{2} \int_{0}^{2\pi} \int_{0}^{1} \sqrt{r^{2} +2} r \ drd\theta \\
 					&\overset{u=r^{2} +2}{=} \frac{1}{\sqrt{2}} \int_{0}^{2\pi} \int_{2}^{3} u^{1/2}\ dud\theta \\
 					&=\frac{1}{\sqrt{2}} \int_{0}^{2\pi} \frac23 \eval{u^{3/2}}_{2}^{3} \ d\theta  \\
 					&= \frac{\sqrt{2}}{3} \left(\int_{0}^{2\pi} 3^{3/2} \ d\theta - \int_{0}^{2\pi} 2^{3/2} \ d\theta\right) =\frac{\sqrt{2}}{3} 2\pi (3^{3/2} - 2^{3/2}).
 				\end{align*}
 			\section*{Question 4}
 				\begin{align*}
 				\vec{T}_u &= (2\cos(u) , -3\sin (u), 0) \\
 				\vec{T}_v &= (0,0,1)\\
 				\vec{T}_{u} \times \vec{T}_v &= \begin{vmatrix} 
 				\vec{i}&\vec{j}&\vec{k} \\ 2\cos u & -3\sin u & 0 \\ 0 & 0& 1
 				\end{vmatrix} = (-3\sin (u) , 2\cos (u), 0).
 				\end{align*}
 				Then we use the surface integral for vector fields definition 
 				\begin{align*}
 				\iint\limits_S F \cdot ds &= \iint\limits_{D} F(\phi(u,v)) \cdot (\vec{T}_{u} \times \vec{T}_{v})  dA\\
 				&=\int_{0}^{2\pi} \int_{0}^{1}(2\sin u , 3\cos u, v)\cdot (-3 \sin u , 2\cos u , 0) \ dvdu \\
 				&=\int_{0}^{2\pi} \int_{0}^{1} 6(\cos^{2}(u) - \sin^{2}(u)) \ dvdu \\
 				&= \int_{0}^{2\pi} 6\cos(2u) \ du =3\eval{\sin(2u)}_{0}^{2\pi} = 0.
 				\end{align*}
 			\section*{Question 5}
 				First and foremost we set $x = \sqrt{1- y^{2} -z^{2}}$ , $y = y $ and $z = z$.
 				\begin{align*}
 					\implies &\vec{T}_{y} = \left(\frac{-y}{\sqrt{1-y^{2} -z^{2}}}, 1, 0\right) \quad ,\vec{T}_{z} = \left(\frac{-z}{\sqrt{1-y^{2} - z^{2}}}, 0 ,1\right)\\
 					& \vec{T}_{y} \cross \vec{T}_{z} = \begin{vmatrix}
 						\vec{i} & \vec{j} & \vec{k} \\
 						\frac{-y}{\sqrt{1- y^{2} - z^{2}}} & 1 & 0 \\
 						\frac{-z}{\sqrt{1- y^{2} - z^{2}}} & 0 & 1
 					\end{vmatrix} 
 					= \left(1, \frac{-y}{\sqrt{1- y^{2} - z^{2}}} ,\frac{z}{\sqrt{1- y^{2} - z^{2}}}\right),
 				\end{align*}
 				$\therefore$ the vector $\vec{n}$ is pointing out of the surface.
 				Since we're given that $x\ge 0,  \implies $ $0 = \sqrt{1-y^{2} - z^{2}} \implies 1 = y^{2} + z^{2}$. Thus, the boundary of the given shape at $x=0$ is a circle of radius $r$ parameterized as
 				$$ c(t) = (0,\sin (t) , \cos (t)) \qquad , 0 \le t \le 2\pi.$$
 				Finally, by Stoke's Theorem, 
 				\begin{align*}
 					\int\limits_{S} \text{Curl} F \cdot ds &= \int_{0}^{2\pi} F \cdot ds = \int_{0}^{2\pi} F(c(t)) \cdot c'(t) \ dt
 					\intertext{Since $c'(t) = (0, \cos(t) , -\sin(t))$ this gives}
 					&= \int_{0}^{2\pi} (0,-\sin^{3}(t) ,0) \cdot (0 ,\cos(t) , -\sin(t) )\ dt\\
 					&= \int_{0}^{2\pi}(0- \sin^{3}(t)\cos(t) + 0) \ dt \\
 					&\overset{u=\sin(t)}{=} \frac{-1}{4} \eval{\sin^{4}(t)}_{0}^{2\pi} = 0.
 				\end{align*}
 			\section*{Question 6}
 				The ellipse that results from the intersection of the cylinder and the plane is parameterized as
 				\begin{align*}
 				C(t) &= (\cos t, \sin t, 1-\cos t - \sin t) \\
 				\implies C'(t) &= (-\sin t, \cos t, \sin t - \cos t).
 				\end{align*}
 				Then using the definition of line integral we get 
 				\begin{align*}
 				\int\limits_{C} -y^3 dx + x^3 dy - zdz &= \int_{0}^{2\pi} F(C(t)) \cdot C'(t) \ dt \\
 				&= \int_{0}^{2\pi}(\sin^{3}t , \cos^{3}t, -1+\cos t + \sin t) \ dt \\
 				&\quad \cdot (-\sin t, \cos t , \sin t - \cos t) \ dt \\
 				&= \int_{0}^{2\pi} (1-2\sin^{2} t \cos^{2} t)-\sin t+\cos t -  \cos(2t) \ d t\\
 				&=\int_{0}^{2\pi} 1 \ dt  - 2\int_{0}^{2\pi}\frac18 (1-\cos(4t)) \ dt - \int_{0}^{2\pi} \sin t +0 - 0\\
 				&= 2\pi - \frac{\pi}{2} -0 = \frac{3\pi}{2}.
 				\end{align*}
 			\section*{Question 7}
 				$$ \text{div} F = \left(\pdv{F_{1}}{x} + \pdv{F_{2}}{y} + \pdv{F_{3}}{z}\right) = 3y^{2} + 3x^{2} + 3z^{3} = 3(y^{2} + x^{2} +z^{2}).$$
 				Let us use spherical coordinates 
 				$$ x = \rho \sin \varphi \cos \theta \quad, y = \rho\sin\varphi \sin\theta \quad, z = \rho\cos\varphi. $$
 				We get ,
 				\begin{align*}
 					\int\limits_{S} \vec{F} \cdot ds  &= 3 \int_{0}^{\pi}\int_{0}^{2\pi} \int_{0}^{1} (\rho^{2} \sin^{2}\varphi \sin^{2} \theta + \rho^{2} \sin^{2}\varphi \rho  \cos^{2}\theta + \rho^{2} \cos^{2} \varphi ) \rho^{2} \sin\varphi \ d\rho d\theta d\varphi \\
 					&= 3 \int_{0}^{\pi}\int_{0}^{2\pi} \int_{0}^{1} \rho^{4} (\sin^{3} \varphi (\sin^{2} \theta + \cos^{2}\theta) + \cos^{2}\varphi \sin\varphi) \ d\rho d\theta d\varphi \\
 					&= \frac35 \int_{0}^{\pi} \int_{0}^{2\pi} \sin^{3}\varphi  + \sin\varphi \cos^{2}\varphi \ d\theta d\varphi \\
 					&= \frac35 \int_{0}^{\pi} \int_{0}^{2\pi} \sin\varphi \ d\theta d\varphi \\
 					&= \frac{6\pi}{5} \int_{0}^{\pi} \sin\varphi \ d\varphi = \frac{12\pi}{5}.
 				\end{align*}
	\end{document}