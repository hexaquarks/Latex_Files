\documentclass[12pt]{article}
\newcommand\hmmax{0}
\newcommand\bmmax{0}
\usepackage{xcolor}
\usepackage[dvipsnames]{xcolor}
\usepackage[many]{tcolorbox}
\usepackage{changepage}
\usepackage{titlesec}
\usepackage{caption}
\usepackage{mdframed, longtable}
\usepackage{mathtools, amssymb, amsfonts, amsthm, bm,amsmath} 
\usepackage{array, tabularx, booktabs}
\usepackage{graphicx,wrapfig, float, caption}
\usepackage{tikz,physics,cancel, siunitx, xfrac}
\usepackage{graphics, fancyhdr}
\usepackage{lipsum}
\usepackage{xparse}
\usepackage{thmtools}
\usepackage{mathrsfs}
\usepackage{undertilde}
\usepackage{tikz}
\usepackage{fullpage,enumitem}
\usepackage[labelfont=bf]{caption}
\newcommand{\td}{\text{dim}}
\newcommand{\tvw}{T : V\xrightarrow{} W }
\newcommand{\ttt}{\widetilde{T}}
\newcommand{\ex}{\textbf{Example}}
\newcommand{\aR}{\alpha \in \mathbb{R}}
\newcommand{\abR}{\alpha \beta \in \mathbb{R}}
\newcommand{\un}{u_1 , u_2 , \dots , n}
\newcommand{\an}{\alpha_1, \alpha_2, \dots, \alpha_2 }
\newcommand{\sS}{\text{Span}(\mathcal{S})}
\newcommand{\sSt}{($\mathcal{S}$)}
\newcommand{\la}{\langle}
\newcommand{\ra}{\rangle}
\newcommand{\Rn}{\mathbb{R}^{n}}
\newcommand{\R}{\mathbb{R}}
\newcommand{\Rm}{\mathbb{R}^{m}}
\usepackage{fullpage, fancyhdr}
\newcommand{\La}{\mathcal{L}}
\newcommand{\ep}{\epsilon}
\newcommand{\de}{\delta}
\usepackage[math]{cellspace}
\setlength{\cellspacetoplimit}{3pt}
\setlength{\cellspacebottomlimit}{3pt}
\newcommand\numberthis{\addtocounter{equation}{1}\tag{\theequation}}
\usepackage{newtxtext, newtxmath}
\usepackage{bbm, aligned-overset}


\usepackage{mathtools}
\DeclarePairedDelimiter{\norm}{\lVert}{\rVert}
\newcommand{\vectorproj}[2][]{\textit{proj}_{\vect{#1}}\vect{#2}}
\newcommand{\vect}{\mathbf}
\newcommand{\uuuu}{\sum_{i=1}^{n}\frac{<u,u_i}{<u_i,u_i>} u_i}
\newcommand{\Ss}{\mathcal{S}}
\newcommand{\A}{\hat{A}}
\newcommand{\B}{\hat{B}}
\newcommand{\C}{\hat{C}}
\newcommand{\dr}{\mathrm{d}}
\allowdisplaybreaks
\usepackage{titling}
\newtheorem{theorem}{Theorem}[section]
\theoremstyle{definition}
\newtheorem{corollary}{Corollary}[theorem]
\theoremstyle{definition}
\newtheorem{lemma}[theorem]{Lemma}
\theoremstyle{definition}
\newtheorem{definition}{Definition}[section]
\theoremstyle{definition}
\newtheorem{Proposition}{Proposition}[section]
\theoremstyle{definition}
\newtheorem*{example}{Example}
\theoremstyle{example}
\newtheorem*{note}{Note}
\theoremstyle{note}
\newtheorem*{remark}{Remark}
\theoremstyle{remark}
\newtheorem*{example2}{External Example}
\theoremstyle{example}
\usepackage{bbold}
\title{PHYS358 Assignment 8}
\titleformat*{\section}{\LARGE\normalfont\fontsize{14}{14}\bfseries}
\titleformat*{\subsection}{\Large\normalfont\fontsize{12}{15}\bfseries}
\author{Mihail Anghelici 260928404 }
\date{\today}

\relpenalty=9999
\binoppenalty=9999

\renewcommand{\sectionmark}[1]{%
	\markboth{\thesection\quad #1}{}}

\fancypagestyle{plain}{%
	\fancyhf{}
	\fancyhead[L]{\rule[0pt]{0pt}{0pt} Assignment 8} 
	\fancyhead[R]{\small Mihail Anghelici $260928404$} 
	\fancyfoot[C]{-- \thepage\ --}
	\renewcommand{\headrulewidth}{0.4pt}}
\pagestyle{plain}
\setlength{\headsep}{1cm}
\captionsetup{margin =1cm}
	\begin{document}
	\maketitle
			\section*{Question 1}
				\subsection*{a) }
					We normalize the state and thereby determine $A$ 
					\begin{align*}
						1 = \int_{- \infty}^{\infty} \dr x \braket{\psi}{x} \braket{x}{\psi} &= \int_{- \infty}^{\infty} \dr x \abs{\psi(x)}^{2} \\ \implies 1 &= \abs{A}^{2} \int_{-R}^{R} \dr x (R^{2} -x^{2}) \\
						&= \abs{A}^{2} \left(x\Big|_{-R}^{R} - 2R^{2} \Bigg[ \frac{x^{3}}{r}\Big|_{-R}^{R} + \Bigg[ \frac{x^{5}}{5} \Big|_{-R}^{R}\right) \\
						&= \abs{A}^{2} \frac{16}{5} R^{5} 
						\shortintertext{\[
								\therefore A = \sqrt{\frac{5}{16 R^{5}}}.
							\]}
					\end{align*}
				\subsection*{b) }
					By definition, 
					$$ \la \hat{x} \ra = \mel{\psi}{\hat{x}}{\psi} = \int \dr x \braket{\psi}{x} x  \braket{x}{\psi} = \int \dr x  x \abs{\psi(x)}^{2}.$$
					Thus we compute 
					\begin{align*}
						 \la \hat{x} \ra = \int \dr x A^{2}  (R^{2} - x^{2})^{2} &= \int \dr x \frac{5}{16 R^{5}} (R^{4} - 2R^{2}x^{2} + x^{4}) \\
						 &= \frac{5}{16} \frac{x}{R} \Big|_{-R}^{R} - \frac{2}{3R^{3}} \Big|_{-R}^{R} + \frac{x^{5}}{5r^{5}} \Big|_{-R}^{R}\\ 
						 &= \frac13 
					\end{align*}
				\subsection*{c) }
					We us the definition, 
					$$ \psi(p) = \frac{1}{\sqrt{2 \pi \hbar}} \int_{-R}^{R} A (R^{2} - x^{2}) e^{\frac{-i p x}{\hbar }} \dr x,$$ 
					solving this integral in Mathematica yields 
					$$\psi (p) = \sqrt{ \frac{15 \hbar^{3}}{2 \pi R^{5}}} \frac{1}{p^{3}} \left( \hbar \sin \left(\frac{pR}{\hbar}\right) - pR \cos \left(\frac{pR}{\hbar }\right)\right).$$
					So then by definition ,
					\begin{align*}
						 \ket{\psi} &= \int \dr P \ket{p}\braket{p}{\psi} \\
						 &=  \sqrt{ \frac{15 \hbar^{3}}{2 \pi R^{5}}} \int_{-\infty}^{\infty} \dr P \ \frac{1}{p^{3}} \left( \hbar \sin \left(\frac{pR}{\hbar}\right) - pR \cos \left(\frac{pR}{\hbar }\right)\right) \ket{p}
					\end{align*}
				\subsection*{d) }	
					\begin{align*} 
					\la \hat{p} \ra = \mel{\psi}{p}{p} \braket{p}{\psi} &= \int \dr P p \abs{\psi(p)^{2}} \\
					&= \int_{-R}^{R} \dr P \sqrt{\frac{15 \hbar^{3}}{2 \pi R^{4}}} \frac{1}{p^{2}} \left( \hbar \sin \left(\frac{pR}{\hbar}\right) - pR \cos \left(\frac{pR}{\hbar }\right)\right)
					\end{align*}
					Solving this integral in Mathematica yields 
					$$ \la \hat{p} \ra = 0.$$
				\subsection*{e) }
					Since we're studying a free-particle, the Hamiltonian is 
					$$ \hat{H} = \frac{\hat{p}^{2}}{2m},$$ 
					so then by definition . 
					$$ \psi(x,t) = \int \dr P e^{\frac{i}{\hbar} \left(px - \frac{p^{2}}{2m}\right)} \psi(p, t=0) = \int_{-\infty}^{\infty} e^{\frac{i}{\hbar} \left(px - \frac{p^{2}}{2m}\right) } \sqrt{\frac{15 \hbar^{3}}{2 \pi R^{5}}} \frac{1}{p^{3}} \left( \hbar \sin \left(\frac{pR}{\hbar}\right) - pR \cos \left(\frac{pR}{\hbar }\right)\right). $$
			\section*{Question 2}
				\subsection*{a) }
					Using the normalization and splitting the absolute value, 
					\begin{align*}
						1 =\int_{-\infty}^{\infty} \abs{\psi (x)}^{2} \dr x &= \abs{A}^{2} \int_{-R}^{0} (R + x)^{2} \dr x + \int_{0}^{R} (R -x )^{2} \dr x \\
						&= \abs{A}^{2} \Biggl\{ \Big[ R^{2} x\Big|_{-R}^{0} + \Big[\frac{2R}{2} x^{2}\Big|_{-R}^{0} + \Big[\frac{1}{3} x^{3} \Big|_{-R}^{0} + \Big[ R^{2}x\Big|_{0}^{R} - \Big[\frac{2R}{2} x^{2} \Big|_{0}^{R} + \Big[ \frac{1}{3} x^{3} \Big|_{0}^{R} \Biggr\}\\
						&= \abs{A}^{2} \frac23 R^{3} \implies A = \sqrt{\frac{3}{2 R^{3}}}.
					\end{align*}
				\subsection*{b) }
					We apply the definition, 
					\begin{align*}
						 \la \hat{x} \ra &= \int \dr x \abs{\psi(x)}^{2} \\
						 &= \abs{A}^{2} \left( \int_{-R}^{0} \dr x \ x (R + x)^{2} +\int_{0}^{R}  \dr x\  x (R -x)^{2} \right) 
						 \intertext{This is an integegral of an odd function over a symmetric interval, so this integrates to $0$.}
						 &= 0.
					\end{align*}
				\subsection*{c) }
					\begin{align*}
						\psi(p) &= \frac{1}{\sqrt{2 \pi \hbar }} \int_{-R}^{R} (R - \abs{x}) e^{-\frac{i px}{\hbar}} \dr x \\
						&= \frac{1}{\sqrt{2 \pi \hbar} } \int_{-R}^{R} Re^{-\frac{i px}{\hbar}} \dr x - \int_{-R}^{0} (-x) e^{-\frac{ipx}{\hbar }} \dr x - \int_{0}^{R} xe^{-\frac{ipx }{\hbar }} \dr x
						\intertext{Solving this integral with Mathematica yields}
						&= \sqrt{\frac{3 \hbar^{3}}{4 \pi R^{3} p^{4}}} \left(1 - e^{\frac{i pR}{\hbar }}\right) \left(-e^{-\frac{i pR}{\hbar }}\right).
					\end{align*}
					So then by definition ,
					\begin{align*}
						\ket{\psi} &= \int \dr P \ket{p}\braket{p}{\psi} \\
						&= \sqrt{\frac{3 \hbar^{3}}{4 \pi R^{3} }} \int_{-\infty}^{\infty} \dr P \frac{1}{p^{2}} \left(1 - e^{\frac{i pR}{\hbar }}\right) \left(-e^{-\frac{i pR}{\hbar }}\right) \ket{p}.  
					\end{align*}
					for $\psi(p)$ as defined above.
				\subsection*{d) }
					By definition , 
					$$ \la \hat{p} \ra = \int \dr P \ p \abs{\psi(p)}^{2}  = \int_{-R}^{R} \dr P \ p \sqrt{\frac{3 \hbar^{3}}{4 \pi R^{3} p^{4}}} \left(1 - e^{\frac{i pR}{\hbar }}\right) \left(-e^{-\frac{i pR}{\hbar }}\right) ,$$
					solving this integral with Mathematica yields
					$$ \la \hat{p} \ra = 0.$$
				\subsection*{e) }
					We apply the definition as before 
					$$ \psi(x,t) = \frac{1}{\sqrt{2 \pi \hbar }} \int_{-\infty}^{\infty} \sqrt{\frac{3 \hbar^{3}}{4 \pi R^{3} p^{4}}} \left(1 - e^{\frac{i pR}{\hbar }}\right) \left(-e^{-\frac{i pR}{\hbar }}\right) e^{\frac{i}{\hbar} \left(px - \frac{p^{2}}{2m}\right)} \dr P .$$
				
	\end{document}