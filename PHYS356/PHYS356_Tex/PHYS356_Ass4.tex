\documentclass[12pt]{article}
		\usepackage{xcolor}
			\usepackage[dvipsnames]{xcolor}
			\usepackage[many]{tcolorbox}
		\usepackage{changepage}
		\usepackage{titlesec}
		\usepackage{caption}
		\usepackage{mdframed, longtable}
		\usepackage{mathtools, amssymb, amsfonts, amsthm, bm,amsmath} 
		\usepackage{array, tabularx, booktabs}
		\usepackage{graphicx,wrapfig, float, caption}
		\usepackage{tikz,physics,cancel, siunitx, xfrac}
		\usepackage{graphics, fancyhdr}
		\usepackage{lipsum}
		\usepackage{xparse}
		\usepackage{thmtools}
		\usepackage{mathrsfs}
		\usepackage{undertilde}
		\usepackage{tikz}
		\usepackage{fullpage,enumitem}
		\usepackage[labelfont=bf]{caption}
	\newcommand{\td}{\text{dim}}
	\newcommand{\tvw}{T : V\xrightarrow{} W }
	\newcommand{\ttt}{\widetilde{T}}
	\newcommand{\ex}{\textbf{Example}}
	\newcommand{\aR}{\alpha \in \mathbb{R}}
	\newcommand{\abR}{\alpha \beta \in \mathbb{R}}
	\newcommand{\un}{u_1 , u_2 , \dots , n}
	\newcommand{\an}{\alpha_1, \alpha_2, \dots, \alpha_2 }
	\newcommand{\sS}{\text{Span}(\mathcal{S})}
	\newcommand{\sSt}{($\mathcal{S}$)}
	\newcommand{\la}{\langle}
	\newcommand{\ra}{\rangle}
	\newcommand{\Rn}{\mathbb{R}^{n}}
	\newcommand{\R}{\mathbb{R}}
	\newcommand{\Rm}{\mathbb{R}^{m}}
	\usepackage{fullpage, fancyhdr}
	\newcommand{\La}{\mathcal{L}}
	\newcommand{\ep}{\epsilon}
	\newcommand{\de}{\delta}
	\usepackage[math]{cellspace}
		\setlength{\cellspacetoplimit}{3pt}
		\setlength{\cellspacebottomlimit}{3pt}
	\newcommand\numberthis{\addtocounter{equation}{1}\tag{\theequation}}
	\usepackage{newtxtext, newtxmath}


	\usepackage{mathtools}
	\DeclarePairedDelimiter{\norm}{\lVert}{\rVert}
	\newcommand{\vectorproj}[2][]{\textit{proj}_{\vect{#1}}\vect{#2}}
	\newcommand{\vect}{\mathbf}
	\newcommand{\uuuu}{\sum_{i=1}^{n}\frac{<u,u_i}{<u_i,u_i>} u_i}
	\newcommand{\Ss}{\mathcal{S}}
	\newcommand{\A}{\hat{A}}
	\newcommand{\B}{\hat{B}}
	\newcommand{\C}{\hat{C}}
	\allowdisplaybreaks
	\usepackage{titling}
	\newtheorem{theorem}{Theorem}[section]
	\theoremstyle{definition}
	\newtheorem{corollary}{Corollary}[theorem]
	\theoremstyle{definition}
	\newtheorem{lemma}[theorem]{Lemma}
	\theoremstyle{definition}
	\newtheorem{definition}{Definition}[section]
	\theoremstyle{definition}
	\newtheorem{Proposition}{Proposition}[section]
	\theoremstyle{definition}
	\newtheorem*{example}{Example}
	\theoremstyle{example}
	\newtheorem*{note}{Note}
	\theoremstyle{note}
	\newtheorem*{remark}{Remark}
	\theoremstyle{remark}
	\newtheorem*{example2}{External Example}
	\theoremstyle{example}
	
	\title{PHYS 350 Assignment 4}
	\titleformat*{\section}{\LARGE\normalfont\fontsize{14}{14}\bfseries}
	\titleformat*{\subsection}{\Large\normalfont\fontsize{12}{15}\bfseries}
	\author{Mihail Anghelici 260928404 }
	\date{\today}
	
	\relpenalty=9999
			\binoppenalty=9999
		
			\renewcommand{\sectionmark}[1]{%
			\markboth{\thesection\quad #1}{}}
			
			\fancypagestyle{plain}{%
			  \fancyhf{}
			  \fancyhead[L]{\rule[0pt]{0pt}{0pt} Assignment 4} 
			  \fancyhead[R]{\small Mihail Anghelici $260928404$} 
			  \fancyfoot[C]{-- \thepage\ --}
			  \renewcommand{\headrulewidth}{0.4pt}}
			\pagestyle{plain}
			\setlength{\headsep}{1cm}
	\captionsetup{margin =1cm}
	\begin{document}
	\maketitle
		\section*{Question 1}
			\subsection*{a) }
				\subsection*{i)}	
					\begin{gather*}
						[\A, [\B, \C]] + [\B , [\C, \A]] + [\C , [\A,\B]] =0 
						\intertext{We use the Lemma $[\A, [\B,\C]] = ABC - ACB -BCA +CBA$ , we get}
						 \begin{split}
							&=ABC-ACB-BCA+CBA +BCA-BAC-CAB \\
							&+ACB +CAB - CBA - ABC +BAC \\
							&=0 \qquad \checkmark.
						\end{split}
					\end{gather*}	
				\subsection*{ii) }
					We use the dagger properties,
						\[\begin{split}
							(\A,\B)^{\dagger} &= (AB - BA)^{\dagger} = (AB)^{\dagger} - (BA)^{\dagger} \\
							&= (B^{\dagger}A^{\dagger} ) - (B^{\dagger}A^{\dagger})= (BA)^{\dagger} -(BA)^{\dagger} = (BA-AB)^{\dagger} = [\B^{\dagger} , \A^{\dagger}] 
						\end{split}\]
			\subsection*{b) }
			First we show $[\A , \B^{n}] = n\B^{n-1}[\A,\B]$.
			\begin{align*}
				[\A,\B^{n}] &= \A \B^{n} - \B^{n} \A \\
				&= [\A,\B]\B^{n-1} + \B[\A,\B^{n-1}] \\
				&= [\A,\B]\B^{n-1} + \B[\A,\B] (n-1)\B^{n-2} \\
				&=[\A,\B](\B^{n-1} +(n-1)\B^{n-1})\\
				&= [\A,\B]n\B^{n-1}
				\intertext{$[\A,\B]n = n[\A,\B]$ so then finally,}
			\end{align*}
			\begin{equation}
				\implies [\A,\B^{n}] = n[\A,\B]\B^{n-1}.
			\end{equation}
			Then we show the claim.
				\begin{align*}
					[\A , F(\B)] &= \A F(\B) - F(\B) \A \\
					&=
						\A \left(I - \left(\frac{i}{\hbar} \B\right) + \frac{1}{4} \left(\frac{i}{\hbar} \B\right)^{2}- \frac{1}{24} \left(\frac{i}{\hbar}\B\right)^{3} + \dots \right) \\
						&\quad -  \left(I - \left(\frac{i}{\hbar} \B\right) + \frac{1}{4} \left(\frac{i}{\hbar} \B\right)^{2}- \frac{1}{24} \left(\frac{i}{\hbar}\B\right)^{3} + \dots \right)\A  \\
						&= 
						\left(\A - \left(\frac{i}{\hbar}\right) \A\B + \frac{1}{4} \left(\frac{i}{\hbar}\right)^{2} \A\B^{2} - \frac{1}{24} \left(\frac{i}{\hbar}\right)^{3}\A\B^{3} + \dots \right) \\
						&\quad - \left(\A - \left(\frac{i}{\hbar}\right) \B\A + \frac{1}{4} \left(\frac{i}{\hbar}\right)^{2} \B^{2}\A- \frac{1}{24} \left(\frac{i}{\hbar}\right)^{3} \B^{3}\A + \dots \right)  \\
						&= \frac{-i}{\hbar} [\A, \B] + \frac14 \left(\frac{i}{\hbar}\right)^{2} [\A, \B^{2}] - \frac{1}{24} \left(\frac{i}{\hbar}\right)^{3} [\A , \B^{3}] + \dots
						\intertext{We use the relationship $[\A,\B^{n}] = cn\B^{n-1}$ proved in $(1) which means that the commutator between $[\A, \B^{n}]$ is the derivative with respect to $\B$ acting on $\B^{n}$. } 
						&= c\left(\frac{-i}{\hbar} + \frac{1}{4} \left(\frac{i}{\hbar}\right)^{2} 2 \B - \frac{1}{24} \left(\frac{i}{\hbar}\right)^{3}  3 \B^{2} + \dots\right)\\
						&= cF'(\B)
						\intertext{For $F'(\B)$ defined as $\displaystyle F'(\B) = \sum_{n=1}^{\infty} \left(\frac{1}{(n-1)!}\right) \left(\frac{-i}{\hbar} \B \right)^{n-1}$, i.e., the derivative applied to an exponential with a matrix argument.}
				\end{align*}
			\subsection*{c) }
				First of all we note that since $[\A, \B]$ commutes with both $\A$ and $\B$ , then
				%TODO $$ $$
				We perform the exact proceedure outlined in $1b$.
					\begin{align*}
				[\A , F(\B)] &= \A F(\B) - F(\B) \A \\
				&=
				\A \left(I - \left(\frac{i}{\hbar} \B\right) + \frac{1}{4} \left(\frac{i}{\hbar} \B\right)^{2}- \frac{1}{24} \left(\frac{i}{\hbar}\B\right)^{3} + \dots \right) \\
				&\quad - \left(I - \left(\frac{i}{\hbar} \B\right) + \frac{1}{4} \left(\frac{i}{\hbar} \B\right)^{2}- \frac{1}{24} \left(\frac{i}{\hbar}\B\right)^{3} + \dots \right) \A  \\
				&= 
				\left(\A - \left(\frac{i}{\hbar}\right) \A\B + \frac{1}{4} \left(\frac{i}{\hbar}\right)^{2} \A\B^{2} - \frac{1}{24} \left(\frac{i}{\hbar}\right)^{3}\A\B^{3} + \dots \right) \\
				&\quad - \left(\A - \left(\frac{i}{\hbar}\right) \B\A + \frac{1}{4} \left(\frac{i}{\hbar}\right)^{2} \B^{2}\A- \frac{1}{24} \left(\frac{i}{\hbar}\right)^{3}\B^{3}\A + \dots \right)  \\
				&= \frac{-i}{\hbar} [\A, \B] + \frac14 \left(\frac{i}{\hbar}\right)^{2} [\A, \B^{2}] - \frac{1}{24} \left(\frac{i}{\hbar}\right)^{3} [\A , \B^{3}]  + \dots 
				\intertext{We use the relationship $[\A,\B^{n}] = n\C\B^{n-1}$. Indeed, since $\A$ and $\B$ commutes with $[\A,\B]$, then it follows that $[\A,\B] = AB-BA \implies AB=BA$. Therefore, $[\A,\B^{n}] = [\A,\B]\B^{n-1} - \B^{n-1}[\A,\B] \implies [\A,\B] \B^{n-1} = \B^{n-1}[\A,\B]$. We conclude that $[\A,\B^{n}] = n\C \B^{n-1}$ holds for non-integer commutator. }
				&= \C\left(\frac{-i}{\hbar} + \frac{1}{4} \left(\frac{i}{\hbar}\right)^{2} 2 \B - \frac{1}{24} \left(\frac{i}{\hbar}\right)^{3}  3 \B^{2} + \dots\right)\\
				&= \C F'(\B)
					\intertext{For $F'(\B)$ defined as $\displaystyle F'(\B) = \sum_{n=1}^{\infty} \left(\frac{1}{(n-1)!}\right) \left(\frac{-i}{\hbar} \B \right)^{n-1}$, i.e., the derivative applied to an exponential with a matrix argument.}
				\end{align*}
			\subsection*{d) }
				Let $\hat{\beta}(x) \equiv e^{x\A}e^{x\B}.$ Then, 
				\begin{align*}
					\dv{}{x} \left(\hat{\beta}(x)\right) &= \dv{}{x}\left(e^{x\A}e^{ x \B} \right) \\
					&= \A e^{x\A} e^{x\B} + e^{x\A} \B e^{x\B} \\
					&= \A e^{x\A} e^{x\B} + [e^{x\A},\B] e^{x\B} + Be^{x\A} e^{x\B} \\
					&=(\A+\B) e^{x\A}e^{x\B} + [e^{x\A},\B]e^{x\B} 
					\intertext{Let $F(x\A) \equiv e^{x\A}$, then as found before $[F'(x\A) ,\B] = xF(x\A)[\A,\B]$, it follows that}
					&=(\A+\B)e^{x\A}e^{x\B} + \left(xe^{x\A} [\A,\B]\right)e^{x\B} \\
					&= \left(\A+\B + x[\A,\B]\right) \hat{\beta}(x) 
				\end{align*}
				\begin{gather*}
					\implies \int \frac{\hat{\beta}'(x)}{\hat{\beta}(x)} \ dx  = \int \A + \B + x[\A,\B] \ dx \\
					Ce^{\ln \hat{\beta}(x)} = e^{\A x + \B x + \frac{x^{2}}{2} [\A, \B]} 
					\intertext{We set $x\equiv 1$ , then $C$ vanishes since it's an integration constant with respect to $x$, we're left off with the requested claim}
					\hat{\beta}(1) = e^{\A}e^{\B} = e^{\A + \B +\frac12 [\A,\B]}\qquad \checkmark.
				\end{gather*}
		\section*{Question 2}
			$\bullet$ For the first part , let $\ket{\varphi} = \ket{+z}.$ Then 
			$$ \ket{+z} \xrightarrow{x \ \text{ basis}} \begin{pmatrix}
				\braket{+x}{+z} \\ \braket{-x}{+z}
			\end{pmatrix} = \frac{1}{\sqrt{2}} \begin{pmatrix}
				1 \\ 1
			\end{pmatrix},$$
			therefore 
			$$ \mel{\varphi}{S_{x}}{\varphi} = \frac{1}{\sqrt{2}} \begin{pmatrix}
				1 & 1
			\end{pmatrix} \frac{\hbar}{2} \begin{pmatrix}
				0 & 1 \\
				1 &  0
			\end{pmatrix}\frac{1}{\sqrt{2}} \begin{pmatrix}
			1 \\ 1
			\end{pmatrix} = \frac{\hbar}{2}.$$
			We conclude that $\Delta S_{x} = 0.$
			 Similarly for $\la S_{y} \ra$ , 
			$$ \ket{+z} \xrightarrow{y \ \text{ basis}} \begin{pmatrix}
			\braket{+y}{+z} \\ \braket{-y}{+z}
			\end{pmatrix} = \frac{1}{\sqrt{2}} \begin{pmatrix}
			1 \\ 1
			\end{pmatrix},$$
			therefore 
			 $$ \mel{\varphi}{S_{x}}{\varphi} = \frac{1}{\sqrt{2}} \begin{pmatrix}
			 1 & 1
			 \end{pmatrix} \frac{\hbar}{2} \begin{pmatrix}
			 0 & -i \\
			 i &  0
			 \end{pmatrix}\frac{1}{\sqrt{2}} \begin{pmatrix}
			 1 \\ 1
			 \end{pmatrix} = 0.$$
			 We conclude that $\Delta S_{y} = \hbar/2.$
			 We expect $\la S_{z} \ra =0$. 
			 $$ \mel{\varphi}{S_{z}}{\varphi} = \frac{1}{\sqrt{2}} \begin{pmatrix}
			 1 & 1
			 \end{pmatrix} \frac{\hbar}{2} \begin{pmatrix}
			 1 & 0 \\
			 0 & -1
			 \end{pmatrix}\frac{1}{\sqrt{2}} \begin{pmatrix}
			 1 \\ 1
			 \end{pmatrix} = 0 \qquad \checkmark.$$
			
			\noindent$\bullet$ Similarly for eigenstates of $\hat{S}_{x}$. Let $\ket{\varphi} = \ket{+x}$.Then 
			$$ \mel{\varphi}{S_{x}}{\varphi} = \frac{1}{\sqrt{2}}\begin{pmatrix}
			1 & 1
			\end{pmatrix} \frac{\hbar}{2} \begin{pmatrix}
				0 & 1 \\ 1 & 0
			\end{pmatrix} \frac{1}{\sqrt{2}} \begin{pmatrix}
				1 \\ 1
			\end{pmatrix} = \frac{\hbar}{2},$$
			we conclude that $\Delta S_{x} = 0$. Then similarly, 
			$$ \ket{+x} \xrightarrow{y \ \text{ basis}} = \begin{pmatrix}
				\braket{+y}{+x} \\\braket{+y}{-x}
			\end{pmatrix} = \frac12 \begin{pmatrix}
				1 - i \\ 1+ i
			\end{pmatrix}. $$
			Thus,
			$$ \mel{\varphi}{S_{y}}{\varphi} = \frac{1}{\sqrt{2}} \begin{pmatrix}
				 1+i & 1-i
			\end{pmatrix} \frac{\hbar}{2} \begin{pmatrix}
			1 0 & -1 \\i & 0
			\end{pmatrix}\frac{1}{\sqrt{2}} \begin{pmatrix}
			1 - i \\ 1+i
			\end{pmatrix} = \frac{\hbar}{2},$$
			we conclude that $\Delta S_{y} = 0$. We finally expect $\la S_{z} \ra = 0$
			$$ \mel{\varphi}{S_{z}}{\varphi} = \frac{1}{\sqrt{2}} \begin{pmatrix}
				 1 & 1
			\end{pmatrix} \frac{\hbar}{2} \begin{pmatrix}
				1 & 0 \\ 0 &-1 
			\end{pmatrix} \frac{1}{\sqrt{2}} \begin{pmatrix}
			1 \\ 1 
			\end{pmatrix} = 0 \qquad \checkmark.$$
		\section*{Question 3}
			By definition the eigenbasis of $S_{z}$ is $(1 \ 0  \ 0 )^{T} , (0 \ 1 \ 0 )^{T} $ and $(0 \ 0 \ 1)^{T}$ since $z$ is the default basis direction. Then we convert the transformation matrix 
			$$ \hat{S}_{x} = \frac{\hat{S}_{+} - \hat{S}_{-}}{2} =\overset{Notes}{=}  \frac{\hbar}{\sqrt{2}} \begin{pmatrix}
				 0 & 1 & 0 \\ 1 & 0 & 1\\ 0 & 1 & 0
			\end{pmatrix}.$$
			By definition, the eigenvalues must be (ignoring the scaling constants) $$\lambda_{1} =1, \qquad \lambda_{2} = 0, \qquad \lambda_{3} = -1.$$			
			We look for the eigenvectors
			\begin{equation*}
				\lambda_{1} =1 \implies \hbar\begin{pmatrix}
				-1 & 1 & 0 \\
				1 & -1 & 0 \\
				0 & 1 & -1
				\end{pmatrix}\begin{pmatrix}
				a \\ b \\ c
				\end{pmatrix} = \begin{pmatrix}
				0 \\ 0 \\0 
				\end{pmatrix} \implies \begin{cases}
				&a-c = 0 \\
				&b- \sqrt{2}c = 0
				\end{cases} \implies \boldsymbol{v_{1}} = \begin{pmatrix}
				1 \\ \sqrt{2} \\ 1
				\end{pmatrix}
			\end{equation*}  
			\begin{equation*}
			\lambda_{2} =0 \implies \hbar\begin{pmatrix}
			0 & 1 & 0 \\
			1 & 0 & 0 \\
			0 & 1 & 0
			\end{pmatrix}\begin{pmatrix}
			a \\ b \\ c
			\end{pmatrix} = \begin{pmatrix}
			0 \\ 0 \\0 
			\end{pmatrix} \implies \begin{cases}
			&a+c = 0 \\
			&b= 0
			\end{cases} \implies \boldsymbol{v_{2}} = \begin{pmatrix}
			-1 \\ 0 \\ 1
			\end{pmatrix}
			\end{equation*}  
			\begin{equation*}
			\lambda_{3} =-1 \implies \hbar\begin{pmatrix}
			1 & 1 & 0 \\
			1 & 1 & 0 \\
			0 & 1 & 1
			\end{pmatrix}\begin{pmatrix}
			a \\ b \\ c
			\end{pmatrix} = \begin{pmatrix}
			0 \\ 0 \\0 
			\end{pmatrix} \implies \begin{cases}
			&c-a = 0 \\
			&b+ \sqrt{2}c = 0
			\end{cases} \implies \boldsymbol{v_{1}} = \begin{pmatrix}
			1 \\ -\sqrt{2} \\ 1
			\end{pmatrix}
			\end{equation*} 
			Finally, we may express the eigenstates normalized in Ket notation. 
			\begin{align*}
				\ket{1,1}_{x} & = \frac12 \ket{1, 1}_{z} + \frac{1}{\sqrt{2}} \ket{1,0}_{z} + \frac12 \ket{1,-1}_{z} \\
				\ket{1,0}_{x} &= -\frac{1}{\sqrt{2}} \ket{1,1}_{z} + \frac{1}{\sqrt{2}} \ket{1,-1}_{z} \\
				\ket{1,-1}_{x} &= \frac12 \ket{1,1}_{z} -\frac{1}{\sqrt{2}} \ket{1,0}_{z} + \frac12 \ket{1,-1}_{x}
			\end{align*}
		\section*{Question 4}
			Following the results from Question $3$, 
			$$ \abs{_{x}\braket{1,0}{1,1}_{z}}^{2} = \abs{{_{z}\braket{1,1}{1,0}_{x}}^{\ast}}^{2} = \abs{-\frac{1}{\sqrt{2}}}^{2} = \frac12.$$
		\section*{Question 5}
			First and foremost, 
			$$ \hat{J}_{x} \overset{Notes }{=}  \frac{\hbar}{2} \begin{pmatrix}
				0 & \sqrt{3} & 0 & 0 \\
				\sqrt{3} & 0 & 2 & 0\\
				0 & 2 & 0 & \sqrt{3} \\
				0 & 0 & \sqrt{3} & 0
			\end{pmatrix}.$$
			For a spin $3/2$ particle, by definition the eigenvalues are (ignoring the scaling constants)
			$$ \lambda_{1} = 3 , \qquad \lambda_{2} = 1 , \qquad \lambda_{3} = -1 ,\qquad \lambda_{4} = -3.$$
			We look for the eigenvectors 
			\begin{equation*}
			\lambda_{1} =3 \implies \hbar\begin{pmatrix}
			-3 & \sqrt{3} & 0 & 0 \\
			\sqrt{3} & -3 & 2 & 0\\
			0 & 2 & -3 & \sqrt{3} \\
			0 & 0 & \sqrt{3} & -3
			\end{pmatrix}\begin{pmatrix}
			a \\ b \\ c \\ d
			\end{pmatrix} = \begin{pmatrix}
			0 \\ 0 \\0  \\0 
			\end{pmatrix} \implies \begin{cases}
			&d = \frac{c}{\sqrt{3}}  \\
			&b = c\\
			&a = \frac{b}{\sqrt{3}}
			\end{cases} \implies \boldsymbol{v_{1}} = \begin{pmatrix}
			1 \\ \sqrt{3} \\ \sqrt{3} \\ 1
			\end{pmatrix}
			\end{equation*}
			\begin{equation*}
			\lambda_{2} =1 \implies \hbar\begin{pmatrix}
			-1 & \sqrt{3} & 0 & 0 \\
			\sqrt{3} & -1 & 2 & 0\\
			0 & 2 & -1 & \sqrt{3} \\
			0 & 0 & \sqrt{3} & -1
			\end{pmatrix}\begin{pmatrix}
			a \\ b \\ c \\ d
			\end{pmatrix} = \begin{pmatrix}
			0 \\ 0 \\0  \\0 
			\end{pmatrix} \implies \begin{cases}
			&d = \sqrt{3}c  \\
			&b = -c\\
			&a = \sqrt{3}b
			\end{cases} \implies \boldsymbol{v_{2}} = \begin{pmatrix}
			-\sqrt{3} \\ -1 \\ 1 \\ \sqrt{3}
			\end{pmatrix}
			\end{equation*}    
			\begin{equation*}
			\lambda_{3} =-1 \implies \hbar\begin{pmatrix}
			1 & \sqrt{3} & 0 & 0 \\
			\sqrt{3} & 1 & 2 & 0\\
			0 & 2 & 1 & \sqrt{3} \\
			0 & 0 & \sqrt{3} & 1
			\end{pmatrix}\begin{pmatrix}
			a \\ b \\ c \\ d
			\end{pmatrix} = \begin{pmatrix}
			0 \\ 0 \\0  \\0 
			\end{pmatrix} \implies \begin{cases}
			&d = -\sqrt{3}c  \\
			&b = c\\
			&a = -\sqrt{3} b
			\end{cases} \implies \boldsymbol{v_3} = \begin{pmatrix}
			\sqrt{3} \\ -1 \\ -1 \\ \sqrt{3}
			\end{pmatrix}
			\end{equation*}  
			\begin{equation*}
			\lambda_{4} =-3 \implies \hbar\begin{pmatrix}
			3 & \sqrt{3} & 0 & 0 \\
			\sqrt{3} & 3 & 2 & 0\\
			0 & 2 & 3 & \sqrt{3} \\
			0 & 0 & \sqrt{3} & 3
			\end{pmatrix}\begin{pmatrix}
			a \\ b \\ c \\ d
			\end{pmatrix} = \begin{pmatrix}
			0 \\ 0 \\0  \\0 
			\end{pmatrix} \implies \begin{cases}
			&d = -\frac{c}{\sqrt{3}}  \\
			&b = -c\\
			&a = -\frac{b}{\sqrt{3}}
			\end{cases} \implies \boldsymbol{v_{1}} = \begin{pmatrix}
			-1 \\ \sqrt{3} \\ -\sqrt{3} \\ 1
			\end{pmatrix}
			\end{equation*} 
			We write the Ket notation for the state $S_{x} = \hbar / 2$, with the proper normalization factor
			$$ \ket{\frac32 , \frac12}_{x} = \frac{1}{\sqrt{8}} \left(-\sqrt{3} \ket{\frac32 ,\frac32}_{z} - \ket{\frac32 , \frac12}_{z} + \ket{\frac32 , -\frac12}_{z} +\sqrt{3}\ket{\frac32 , -\frac32}_{z}\right).$$
			It then follows that
			\begin{align*}
				 P_{z = \frac{3\hbar}{2}} &= \abs{_{z}\braket{\frac32 , \frac32}{\frac32 , \frac12}_{x}}^{2} = \abs{-\sqrt{\frac38}}^{2} = \frac38 \\
				 P_{z = \frac{\hbar}{2}} &= \abs{_{z}\braket{\frac32, \frac12}{ \frac32, \frac12}_{z}}^{2} = \abs{-\frac{1}{\sqrt{8}}}^{2} = \frac18 \\
				 P_{z = -\frac{\hbar}{2}} &= \abs{_{z}\braket{\frac32 , -\frac12}{\frac32 , \frac12}_{x}}^{2} = \abs{\frac{1}{\sqrt{8}}}^{2} = \frac18\\
				 P_{z = -\frac{3\hbar}{2}} &= \abs{_{z}\braket{\frac32 , -\frac32}{\frac32, \frac12}_{x}}^{2} = \abs{\sqrt{\frac{3}{8}}}^{2} = \frac38.
			\end{align*}
	\end{document}