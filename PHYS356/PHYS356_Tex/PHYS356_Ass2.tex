\documentclass[
	12pt,
	]{article}
		\usepackage{xcolor}
			\usepackage[dvipsnames]{xcolor}
			\usepackage[many]{tcolorbox}
		\usepackage{changepage}
		\usepackage{titlesec}
		\usepackage{caption}
		\usepackage{mdframed, longtable}
		\usepackage{mathtools, amssymb, amsfonts, amsthm, bm,amsmath} 
		\usepackage{array, tabularx, booktabs}
		\usepackage{graphicx,wrapfig, float, caption}
		\usepackage{tikz,physics,cancel, siunitx, xfrac}
		\usepackage{graphics, fancyhdr}
		\usepackage{lipsum}
		\usepackage{xparse}
		\usepackage{thmtools}
		\usepackage{mathrsfs}
		\usepackage{undertilde}
		\usepackage{tikz}
		\usepackage{fullpage,enumitem}
		\usepackage[labelfont=bf]{caption}
	\newcommand{\td}{\text{dim}}
	\newcommand{\tvw}{T : V\xrightarrow{} W }
	\newcommand{\ttt}{\widetilde{T}}
	\newcommand{\ex}{\textbf{Example}}
	\newcommand{\aR}{\alpha \in \mathbb{R}}
	\newcommand{\abR}{\alpha \beta \in \mathbb{R}}
	\newcommand{\un}{u_1 , u_2 , \dots , n}
	\newcommand{\an}{\alpha_1, \alpha_2, \dots, \alpha_2 }
	\newcommand{\sS}{\text{Span}(\mathcal{S})}
	\newcommand{\sSt}{($\mathcal{S}$)}
	\newcommand{\la}{\langle}
	\newcommand{\ra}{\rangle}
	\newcommand{\Rn}{\mathbb{R}^{n}}
	\newcommand{\R}{\mathbb{R}}
	\newcommand{\Rm}{\mathbb{R}^{m}}
	\usepackage{fullpage, fancyhdr}
	\newcommand{\La}{\mathcal{L}}
	\newcommand{\ep}{\epsilon}
	\newcommand{\de}{\delta}
	\usepackage[math]{cellspace}
		\setlength{\cellspacetoplimit}{3pt}
		\setlength{\cellspacebottomlimit}{3pt}
	\newcommand\numberthis{\addtocounter{equation}{1}\tag{\theequation}}


	\usepackage{mathtools}
	\DeclarePairedDelimiter{\norm}{\lVert}{\rVert}
	\newcommand{\vectorproj}[2][]{\textit{proj}_{\vect{#1}}\vect{#2}}
	\newcommand{\vect}{\mathbf}
	\newcommand{\uuuu}{\sum_{i=1}^{n}\frac{<u,u_i}{<u_i,u_i>} u_i}
	\newcommand{\B}{\mathcal{B}}
	\newcommand{\Ss}{\mathcal{S}}
	
	\newtheorem{theorem}{Theorem}[section]
	\theoremstyle{definition}
	\newtheorem{corollary}{Corollary}[theorem]
	\theoremstyle{definition}
	\newtheorem{lemma}[theorem]{Lemma}
	\theoremstyle{definition}
	\newtheorem{definition}{Definition}[section]
	\theoremstyle{definition}
	\newtheorem{Proposition}{Proposition}[section]
	\theoremstyle{definition}
	\newtheorem*{example}{Example}
	\theoremstyle{example}
	\newtheorem*{note}{Note}
	\theoremstyle{note}
	\newtheorem*{remark}{Remark}
	\theoremstyle{remark}
	\newtheorem*{example2}{External Example}
	\theoremstyle{example}
	
	\title{PHYS 357 Assignment 2}
	\titleformat*{\section}{\LARGE\normalfont\fontsize{12}{12}\bfseries}
	\titleformat*{\subsection}{\Large\normalfont\fontsize{10}{15}\bfseries}
	\author{Mihail Anghelici 260928404 }
	\date{\today}
	
	\relpenalty=9999
			\binoppenalty=9999
		
			\renewcommand{\sectionmark}[1]{%
			\markboth{\thesection\quad #1}{}}
			
			\fancypagestyle{plain}{%
			  \fancyhf{}
			  \fancyhead[L]{\rule[0pt]{0pt}{0pt} Assignment 2 } 
			  \fancyhead[R]{\small Mihail Anghelici $260928404$} 
			  \fancyfoot[C]{-- \thepage\ --}
			  \renewcommand{\headrulewidth}{0.4pt}}
			\pagestyle{plain}
			\setlength{\headsep}{1cm}
	\captionsetup{margin =1cm}
	\begin{document}
	\maketitle
		\section*{Problem 2.4}
			\begin{equation} \ket{+x} \xrightarrow{S_{y} \ \text{basis}} \begin{pmatrix}
				\braket{+y}{+x} \\ \braket{-y}{+x}
			\end{pmatrix} \quad, \ket{-x} \xrightarrow{S_{y} \ \text{basis}} \begin{pmatrix}
			\braket{+y}{-x} \\ \braket{-y}{-x}
			\end{pmatrix} 
			\end{equation}
			Computing each braket, 
			\begin{align*}
				\braket{+y}{+x} &= \frac{1}{\sqrt{2}} \braket{+y}{+z} + \frac{1}{\sqrt{2}} \braket{+y}{-z} \\
				&= \frac{1}{\sqrt{2}}\left(\frac{1}{\sqrt{2}}\right) + \frac{1}{\sqrt{2}}\left(\frac{-i}{\sqrt{2}} \right) = \frac12 \left(1-i\right).\\
				\braket{-y}{+x} &=\frac{1}{\sqrt{2}} \braket{-y}{+z} + \frac{1}{\sqrt{2}} \braket{-y}{-z} \\
				&= \frac{1}{\sqrt{2}} \left(\frac{1}{\sqrt{2} }\right) + \frac{1}{\sqrt{2}} \left(\frac{i}{\sqrt{2}}\right) = \frac12 \left(1+i\right)
			\end{align*}	
			Following $(1)$ and using the conjugate for the $\ket{-x}$ we conclude that
			$$ \ket{+x} = \frac12 \begin{pmatrix}
				1-i \\ 1+i
			\end{pmatrix} \quad, \ket{-x} = \frac12 \begin{pmatrix}
				1+i \\ 1-i
			\end{pmatrix}.$$
		\section*{Problem 2.5}
				\begin{equation} \ket{+z} \xrightarrow{S_{y} \ \text{basis}} \begin{pmatrix}
				\braket{+y}{+z} \\ \braket{-y}{+z}
				\end{pmatrix} \quad, \ket{-z} \xrightarrow{S_{y} \ \text{basis}} \begin{pmatrix}
				\braket{+y}{-z} \\ \braket{-y}{-z}
				\end{pmatrix} 
				\end{equation}
				Computing each braket, 
				\begin{align*}
					\braket{+y}{+z} &= \braket{+z}{+y}^{\ast} = \left(\frac{1}{\sqrt{2 }}\right)^{\ast} = \frac{1}{\sqrt{2}}\\
					\braket{+y}{-z} &= \braket{-z}{+y}^{\ast} = \left(\frac{i}{\sqrt{2 }}\right)^{\ast} = \frac{-i}{\sqrt{2}} \\
					\braket{-y}{+z} &= \braket{+z}{-y}^{\ast} = \left(\frac{1}{\sqrt{2 }}\right)^{\ast} = \frac{1}{\sqrt{2}} \\
					\braket{-y}{-z} &= \braket{-z}{-y}^{\ast} = \left(\frac{-i}{\sqrt{2 }}\right)^{\ast} = \frac{i}{\sqrt{2}}
				\end{align*} 
				We conclude that 
				$$ \ket{+z} = \frac{1}{\sqrt{2}} \begin{pmatrix}
				1 \\1 
				\end{pmatrix} \quad , \ket{-z} = \frac{1}{\sqrt{2}} \begin{pmatrix}
					-i \\ i
				\end{pmatrix}.$$
				Finally, 
				\begin{align*}
					\hat{S}_{z} &= \frac{\hbar }{2} (\ket{+z}\bra{+z} - \ket{-z}\bra{-z}) \\
					&=\left(\frac{\hbar}{2} \frac{1}{\sqrt{2}} \begin{pmatrix}
					1 \\ 1
					\end{pmatrix} \frac{1}{\sqrt{2}} \begin{pmatrix}
					1 & 1
					\end{pmatrix} - \frac{1}{\sqrt{2}} \begin{pmatrix}
					-i \\ i
					\end{pmatrix}\frac{1}{\sqrt{2}}\begin{pmatrix}
					i & -i
					\end{pmatrix}\right) = \frac{\hbar}{2} \begin{pmatrix}
					0 &1 \\
					1 & 0
					\end{pmatrix}
				\end{align*}
				Therefore, 
				\begin{align*}
					\la S_{z} \ra &= \frac{1}{\sqrt{2}}\begin{pmatrix}
					i & 1
					\end{pmatrix} \frac{\hbar}{2} \begin{pmatrix}
					0&1 \\ 1& 0
					\end{pmatrix} \frac{1}{\sqrt{2}} \begin{pmatrix}
					1 \\ -i
					\end{pmatrix} = \frac{\hbar}{2}.
				\end{align*}
			\section*{Problem 2.6}
				\begin{alignat*}{2}
					\ket{+z} &= \frac{1}{\sqrt{2}} \ket{+y} + \frac{1}{\sqrt{2}} \ket{-y} \quad  && \\
					\ket{+y} &= \frac{1}{\sqrt{2}} \ket{+z} +  \frac{i}{\sqrt{2}} \ket{-z} \quad \ket{-y} &&= \frac{1}{\sqrt{2}}\ket{+z} -\frac{i}{\sqrt{2}}\ket{-z }
				\end{alignat*}
				We use the relationship
				$$ \hat{J}_{y} \ket{\pm y} = \pm \frac{\hbar}{2} \ket{\pm y}, $$
				therefore, 
				\begin{align*}
					\hat{R}_{y} (\theta)\ket{+z} &= e^{\frac{-i}{\hbar} \hat{S}_{y} \theta} \left(\frac{\ket{+y}}{\sqrt{2}} + \frac{\ket{-y}}{\sqrt{2}}\right) \\
					&= e^{\frac{i \theta}{2}} \frac{1}{\sqrt{2}}\ket{+y} + e^{\frac{-i \theta}{2}} \frac{1}{\sqrt{2}}\ket{-y}
					\intertext{We switch back in the $z$ basis}
					&= \frac{e^{\frac{i \theta}{2}}\ket{+z} +i e^{\frac{i\theta}{2}} \ket{-z}}{2} +\frac{e^{\frac{-i \theta}{2}}\ket{+z} -i e^{\frac{-i\theta}{2}} \ket{-z}}{2}  \\
					&= \frac{\left(e^{\frac{i\theta}{2}} + e^{\frac{-i\theta}{2}}\right)}{2} \ket{+z} + \frac{i\left(e^{\frac{i\theta}{2}} - e^{\frac{-i\theta}{2}}\right)}{2} \ket{-z} \\
					&= \cos \left(\frac{\theta}{2}\right) \ket{+z} +\sin\left(\frac{\theta}{2}\right) \ket{-z}.
					\intertext{We evalaute at $\theta =\pi/2$,}
					\hat{R}_{y} \left(\frac{\pi}{2}\right) &= \cos \left(\frac{\pi}{4}\right) \ket{+z} + \sin \left(\frac{\pi}{4}\right)\ket{-z} \\
					&= \frac{1}{\sqrt{2}} \ket{+z} + \frac{1}{\sqrt{2}} \ket{-z},
				\end{align*}
				which is indeed $\ket{+x}$ in the $z$ basis.
			\section*{Problem 2.8}
				The Pauli matrix $\sigma_{x}$ is already written in the $z$ basis , so immediately, 
				\begin{align*}
					\la \varphi \lvert  \sigma_{x} \rvert \varphi \ra &= \frac{1}{\sqrt{5}} \begin{pmatrix}
					-i & 2
					\end{pmatrix} \frac{\hbar}{2} \begin{pmatrix}
					0 & 1 \\ 1 & 0 
					\end{pmatrix} \frac{1}{\sqrt{5}} \begin{pmatrix}
					i \\ 2
					\end{pmatrix} = 0
				\end{align*}
				Then it follows that 
				$$ \Delta S_{x}  = \sqrt{\la S_{x}^{2} \ra - \la S_{x} \ra^{2}}  = \sqrt{\left(\frac{\hbar}{2}\right)^{2} -0 } = \frac{\hbar}{2}.$$
			\section*{Question 5}
				\subsection*{a) }
					Let us compute with the Pauli matrices since they are both in the same basis
					\begin{align*}
						 [\hat{S}_{x} , \hat{S}_{z}] = \hat{S}_{x} \hat{S}_{z} - \hat{S}_{z}\hat{S}_{x} &=\frac{\hbar}{2} \begin{pmatrix}
						 	0 & 1 \\ 1& 0 
						 \end{pmatrix} 
						 \frac{\hbar}{2} \begin{pmatrix}
						 	1 & 0 \\ 0& -1 
						 \end{pmatrix} - \frac{\hbar}{2} \begin{pmatrix}
						 1 & 0 \\ 0& -1 
						 \end{pmatrix}\frac{\hbar}{2} \begin{pmatrix}
						 0 & 1 \\ 1& 0 
						 \end{pmatrix} \\
						 &= \frac{\hbar^{2}}{2} \begin{pmatrix}
						 0 & -1 \\
						 1 & 0
						 \end{pmatrix} = i \hbar \hat{S}_{y} \neq 0.,
 					\end{align*}
 					we conclude that the operators do not commute.
 				\subsection*{b) }
 					Let us apply these operators to $\ket{+x}$ written in the $z$ basis. 
 					\begin{align*}
 						\hat{S}_{x} \hat{S}_{z} \ket{+x} &= \frac{\hbar}{2} \begin{pmatrix}
 						0 & 1 \\1& 0 
 						\end{pmatrix} \frac{\hbar}{2} \begin{pmatrix}
 						1 & 0 \\0 & -1 
 						\end{pmatrix}
 						\frac{1}{\sqrt{2}} \begin{pmatrix}
 						1 \\ 0
 						\end{pmatrix} = \frac{\hbar^{2}}{4 \sqrt{2}} \begin{pmatrix}
 						0 \\1
 						\end{pmatrix} \\
 						\hat{S}_{z} \hat{S}_{x} \ket{+x} &= \frac{\hbar}{2} \begin{pmatrix}
 						1 & 0 \\0 & -1 
 						\end{pmatrix} \frac{\hbar}{2} \begin{pmatrix}
 						 0 & 1 \\1 & 0 
 						\end{pmatrix}
 						\frac{1}{\sqrt{2}} 
 						\begin{pmatrix}
 						1 \\ 0
 						\end{pmatrix} = \frac{-\hbar^{2}}{4 \sqrt{2}} \begin{pmatrix}
 						0 \\ 1
 						\end{pmatrix}.
 					\end{align*}
				
	\end{document}