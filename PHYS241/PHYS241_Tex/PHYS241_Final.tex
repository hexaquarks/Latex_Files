\documentclass[
	12pt,
	]{article}
		\usepackage{xcolor}
			\usepackage[dvipsnames]{xcolor}
			\usepackage[many]{tcolorbox}
		\usepackage{changepage}
		\usepackage{titlesec}
		\usepackage{caption}
		\usepackage{mdframed, longtable}
		\usepackage{mathtools, amssymb, amsfonts, amsthm, bm,amsmath} 
		\usepackage{array, tabularx, booktabs}
		\usepackage{graphicx,wrapfig, float, caption}
		\usepackage{tikz,physics,cancel, siunitx, xfrac}
		\usepackage{graphics, fancyhdr}
		\usepackage{lipsum}
		\usepackage{xparse}
		\usepackage{thmtools}
		\usepackage{mathrsfs}
		\usepackage{undertilde}
		\usepackage{tikz}
		\usepackage{fullpage,enumitem}
		\usepackage[labelfont=bf]{caption}
	\newcommand{\td}{\text{dim}}
	\newcommand{\tvw}{T : V\xrightarrow{} W }
	\newcommand{\ttt}{\widetilde{T}}
	\newcommand{\ex}{\textbf{Example}}
	\newcommand{\aR}{\alpha \in \mathbb{R}}
	\newcommand{\abR}{\alpha \beta \in \mathbb{R}}
	\newcommand{\un}{u_1 , u_2 , \dots , n}
	\newcommand{\an}{\alpha_1, \alpha_2, \dots, \alpha_2 }
	\newcommand{\sS}{\text{Span}(\mathcal{S})}
	\newcommand{\sSt}{($\mathcal{S}$)}
	\newcommand{\la}{\langle}
	\newcommand{\ra}{\rangle}
	\newcommand{\Rn}{\mathbb{R}^{n}}
	\newcommand{\R}{\mathbb{R}}
	\newcommand{\Rm}{\mathbb{R}^{m}}
	\usepackage{fullpage, fancyhdr}
	\newcommand{\La}{\mathcal{L}}


	\usepackage{mathtools}
	\DeclarePairedDelimiter{\norm}{\lVert}{\rVert}
	\newcommand{\vectorproj}[2][]{\textit{proj}_{\vect{#1}}\vect{#2}}
	\newcommand{\vect}{\mathbf}
	\newcommand{\uuuu}{\sum_{i=1}^{n}\frac{<u,u_i}{<u_i,u_i>} u_i}
	\newcommand{\B}{\mathcal{B}}
	\newcommand{\Ss}{\mathcal{S}}
	
	\newtheorem{theorem}{Theorem}[section]
	\theoremstyle{definition}
	\newtheorem{corollary}{Corollary}[theorem]
	\theoremstyle{definition}
	\newtheorem{lemma}[theorem]{Lemma}
	\theoremstyle{definition}
	\newtheorem{definition}{Definition}[section]
	\theoremstyle{definition}
	\newtheorem{Proposition}{Proposition}[section]
	\theoremstyle{definition}
	\newtheorem*{example}{Example}
	\theoremstyle{example}
	\newtheorem*{note}{Note}
	\theoremstyle{note}
	\newtheorem*{remark}{Remark}
	\theoremstyle{remark}
	\newtheorem*{example2}{External Example}
	\theoremstyle{example}
	
	\title{PHYS 241 Final Exam}
	\titleformat*{\section}{\LARGE\normalfont\fontsize{12}{12}\bfseries}
	\titleformat*{\subsection}{\Large\normalfont\fontsize{10}{15}\bfseries}
	\author{Mihail Anghelici 260928404 }
	\date{\today}
	
	\relpenalty=9999
			\binoppenalty=9999
		
			\renewcommand{\sectionmark}[1]{%
			\markboth{\thesection\quad #1}{}}
			
			\fancypagestyle{plain}{%
			  \fancyhf{}
			  \fancyhead[L]{\rule[0pt]{0pt}{0pt} Final Exam } 
			  \fancyhead[R]{\small Mihail Anghelici $260928404$} 
			  \fancyfoot[C]{-- \thepage\ --}
			  \renewcommand{\headrulewidth}{0.4pt}}
			\pagestyle{plain}
			\setlength{\headsep}{1cm}
	\captionsetup{margin =1cm}
	\begin{document}
	\maketitle
		\section*{Preliminary }
			\textit{My signature below certifies that I have not, nor will
			I, consult with any other person about the exam, or any other subject
			related to it}
		\section*{Question 1 }
			\subsection*{(a)}
				We first and foremost write the KVL equations 
				\begin{equation*}
					\begin{alignat*}{2}
										&\textbf{(KCL)} \qquad &I_{1}+I_{3} = I_{2}\\
										&\textbf{(KVL1)}\qquad&V_{1}-R_{1}I_{1} - R_{2}I_{2} = 0  \\
										 &\textbf{(KVL2)}\qquad&V_{2} - I_{2}(R_{2} + R_{3}) + I_{1}(R_{3}) = 0 			
									\end{alignat*} \quad \implies 
									\begin{cases*}
										\hphantom{-}R_{1}I_{1} + R_{2}I_{2} &= V_{1} \\
										-R_{3}I_{1} + (R_{2}+R_{3})I_{2} &= V_{2}
									\end{cases*}
				\end{equation*}
				This leads to the linear equation system 
				$$ \begin{pmatrix}
					R_{1} & R_{2} \\ -R_{3} & (R_{2}+R_{3}) 
				\end{pmatrix}\begin{pmatrix}
					I_{1} \\ I_{2}
				\end{pmatrix} = \begin{pmatrix}
					V_{1} \\ V_{2}
				\end{pmatrix} \xrightarrow{} 
				\begin{pmatrix}
									5 \ \si{\kilo\ohm} & 1 \ \si{\kilo\ohm} \\ -1 \ \si{\kilo\ohm} & 2 \ \si{\kilo\ohm} 
								\end{pmatrix}\begin{pmatrix}
									I_{1} \\ I_{2}
								\end{pmatrix} = \begin{pmatrix}
									12 \ \si{\volt} \\ 3 \ \si{\volt}
								\end{pmatrix}.$$
				This system is trivially solved using \textit{Cramer's Rule}
				$$ \begin{pmatrix}
													5 \ \si{\kilo\ohm} & 1 \ \si{\kilo\ohm} \\ -1 \ \si{\kilo\ohm} & 2 \ \si{\kilo\ohm} 
												\end{pmatrix}^{-1}
												\begin{pmatrix}
												12 \ \si{\volt} \\ 3 \ \si{\volt}
												\end{pmatrix} = \frac{1}{11}
				\begin{pmatrix}
																	2 \ \si{\kilo\ohm} & -1 \ \si{\kilo\ohm} \\ 1 \ \si{\kilo\ohm} & 5 \ \si{\kilo\ohm} 
																\end{pmatrix}	
																\begin{pmatrix}
																												12 \ \si{\volt} \\ 3 \ \si{\volt}
																												\end{pmatrix} = \begin{pmatrix}
																													\sfrac{21}{11} \\ \sfrac{27}{11}
																												\end{pmatrix}							
												.$$
			Therefore, the values for $I_{1}$ and $I_{2}$ are respectively $1.909 \ \si{\milli\ampere}$ and $2.455 \ \si{\milli\ampere}$. \\
			
			\noindent To find the values of $V_{A}$ and $V_{B}$ the nodal method is applied
			$$ I_{1} = \frac{V_{1} - V_{A}}{R_{2}} \implies V_{A} = V_{1}-I_{1}R_{1} = 12 \ \si{\volt} - 1.909\cross10^{-3} \ \si{\ampere} \times 5000 \ \si{\ohm} = 2.455 \ \si{\volt}. $$
			The voltage at the point $V_{B}$ is simply $0 \ \si{\volt}$ since that point is a ground.
			\subsection*{(b)}
				 Since voltage is conserved along a loop and since there is a ground placed right after the positive polarity of the f.e.m source, then $V_{B}$ must necessarily be $-12 \ \si{\volt}$. Moreover, the direction of the currents is not altered by changing the ground's place in the present configuration so we may apply the nodal method for $I_{2}$ that was found in $(a)$
				 $$ I_{2} = \frac{V_{A} - V_{B}}{R_{2}} \implies V_{A} = I_{2}R_{2} + V_{B} = 2.455 \ \si{\milli\ampere} \times 1000 \ \si{\ohm} - 12 \ \si{\volt} = -9.455 \ \si{\volt}.$$
		\section*{Question 2}
			\subsection*{(a)}
				At $t=0$ the capacitor is fully discharged,hence $Q(0) = 0 \ \si{\coulomb} \implies \Delta V_{C}(0) = 0 \ \si{\volt}$. Since $R_{2}$ is in parallel with $C$ ,then it follows that $\Delta V_{R_{2}}(0) = 0 \ \si{\volt}$ as well. Applying Kirchhoff's law along the first loop and isolating $I$ yields 
				$$ I(0) = \frac{V_{0}}{R_{1}}.$$
			\subsection*{(b)}
				After a long  time, the capacitor is \textit{"fully"} charged such that no voltage is dropped across it. So we may effectively remove this component from the circuit ,compute the equivalent resistance from the resistors in series and apply Kirchhoff's law along the circuit to obtain 
				$$ I(t)_{t\to \infty} = \frac{V_{0}}{R_{1}+ R_{2}}.$$
			\subsection*{(c)}
				Three equations are needed 
				$$ \textcircled{1}. \ \ I = I_{1} + I_{2} \qquad\qquad\quad \textcircled{2}. \ \ V_{0} = I_{1}R_{1} + I_{2}R_{2} \qquad\qquad\quad \textcircled{3}. \ \ R_{2}I_{2}= \frac{Q}{C}.$$
				We first differentiate with respect to time \textcircled{3} and isolate $\dot{I_{2}}$ yielding 
				\begin{equation} 
				\dv{I_{2}}{t} = \frac{I_{1}}{R_{2}C}.
				\end{equation}
				Then we substitute \textcircled{1} into \textcircled{2} giving 
				\begin{equation}
					V_{0} = R_{1}I_{1} + (R_{1}+R_{2})I_{2}.
				\end{equation}
				Differentiating with respect to time Equation 2 and substitute Equation 1 inside
				\begin{align*}
					V_{0} = R_{1}I_{1} +(R_{1}+ R_{2})I_{2} \xrightarrow{d/dt} 0 &= R_{1}\dv{I_{1}}{t} + (R_{1}+ R_{2})\dv{I_{2}}{t} \\
					&=R_{1}\dv{I_{1}}{t} + \frac{(R_{1} + R_{2})}{R_{2}C} I_{1} 
					\intertext{Diving both sides by $R_{1}$ and letting $\tau \equiv R_{1}R_{2}C/(R_{1}+R_{2})$ we have a first order linear differential equation}
					&\dv{I_{1}}{t} + \frac{I_{1}}{\tau} =0.
				\end{align*}
				From ODEs , there exists an integrating factor $\mu(t) = e^{\int p(t) \ dt} = e^{\int \sfrac{1}{\tau} \ dt} = e^{\sfrac{t}{\tau}+C} = e^{\sfrac{t}{\tau}}e^{C} = C_{1}e^{\sfrac{t}{\tau}}$, where $C_{1} \equiv e^{C}$. We let $C_{1} = 1$ since only one integrating factor is needed and thus $\mu(t) = e^{\sfrac{t}{\tau}}$. Moreover, the general solution is given by 
				$y(t) = \frac{1}{\mu(t)} \int \mu (t) q(t) \ dt$ , i.e.,
				\begin{gather}
					I_{1}(t) = \frac{1}{e^{\sfrac{t}{\tau}}} \int e^{\sfrac{t}{\tau}} (0) \ dt = C_{2} e^{-\sfrac{t}{\tau}} \nonumber
					\intertext{The initial condition is the initial current therefore $C_{2}\equiv I_{1}(0)$, finally}
					I_{1}(t) = I_{1}(0) e^{-\sfrac{t}{\tau}}.
				\end{gather}
				We now integrate Equation 1 and substitute Equation 3 to have an expression for $I_{2}(t)$
				\begin{align*}
					I_{2}(t) = \int_{0}^{t} \frac{I_{1}}{C R_{2}} \ dt= \frac{I_{1}(0)}{C R_{2}} \int_{0}^{t} e^{-\sfrac{t}{\tau}} \ dt= \frac{I_{1}(0)}{C R_{2}} \tau \left(1 - e^{-\sfrac{t}{\tau}}\right).
				\end{align*}
				Combining the previous result with Equation 3 provides a value for the current $I(t)$
				\begin{align*}
					I(t) = I_{1}(t) + I_{2}(t) &= I_{1}(0) e^{-\sfrac{t}{\tau}} + \frac{I_{1}(0)}{C R_{2}} \tau \left(1- e^{-\sfrac{t}{\tau}}\right) \\
					&=I_{1}(0) \bigg[e^{-\sfrac{t}{\tau}} + \frac{\tau\left(1- e^{-\sfrac{t}{\tau}}\right)}{c R_{2}}\bigg]
					\intertext{Using the result from (a) , $I_{1}(0)$ is the initial current in the circuit therefore}
					I(t) &=\frac{V_{0}}{R_{1}} \bigg[e^{-\sfrac{t}{\tau}} + \frac{\tau\left(1- e^{-\sfrac{t}{\tau}}\right)}{c R_{2}}\bigg]\qquad ,\text{with } \ \tau = \frac{R_{1}R_{2}C}{R_{1} + R_{2}}.	
				\end{align*}
			\section*{Question 3}
				\subsection*{(a)}
					$$ \frac{V'}{V} = \frac{Z_{R_{2}}}{Z_{R_{1}} + Z_{R_{2}}} = \frac{R_{2}}{R_{1} + R_{2}}.$$
					The amplitude is simply $R_{2} / (R_{1} + R_{2})$ since this is not a complex number. The phase is $0$ since there is no imaginary part and $\tan^{-1}(0) = 0$. Having found the essential terms , we write 
					$$ V_{\text{out}} = \frac{V_{0} \cos(\omega t) R_{2}}{(R_{1} + R_{2})}.$$
				\subsection*{(b)}
				Let $R_{T} \equiv R_{1} + R_{2}$ since the resistors are in series. Then
					$$ \frac{V'}{V} = \frac{Z_{C}}{Z_{C} + Z_{R_{T}}} = \frac{\frac{1}{j \omega C} }{\frac{1}{j \omega C} + R_{T}} = \frac{1}{1 + j \omega C R_{T}} = \frac{1}{1 + j\omega \tau} \qquad ,\text{with } \ \tau = R_{T}C.$$
					We compute the amplitude 
					$$ \abs{\frac{V'}{V}} =\left(\left(\frac{1}{1 + j\omega \tau}\right)\left(\frac{1}{1-j\omega \tau}\right)\right)^{\sfrac{1}{2}} = \frac{1}{\sqrt{1 + (\omega \tau)^{2}}}.$$
					It follows that for the phase  
					\begin{align*} &\because \ \left(\frac{1}{1 + j\omega\tau}\right)\left(\frac{1-j\omega \tau}{1- j\omega\tau}\right) = \frac{1}{1+ (\omega \tau)^{2}}-j \frac{\omega\tau}{1+(\omega\tau)^{2}} \\ 
					&\hphantom{\therefore \ } \ \varphi(\omega) =\tan^{-1}\left(\frac{\text{Im}[V'/V]}{\text{Re}[V'/V]}\right) = -\tan^{-1} \omega \tau.
					\end{align*}
					Having found all the essential terms it follows that
					$$ V'(t) = V_{\text{out}} = \frac{V_{0}}{\sqrt{1 + (\omega \tau)^{2}}}\cos (\omega t - \tan^{-1} \omega\tau).$$
				\subsection*{(c)}
					As $\omega \to 0$ the impedance of the capacitor is very high compared to the equivalent resistance, therefore little voltage is dropped across the resistors such that $V_{0} ' \approx 0$. The current leads by $\pi/2$ meaning that the phase shift is $\pi/2$. \\
					
					\noindent As $\omega \to \infty$ the capacitor's impedance vanishes and all voltage is dropped across the equivalent resistors such that $V_{0}' \approx V_{0}$ and thus the phase shift is $\approx 0$.\\
					
					\noindent Given these limits and the circuit's configuration , we can regard this circuit as a low-pass RC filter.
				\section*{Question 4}
					\subsection*{(a)}
						Let us first compute $a_{0}$
						$$ \frac{a_{0}}{2} = \frac{1}{T}\left( \int_{-T/2}^{0} -V_{0} \ dt + \int_{0}^{T/2} V_{0} \ dt \right) = \frac{1}{T}\left(-\frac{T}{2} + \frac{T}{2}\right) = 0 \implies a_{0} = 0.$$
						The given square wave is an odd function therefore $a_{n} = 0 \ \forall \ n \in \mathbb{N}_{+}.$
						\begin{align*}
							b_{n> 0} &= \frac{1}{T} \bigg[\int_{-T/2}^{0} -V_{0}\sin(\omega_{n}t) \ dt + \int_{0}^{T/2} V_{0} \sin(\omega_{n}t) \ dt\bigg] \\
							&= \frac{1}{T} \bigg[-\frac{V_{0}}{\omega_{n}} \left(\eval{-\cos(\omega_{n} t)}_{-T/2}^{0}\right) + \frac{V_{0}}{\omega_{n} } \left(\eval{-\cos(\omega_{n}t)}_{0}^{T/2}\right)\bigg] \\
							&=\frac{1}{T} \bigg[\frac{-V_{0}}{\omega_{n}} \left(-1 + \cos\left(\frac{-2 n \pi T }{2 T}\right)\right) + \frac{V_{0}}{\omega_{n}}\left(-\cos \left(\frac{2 n \pi T}{2T}\right) + 1\right)\bigg] \\
							&= \frac{-V_{0}}{2 n \pi} \cos(-\pi n ) + \frac{V_{0}}{n \pi} - \frac{V_{0}}{2n \pi} \cos(n \pi) 
							\intertext{Using the identity $\cos(-x) = \cos(x)$ and rearranging yields} 
							&= \frac{V_{0}}{n \pi} (1- \cos(n \pi)) \\
							&=\frac{2V_{0}}{n \pi} \left(\frac{1-\cos(n \pi)}{2}\right) = \frac{2V_{0}}{n\pi} \sin^{2}\left(\frac{n \pi}{2}\right).
						\end{align*}
					We note that 
					\begin{equation*}
					\frac{2V_{0}}{n\pi} \sin^{2}\left(\frac{n \pi}{2}\right) \ \ 
						\begin{cases}
							&1 \qquad, \text{for } n \ \text{odd}\\
							&0 \qquad, \text{for } n \ \text{even}
						\end{cases}
					\end{equation*}
					Finally, we write the Fourier series for the given square wave 
					$$ f(t) = \frac{2 V_{0}}{\pi}\sum_{n=1}^{\infty}\frac{1}{(2n-1)} \sin\left(\frac{2\pi (2n-1)}{T} t\right).$$
					\subsection*{(b)}
					We first and foremost find an expression for the transfer function 
					\begin{align*}
						H(\omega) &= \frac{\left(\frac{1}{Z_{L}} + \frac{1}{Z_{C}}\right)^{-1}}{\left(\frac{1}{Z_{L}} + \frac{1}{Z_{C}}\right)^{-1} + Z_{R}} = \frac{\left(\frac{1}{j\omega L} + j\omega C\right)^{-1}}{\left(\frac{1}{j\omega L} + j\omega C\right)^{-1} + R} = \frac{1}{1 + R\left(\frac{1}{j\omega L} + j\omega C\right)} 
						\intertext{Multiplying by $j\omega L$ and diving by $R$ whilst defining $\tau \equiv L/R$ with $LC = 1/\omega_{0}^{2}$ yields}
						H(\omega) &= \frac{j\omega L }{j \omega L + R(1- (\omega/ \omega_{0})^{2})} = \frac{j\omega \tau}{j\omega \tau + 1 - (\omega/ \omega_{0})^{2}}.
					\end{align*}
					We carry on by finding the amplitude of $H(\omega)$
					$$ \abs{H(\omega)} = \left(\left(\frac{j\omega \tau}{j\omega \tau + 1 - (\omega/ \omega_{0})^{2}} \right) \left( \frac{-j\omega \tau}{-j\omega \tau + 1 - (\omega/ \omega_{0})^{2}}\right)\right)^{\sfrac{1}{2}} = \frac{\omega \tau}{\sqrt{\left(1- (\omega / \omega_{0})^{2}\right)^{2} + (\omega \tau)^{2}}}.$$
					Finally,we look for the phase
					\begin{gather*} 
					\left(\frac{j\omega \tau}{j\omega \tau + 1 - (\omega/ \omega_{0})^{2}}\right)\left(\frac{1-(\omega /\omega_{0})^{2} - j\omega\tau}{1- (\omega / \omega_{0})^{2} - j\omega\tau}\right)= \frac{(\omega \tau)^{2}}{\left(1- (\omega / \omega_{0})^{2}\right)^{2} + (\omega \tau)^{2}} + j \frac{(\omega \tau)(1-(\omega /\omega_{0})^{2})}{\left(1- (\omega / \omega_{0})^{2}\right)^{2} + (\omega \tau)^{2}} \\
					\therefore \varphi_{n}(\omega_{n}) = \tan^{-1}\left(\frac{(\omega\tau)(1-(\omega /\omega_{0})^{2})}{(\omega \tau)^{2}}\right) = \tan^{-1}\left(\frac{1-(\omega/\omega_{0})^{2}}{\omega\tau }\right).
					\end{gather*}
					Using the result from (a) we write the $V_{\text{out}}$ function
					\begin{equation} 
					V_{\text{out}} = \sum_{n=1}^{\infty} \frac{2V_{0}}{\pi(2n -1)}\frac{\omega_{n} \tau}{\sqrt{\left(1- (\omega_{n} / \omega_{0})^{2}\right)^{2} + (\omega_{n} \tau)^{2}}} \sin\left(\omega_{n} t + \tan^{-1} \left(\frac{1-(\omega_{n}/\omega_{0})^{2}}{\omega_{n}\tau }\right)\right).
					\end{equation}
					To see which frequencies go through , we look for $ \omega_{0} \pm \Delta \omega $
					\begin{align*}
						\omega_{0} &= \frac{1}{\sqrt{LC}} = \frac{1}{\sqrt{(1 \ \si{\milli\henry})(2.8 \ \si{\micro\farad})}} = 18 898 \ \si{\per\second}. \\
						\tau &= \frac{L}{R} = \frac{1 \ \si{\milli\henry}}{3 \ \si{\kilo\ohm}} = 3.33 \cross 10^{-7} \ \si{\second}.\\ 
						Q &= \frac{1}{\tau \omega_{0}} = \frac{1}{(3.33\cross 10^{-7} \ \si{\second}) (18898 \ \si{\per\second})} = 158.74\\
						\Delta \omega &= \frac{\omega_{0}}{Q} = \frac{18 898 \ \si{\per\second}}{158.74} = 119.05 \ \si{\per\second}.\\
						&\qquad \therefore \omega_{0} \pm \Delta \omega = (18 898 \pm 119) \ \si{\per\second}.
					\end{align*}
					We now look for the frequency components which lie within this range 
					$$ \omega_{1} = \frac{2\pi (1)}{T} = \cancel{6283.2 \ \si{\per\second}} \qquad \omega_{2} = \frac{2\pi (2)}{T} = \cancel{12566.4 \ \si{\per\second}} \qquad \omega_{3}=\frac{2\pi (3)}{T} = 18849.6 \ \si{\per\second} \ \checkmark.$$
					The amplitude can then be computed with 
					\begin{align*} 
					A_{\text{out}} &=  \frac{2V_{0}}{\pi(2(3) -1)}\frac{\omega_{3} \tau}{\sqrt{\left(1- (\omega_{3} / \omega_{0})^{2}\right)^{2} + (\omega_{3} \tau)^{2}}} \\
					&=  \frac{2(1)}{\pi(2(3) -1)}\frac{(18849.6)(3.33\cross10^{-7})}{\sqrt{\left(1- (18849.6 / 18898)^{2}\right)^{2} + ((18849.6)(3.33\cross10^{-7}))^{2}}} = 0.09869 \ \si{\volt}.
					\end{align*}
					THe phase is also computed with 
					\begin{align*}
						\varphi = -\frac{\pi}{2} + \tan^{-1}\left(\frac{1-\left(\frac{18849.6}{18898}\right)^{2}}{18849.6(3.33\cross 10^{-7})}\right) = -\frac{\pi}{2} + 39^{\circ} \approxeq -\frac{\pi}{4}.
					\end{align*}
				\section*{Question 5}
				Since $f(t) = t/T$ for $\abs{t} \le T/2$,
					let us compute $C_{k}$ using the definition
					\begin{align*}
						C_{k} = \frac{1}{T}\int_{-T/2}^{T/2} f(t) e^{-j \omega_{k}t} \ dt &= \frac{1}{T^{2}} \int_{-T/2}^{T/2}t e^{-j\omega_{k} t } \ dt \\
						&= \frac{1}{T^{2}} \bigg[\eval{\frac{-te^{-j\omega_{k}t}}{j\omega_{k}}}_{-T/2}^{T/2} + \frac{1}{j \omega_{k}}\int_{-T/2}^{T/2} e^{-j \omega_{k} t} \ dt\bigg] \\
						&= \frac{1}{T^{2}}\bigg[\frac{1}{j \omega_{k}} \left(\frac{-Te^{\frac{-j \omega_{k} T}{2}}}{2} - \frac{Te^{\frac{j\omega_{k}T}{2}}}{2}\right) + \frac{2}{j\omega_{k}^{2}}\left(\frac{e^{\frac{j\omega_{k}T}{2}} - e^{-\frac{j\omega_{k}T}{2}}}{2 j }\right)\bigg]\\
						&=\frac{1}{T^{2}} \bigg[\frac{-T \cos\left(\frac{\omega_{k}T}{2}\right)}{j \omega_{k}} + \frac{2 \sin\left(\frac{\omega_{k}T}{2}\right)}{j \omega_{k}^{2}}\bigg] \\
						&=\frac{1}{T^{2}} \bigg[\frac{Ti\cos\left(\frac{\omega_{k}T}{2}\right)}{\omega_{k}} - \frac{2i\sin\left(\frac{\omega_{k}T}{2}\right)}{\omega_{k}^{2}}\bigg]
						\intertext{Using $\omega_{k} = 2\pi k /T$ the expresion reduces to }
						C_{k}&= \frac{\pi i k \cos(\pi k ) - i \sin(\pi k )}{2\pi^{2}k^{2}}.
					\end{align*}
					We now compute the Fourier transform
					\begin{align*}
						f(t) &= \sum_{k=-\infty}^{\infty}\left(\frac{\pi i k \cos(\pi k ) - i \sin(\pi k )}{2\pi^{2}k^{2}}\right)e^{j\omega_{k}t} \\ 
						\therefore \ \ F(\omega) &= \int_{-\infty}^{\infty} \sum_{k=-\infty}^{\infty}\left(\frac{\pi i k \cos(\pi k ) - i \sin(\pi k )}{2\pi^{2}k^{2}}\right)e^{j\omega_{k} t} e^{-j\omega t} \ dt \\
						&=\sum_{k=-\infty}^{\infty} \left(\frac{\pi i k \cos(\pi k ) - i \sin(\pi k )}{2\pi^{2}k^{2}}\right)\int_{-\infty}^{\infty} e^{j(\omega_{k}- \omega)t} \ dt \\
						&=\sum_{k=-\infty}^{\infty} \left(\frac{\pi i k \cos(\pi k ) - i \sin(\pi k )}{2\pi^{2}k^{2}}\right)\underbrace{\int_{-\infty}^{\infty} e^{-j(\omega- \omega_{k})t} \ dt}_{\equiv \delta(\omega - \omega_{k})} \\
						F(\omega) &= \sum_{k=-\infty}^{\infty}\left(\frac{\pi i k \cos(\pi k ) - i \sin(\pi k )}{2\pi^{2}k^{2}}\right) \delta(\omega - \omega_{k}).
					\end{align*}
					
				
	\end{document}