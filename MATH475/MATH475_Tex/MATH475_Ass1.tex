\documentclass[
	12pt,
	]{article}
		\usepackage{xcolor}
			\usepackage[dvipsnames]{xcolor}
			\usepackage[many]{tcolorbox}
		\usepackage{changepage}
		\usepackage{titlesec}
		\usepackage{caption}
		\usepackage{mdframed, longtable}
		\usepackage{mathtools, amssymb, amsfonts, amsthm, bm,amsmath} 
		\usepackage{array, tabularx, booktabs}
		\usepackage{graphicx,wrapfig, float, caption}
		\usepackage{tikz,physics,cancel, siunitx, xfrac}
		\usepackage{graphics, fancyhdr}
		\usepackage{lipsum}
		\usepackage{xparse}
		\usepackage{thmtools}
		\usepackage{mathrsfs}
		\usepackage{undertilde}
		\usepackage{tikz}
		\usepackage{fullpage,enumitem}
		\usepackage[labelfont=bf]{caption}
	\newcommand{\td}{\text{dim}}
	\newcommand{\tvw}{T : V\xrightarrow{} W }
	\newcommand{\ttt}{\widetilde{T}}
	\newcommand{\ex}{\textbf{Example}}
	\newcommand{\aR}{\alpha \in \mathbb{R}}
	\newcommand{\abR}{\alpha \beta \in \mathbb{R}}
	\newcommand{\un}{u_1 , u_2 , \dots , n}
	\newcommand{\an}{\alpha_1, \alpha_2, \dots, \alpha_2 }
	\newcommand{\sS}{\text{Span}(\mathcal{S})}
	\newcommand{\sSt}{($\mathcal{S}$)}
	\newcommand{\la}{\langle}
	\newcommand{\ra}{\rangle}
	\newcommand{\Rn}{\mathbb{R}^{n}}
	\newcommand{\R}{\mathbb{R}}
	\newcommand{\Rm}{\mathbb{R}^{m}}
	\usepackage{fullpage, fancyhdr}
	\newcommand{\La}{\mathcal{L}}
	\newcommand{\ep}{\epsilon}
	\newcommand{\de}{\delta}
	\usepackage[math]{cellspace}
		\setlength{\cellspacetoplimit}{3pt}
		\setlength{\cellspacebottomlimit}{3pt}
	\newcommand\numberthis{\addtocounter{equation}{1}\tag{\theequation}}


	\usepackage{mathtools}
	\DeclarePairedDelimiter{\norm}{\lVert}{\rVert}
	\newcommand{\vectorproj}[2][]{\textit{proj}_{\vect{#1}}\vect{#2}}
	\newcommand{\vect}{\mathbf}
	\newcommand{\uuuu}{\sum_{i=1}^{n}\frac{<u,u_i}{<u_i,u_i>} u_i}
	\newcommand{\B}{\mathcal{B}}
	\newcommand{\Ss}{\mathcal{S}}
	\usepackage{newtxtext, newtxmath}
	\newtheorem{theorem}{Theorem}[section]
	\theoremstyle{definition}
	\newtheorem{corollary}{Corollary}[theorem]
	\theoremstyle{definition}
	\newtheorem{lemma}[theorem]{Lemma}
	\theoremstyle{definition}
	\newtheorem{definition}{Definition}[section]
	\theoremstyle{definition}
	\newtheorem{Proposition}{Proposition}[section]
	\theoremstyle{definition}
	\newtheorem*{example}{Example}
	\theoremstyle{example}
	\newtheorem*{note}{Note}
	\theoremstyle{note}
	\newtheorem*{remark}{Remark}
	\theoremstyle{remark}
	\newtheorem*{example2}{External Example}
	\theoremstyle{example}
	
	\title{MATH 475 Assignment 1}
	\titleformat*{\section}{\LARGE\normalfont\fontsize{12}{12}\bfseries}
	\titleformat*{\subsection}{\Large\normalfont\fontsize{10}{15}\bfseries}
	\author{Mihail Anghelici 260928404 }
	\date{\today}
	
	\relpenalty=9999
			\binoppenalty=9999
		
			\renewcommand{\sectionmark}[1]{%
			\markboth{\thesection\quad #1}{}}
			
			\fancypagestyle{plain}{%
			  \fancyhf{}
			  \fancyhead[L]{\rule[0pt]{0pt}{0pt} Assignment 1 } 
			  \fancyhead[R]{\small Mihail Anghelici $260928404$} 
			  \fancyfoot[C]{-- \thepage\ --}
			  \renewcommand{\headrulewidth}{0.4pt}}
			\pagestyle{plain}
			\setlength{\headsep}{1cm}
	\captionsetup{margin =1cm}
	\begin{document}
	\maketitle
		\section*{Question 1}
		First and foremost we know that
		$$ \text{Net Heat} = \text{External heat source} + \text{Heat through edges},$$
	 	where the external heat source is 
		$$ \int_{V} r(x,t) p \ dx$$
		and the heat through the edges corresponds to 
		$$  - \int_{\partial V} \boldsymbol{q}(x,t) \cdot n \ d\sigma.$$
		Then the rate of change in energy ,defined by the thermal energy per unit mass $e(x,t)$ ,is equal to the the sum of the net flux and the external heat source;
			\begin{align*} 
				\int_{V} \pdv{e(x,t)}{t} p &= \int_{V} r(x,t) p \ dx - \int_{\partial V} \boldsymbol{q}(x,t) \cdot n \ d\sigma \\
				&= \int_{V} r(x,t) p \ dx + \int_{V} \text{div} \boldsymbol{q}(x,t) \ dx
				\end{align*}
				Since $\partial (e(x,t)) / \partial t = cu_{t}$ then
				\begin{gather*}
				\int_{V} cu_{t} p -\text{div} \boldsymbol{q} (x,t)  \ dx = \int_{V} r(x,t) p
				\intertext{Sicne $\text{div} \boldsymbol{q} (x,t) = \text{div} (\kappa (x) \nabla u)$ we obtain}
				\int_{V} cu_tp  -\text{div} (\kappa(x)\nabla u ) \ dx = \int_{V} r(x,t) p \\
				\intertext{This holds $\forall  \ V \subseteq \mathbb{R}^{n}$, i.e, it is pointwise ,therefore }
				\implies cu_tp -\text{div} (\kappa(x) \nabla u)  = r(x,t) p 
				\intertext{We rearrange by dividing $cp$ on both sides}
				\implies u_t -\frac{1}{cp}\text{div} (\kappa(x) \nabla u ) = \frac{r(x,t)}{c}
				\intertext{Let us define $k(x) = \kappa(x) / cp$, we can take a constant out of a $\text{div}$ argument such that  }
				\therefore u_t - \text{div}(k(x) \nabla u) = r(x,t).
				\end{gather*}
		\section*{Question 2}
			Let $w = u-v$ solve 
			\[\begin{cases}
				w_{t} -\nabla\cdot(k(x) \nabla w) = 0 & \text{in } \Omega_{T},\\
				w(x,0) = 0 & \text{in } \ \Omega,\\
				w(\sigma,t) = 0 & \text{in } \ \partial \Omega \times (0,T],
			\end{cases}
			\]
			Then by the energy method,
			\begin{align*}
			E(t) &:= \int_{\Omega} w^{2}(x,t) \ dx \\
			E'(t) &= \int_{\Omega}\frac{d}{dt} w^{2} (x,t) \ dx \\
			&=2 \int_{\Omega} w w_{t} \ dx
			\intertext{Using $w_{t} = \nabla \cdot(k(x) \nabla u)$,}
			E'(t) &= 2\int_{\Omega} w (\nabla \cdot (k(x)\nabla w)) \ dx \\
			&= 2\int_{\Omega} w(x) \text{div} (k(x) \nabla w) \\
			&= 2\int_{\Omega} w(x) \sum_{i} (k(x) w_{x_{i}})_{x_{i}} \\
			&= 2\int_{\partial \Omega} w(x) k(x) \sum_{i} w_{x_{i}} \ d\sigma - 2 \int_{\Omega} k(x)\sum_{i}(w_{x_{i}})^{2} \ dx \\
			&= - 2 \int_{\Omega} k(x)\sum_{i}(w_{x_{i}})^{2} \ dx \\
			E'(t) &= -2 \int_{\Omega} k(x) \Delta(w) 
			\end{align*}
			The last expression is $\le 0$ since $k(x) >0$, therefore since $E(0) = 0$ it follows that $E(t) \le 0 $, but by definition $E(t) \ge 0$, we conclude that 
			$$ E'(t) = -2 \int_{\Omega} k(x) \Delta w = 0 \implies w \equiv 0,$$
			the solution is unique.
		\section*{Question 3}
		We first note that $x+1+2\sin^{2}(2 \pi x) = x+2 - \cos(4\pi x).$ We let $w(x,t) = u(x,t) + v(x,t)$ and set $v(x,t) = -(x+2)$. Then $w(x,t)$ solves 
		$$\begin{cases}
				&w_{t} -2w_{xx} = 0\\
				&w(x,0) = -\cos(4 \pi x) \\
				&w_{x} (0,t) =0, \quad w_{x} (1,t) = 0.
		\end{cases}$$
			
			 The form of the solution is evidently $w(x,t) = e^{-2 \lambda^{2} t} (A \cos \lambda x + B \sin \lambda x)$, since it's the only one that matches the sinusoidal nature of the initial condition. 
			\begin{align*}
				w_{x} (x,t) &= \lambda e^{-2 \lambda^{2} t} (B \cos \lambda x - A \sin \lambda x ) \\
				w_{x} (0,t) &= \lambda e^{-2 \lambda^{2 }t} B =0 \implies B = 0. \\
				w_{x} (1,t) &= -\lambdae^{-2 \lambda^{2 }t} A \sin \lambda = 0 \implies \sin \lambda =0 \implies \lambda = n\pi ,  \ \ \text{for } n \in \mathbb{Z}.
			\end{align*}
			So we have that the solution is 
			 $$ w(x,t) = \sum_{n=0}^{N} e^{-2 n^{2} \pi^{2 }t} A_{n} \cos(n \pi x).$$
			 We carry on with the initial condition ; 
			 \begin{align*}
			 	w(x,0) &= \sum_{n=0}^{N} A_{n} \cos(n \pi x) = -\cos(4 \pi x) \\
			 	w(x,0) &= 0 + 0 + 0 + 0 - 1 \cos (4 \pi x) = - \cos(4 \pi x) \\
 			 \end{align*}
 			 Finally, having found the appropriate constants we write the general solution ; 
 			 $$ w(x,t) = e^{-2 (16) \pi^{2} t} \cos(4 \pi x) \implies  u(x,t) = e^{-32\pi^{2} t} \cos(4 \pi x) + x + 2. $$
		\section*{Question 4}
			Let $w = u-v$ solve 
			\[\begin{cases}
			w_{t} -kw_xx = 0 & \text{in } \ L_{T} := (0,L) \times (0,T]\\
			w(x,0) = 0 & \text{in } (0,L],\\
			w_{x}(0,t) - \alpha w(0,t) =0 \quad , w_{x} (L,t) + \alpha w(L,t) =0 & \text{in } \times (0,T]
			\end{cases}
			\]
			Then by the energy method 
			\begin{align*}
				E(t) &:= \int_{\Omega}w^{2} (x,t) \ dx \\
				E'(t) &= \int_{\Omega} 2 w w_{t} \ dx \\
				&= \int_{\Omega} 2k w w_xx \ dx
				\intertext{Integrating by parts, }
				&=2k \Big[w(x,t) w_{x}(x,t) \Big|_{0}^{L} - \int_{0}^{L} w_{x}^{2}(x,t) \ dx\Big]\\
				&=2k \Big[(w(L,t) w_{x}(L,t) - w(0,t)w_{x}(0,t))- \int_{0}^{L} w_{x}^{2}(x,t) \ dx\Big]\\
				&=2k \Big[-\alpha(w_{x}^{2}(L,t) - w_{x}^{2}(0,t))- \int_{0}^{L} w_{x}^{2}(x,t) \ dx\Big]
				\intertext{The RHS $\le 0$ since $\alpha >0, k>0$ and $w_{x}^{2} > 0$ with positive integral bounds given that $L \neq 0$. All of which implies that $E'(t) \le 0 \implies E'(t) = 0$ ,then since $E(t) \ge 0$ by definition ,it follows that}
				\underbrace{2k\alpha(w_{x}^{2}(L,t) - w_{x}^{2}(0,t))}_{k> 0, \alpha >0 \implies \ge 0} &= \underbrace{-2k \int_{0}^{L} w_{x}^{2}(x,t) \ dx}_{k> 0 \implies \le 0} \\
				\therefore 2k\alpha(w_{x}^{2}(L,t) &- w_{x}^{2}(0,t)) = -2k \int_{0}^{L} w_{x}^{2}(x,t) \ dx = 0 \\
				\implies w_{x} \equiv 0 &\implies w \equiv 0
			\end{align*}
		\section*{Question 5}
			Let the solution be of the form $u(x,t) = e^{-3 \lambda ^{2} t} (A \cos \lambda x + B \sin \lambda x)$. Then, 
			\begin{align*}
				u(0,t) &= 0 \implies e^{-3 \lambda^{2} t } A = 0 \implies A =0.\\
				u_{x} (x, t) &= \lambda e^{-3 \lambda^{2 }t} (B \cos \lambda x - A\sin \lambda x ) \\
				\implies u_{x} (\pi ,t) &= \lambdae^{-3 \lambda^{2 }t } B \cos \lambda \pi =0 \implies \cos \lambda n =0 \implies \lambda =\frac{n}{2} \quad \text{for } \ n \in \mathbb{Z}. 
				\intertext{So we have, }
				u(x,t) &= e^{-3 \lambda^{2 }t} B\sin\left(\frac{nx}{2}\right) 
				\intertext{Extending the solution in the summation form }
				u(x,t) &= \sum_{n=0}^{N} e^{-3 \lambda^{2}t} B_{n} \sin\left(\frac{nx}{2}\right)\\
				u(x,0) &= \sum_{n=0}^{N} B_{n} \sin\left(\frac{nx}{2}\right) = 4\sin\frac{x}{2} -\frac43 \sin\frac{3x}{2} \\
				&\implies B_{1} = 0 , B_{2} =5, B_{3} =0 , B_{4} = -\frac{4}{3}.\\
				\therefore u(x,t) &= 4e^{-3t} \sin \left(\frac{x}{2}\right) - \frac43 e^{-12t} \sin(\frac{3x}{2})
			\end{align*}
		\section*{Question 6} 
			By the Weak Maximum Principle, 
			$$ \min_{\partial_{P}\Omega_{T}}v \le v(x,t) \le \max_{\partial_{P}\Omega_{T}}v,$$
			where $\partial_{P}\Omega_{T}$ represents the parabolic boundary of a space-time cylinder (i.e., the base and the sides). Now since $\partial_{P}\Omega_{T} = \Omega \times \{ t=0\} \cup \partial \Omega \times (0,T]$, it follows that 
			\begin{align*}
			\min_{\partial_{P}\Omega_{T}}v &= \min \left( \min_{\Omega \times \{ t=0\}}v , \min_{\partial \Omega \times (0,T]}v \right)\\
			&= \min \left(\min_{\Omega \times \{ t=0\}}v , \min \left(\min_{(0,T]}(1+\sqrt{3}t), \min_{(0,T]}(\pi^{4} + e^{t}) \right) \right) \\
			&=\min(1, \min(1, \pi^{4})) \\
			&=1.\\
			&\qquad \quad \qquad \therefore v(x,t) \ge 1,
			\end{align*} 
			We conclude that $v(x,t)$ is a super solution.
			Then, we verify if the given function we're comparing to is a sub solution or a super solution to the PDE ; 
			Let $w = x^{4} + e^{-t}\sin(x)$ solve the PDE, then
			\begin{align*}
				w_{t} -w_{xx} &= (x^{4} + e^{-5}\sin(x))_{t} - (x^{4} + e^{-5}\sin(x))_{xx} \\
				&=  -e^{-t} \sin(x) - ((4x^{3} + e^{-t}\cos(x))_{x} \\
				&= -e^{-t}\sin(x) -12x^{2} +e^{-t}\sin(x) \\
				&= -12 x^{2} \le 0 \ \forall (x,t) \in (0,\pi) \cross (0,T]
			\end{align*}
			We conclude that $w$ as defined is a sub solution. Finally, by definition, super solutions are bigger than sub solutions so
			$$ v(x,t) \ge x^{4} + e^{-5}\sin(x) \quad \forall (x,t) \in \pi_{T}.$$
		\section*{Question 7}
			\subsection*{a ) }
				Since $u$ solves the heat equation, and $u$ is the fundamental solution which is infinitely smooth ,we can take the derivative inside the integral. Indeed,
				\begin{align*}
					\frac{\partial }{\partial t} \int_{-\infty }^{\infty} u \ dx &= \int_{-\infty }^{\infty} \pdv{u}{t} \ dx \overset{u_{t} = u_{xx}}{=} \int_{-\infty}^{\infty } \pdv{u_{x}}{x} \ dx = u_{x}(\infty) - u_{x}(-\infty)
					\intertext{Since $\lim\limits_{t \to -\infty} u_{x} =0$ and $\lim\limits_{t \to \infty} u_{x} =0$ then it follows that }
					E'(t) &= 0.
				\end{align*}
				Now since, 
				\begin{align*} 
				E'(t) =0 \implies \int_{-\infty}^{\infty} u(x,t) \ dx &= \int_{-\infty}^{\infty} u(x,0) \ dx 
				\intertext{Since $g(x)$ is continuous on the specified domain and is integrable, and by theorem }
				\lim\limits_{t \to 0} u(x,t) &= g(x),
				\intertext{It follows that }
				\int_{-\infty}^{\infty} u(x,0) \ dx &= \int_{-\infty}^{\infty} g(x) \ dx  \implies \int_{-\infty}^{\infty} u(x,t) \ dx =\int_{-\infty}^{\infty} g(x) \ dx.
				\end{align*}
			\subsection*{b) }
				\begin{align*}
					\int_{-\infty}^{\infty} u(x,t) \ dx &= \int_{-\infty}^{\infty}  \int_{-\infty}^{\infty}  \Gamma(x-y,t) g(y) \ dy \ dx 
					\intertext{By Fubinni's theorem, we may interchange the order of integration}
					&=   \int_{-\infty}^{\infty}  \int_{-\infty}^{\infty}  \Gamma(x-y, t)g(y) \ dx \ dy
					\intertext{Since $\displaystyle \int_{-\infty}^{\infty} \Gamma(x,t) \ dx = 1 \implies \int_{-\infty}^{\infty} \Gamma(x-y, t) \ dx =1 $, therefore}
					&=  \int_{-\infty}^{\infty} g(y) \ dy
					\intertext{Since $y$ in this case is a dummy variable , we may let $y \to x$ such that }
					 \int_{-\infty}^{\infty}  u(x,t) \ dx &=  \int_{-\infty}^{\infty}  g(x) \ dx.
				\end{align*}
		\section*{Question 8}
			Fixing $t_{0} \in (0,T]$, let 
			\begin{align*}
				r(x,t) &= t \sup\limits_{\mathbb{R} \cross (0,T]} f + \sup g = At +B \\
				p(x,t) &= -t \inf\limits_{\mathbb{R} \cross (0,T]} f - \inf g = -At -B.
			\end{align*}
			Let $(u-r)$ and $(p-v)$ be two solutions. Plugging them in the given PDE we find trivially that for  $(u-r)$ the PDE is $f-\sup f \le 0$ therefore it is a sub solution and similarly for $(p-v)$  the PDE is $-f - \sup f \le 0$ which is also a sub solution.Thus, we may apply the Global Maximum Principle .
			Let $w = u-r$, then 
			\begin{align*}
				w(x,t) &\le \sup\limits_{\mathbb{R}^{n}}w(x,0)  \\
				u-r &\le \sup\limits_{\mathbb{R}^{n}}w(x,0) \\
				u &\le \underbrace{\sup\limits_{\mathbb{R}^{n}}(g(x) - \sup g)}_{\le 0} +r\\
				\therefore u &\le r
			\end{align*} 
			Similarly letting $w = p-u$
			\begin{align*}
			w(x,t) &\le \sup\limits_{\mathbb{R}^{n}}w(x,0)  \\
			p-u &\le \sup\limits_{\mathbb{R}^{n}}w(x,0) \\
			p &\le \underbrace{\sup\limits_{\mathbb{R}^{n}}(g(x) + \sup g)}_{\ge 0} +u\\
			\therefore p &\le u
			\end{align*} 
			We conclude that $Pp \le u \le r$ which translates to 
			$$ -t \inf\limits_{\mathbb{R} \cross (0,T]} f - \inf g \le u(x,t) \le t \sup\limits_{\mathbb{R} \cross (0,T]} f + \sup g,$$
			as it should.
		\section*{Question 9}
			\begin{align*}
				\abs{u(x,t)} = \abs{\int_{-\infty}^{\infty} \Gamma(x-y,t) g(y) \ dy } &\le \int_{-\infty}^{\infty} \abs{\Gamma (x-y,t)} \abs{g(y)} \ dy 
				\intertext{Following the propreties of the absolute value and since the fundamental solution is always positive, }
				&= \int_{-\infty}^{\infty} \Gamma(x-y,t) \abs{g(y)} \ dy \\
				&= \frac{1}{\sqrt{4 \pi k t}} \int_{-\infty}^{\infty} e^{\frac{-(x-y)^{2}}{4 k t}} \abs{g(y)} \ dy
				\intertext{$(x-y)^{2} \ge 0 \ \forall x,y \implies -(x-y)^{2} / 4kt$ is strictly decreasing. Moreover, $\max \exp{-(x-y)^{2} / 4kt} = 1$ and $\min \exp{-(x-y)^{2} / 4kt} =0,$ but it is never reached since it's an exponential function. We conclude that $0 < \exp{-(x-y)^{2} / 4kt} \le 1$, hence it follows that  }
				&\le \frac{1}{\sqrt{4 \pi kt}} \int_{-\infty }^{\infty} \abs{g(y)} \ dy
				\intertext{Since $g$ is bounded by $C$ as defined , we conclude that}
				\abs{u(x,t)} &\le \frac{C}{\sqrt{4 \pi kt}} \ge 0 \\
				\implies u(x,t) &\le \frac{C}{\sqrt{4 \pi k t}}.
			\end{align*}
			It follows evidently that for each $x \in \mathbb{R}$ 
			$$\lim\limits_{t \to \infty } u(x,t) = \frac{C}{\infty} = 0, $$
			as it should.
		\section*{Question 10}
			\subsection*{a) }
			
			\begin{align*}
				p(x,t+\tau) &= \frac14 p(x+he_{1} , t) +\frac14 p(x-he_{1} , t) +\frac14 p(x+he_{2} , t)  +\frac14 p(x-he_{2} , t) \\
				&= p(x,t) + p_{t}(x,t) \tau + \mathcal{O}(\tau) \\
				p(x+he_{1} ,t) &= p(x,t) + p_{x} (x,t) he_{1} +\frac{p_{xx}}{2} (x,t) (he_{1})^{2} + \mathcal{O}(\abs{he_{1}}^{2}) \\ 
				p(x-he_{1} ,t) &= p(x,t) - p_{x} (x,t) he_{1} +\frac{p_{xx}}{2} (x,t) (he_{1})^{2} +\mathcal{O}(\abs{he_{1}}^{2})  \\ 
				p(x+he_{2} ,t) &= p(x,t) + p_{x} (x,t) he_{2} +\frac{p_{xx}}{2} (x,t) (he_{2})^{2}+\mathcal{O}(\abs{he_{2}}^{2})  \\ 
				p(x-he_{2} ,t) &= p(x,t) - p_{x} (x,t) he_{2} +\frac{p_{xx}}{2} (x,t) (he_{2})^{2} +\mathcal{O}(\abs{he_{2}}^{2}) 
			\end{align*}
			Therefore, 
			\begin{align*}
				p(x,t) + p_{t}(x,t) \tau + \mathcal{O}(\tau) &= \frac14 \left(p(x,t) + p_{x} (x,t) he_{1} +\frac{p_{xx}}{2} (x,t) (he_{1})^{2} +\mathcal{O}(\abs{he_{1}}^{2}) \right) \\ 
				&+ \frac14 \left(p(x,t) - p_{x} (x,t) he_{1} +\frac{p_{xx}}{2} (x,t) (he_{1})^{2} +\mathcal{O}(\abs{he_{1}}^{2}) \right) \\
				&+ \frac14\left(p(x,t) + p_{x} (x,t) he_{2} +\frac{p_{xx}}{2} (x,t) (he_{2})^{2}+\mathcal{O}(\abs{he_{2}}^{2}) \right) \\
				&+ \frac14\left(p(x,t) - p_{x} (x,t) he_{2} +\frac{p_{xx}}{2} (x,t) (he_{2})^{2}+\mathcal{O}(\abs{he_{2}}^{2}) \right) \\
				\therefore p_{t}(x,t) \tau + \mathcal{O} (\tau) &= \frac{p_{xx}}{2} (x,t) (he_{1})^{2} + \frac{p_{xx}}{2} (x,t) (he_{2})^{2} + \mathcal{O}(\abs{he_{1}}^{2}) + \mathcal{O}(\abs{he_{1}}^{2}) 
				\intertext{Dividing both sides by $\tau$}
				p_{t}(x,t) +\frac{\mathcal{O}(\tau)}{\tau} &= \frac{p_{xx} (x,t) }{2} \frac{((he_{1})^{2} + (he_{2})^{2})}{\tau} + \frac{\mathcal{O}(\abs{he_{1}}^{2} + \abs{he_{2}}^{2})}{\tau} 
			\end{align*}
			Let us suppose $\lim\limits_{\tau \to 0} ((he_{1})^{2} + (he_{2})^{2}) /\tau \neq 0$ , then $\tau \to 0$ should return something non trivial. Suppose $((he_{1})^{2} + (he_{2})^{2}) /\tau = 2k > 0$. Then sending $\tau \to 0$ and $he_{i} \to 0$, which in return makes the remainder terms vanish results in 
			$$ p_{t} (x,t) - kp_{xx}(x,t) =0.$$
			\subsection*{b) }	
			\begin{align*}
			p(x,t+\tau) &= \frac18 p(x+he_{1} , t) +\frac18 p(x-he_{1} , t) +\frac38 p(x+he_{2} , t)  +\frac38 p(x-he_{2} , t) \\
			&= p(x,t) + p_{t}(x,t) \tau + \mathcal{O}(\tau) \\
			p(x+he_{1} ,t) &= p(x,t) + p_{x} (x,t) he_{1} +\frac{p_{xx}}{2} (x,t) (he_{1})^{2} + \mathcal{O}(\abs{he_{1}}^{2}) \\ 
			p(x-he_{1} ,t) &= p(x,t) - p_{x} (x,t) he_{1} +\frac{p_{xx}}{2} (x,t) (he_{1})^{2} +\mathcal{O}(\abs{he_{1}}^{2})  \\ 
			p(x+he_{2} ,t) &= p(x,t) + p_{x} (x,t) he_{2} +\frac{p_{xx}}{2} (x,t) (he_{2})^{2}+\mathcal{O}(\abs{he_{2}}^{2})  \\ 
			p(x-he_{2} ,t) &= p(x,t) - p_{x} (x,t) he_{2} +\frac{p_{xx}}{2} (x,t) (he_{2})^{2} +\mathcal{O}(\abs{he_{2}}^{2}) 
			\end{align*}
			Therefore, 
			\begin{align*}
			p(x,t) + p_{t}(x,t) \tau + \mathcal{O}(\tau) &= \frac18 \left(p(x,t) + p_{x} (x,t) he_{1} +\frac{p_{xx}}{2} (x,t) (he_{1})^{2} +\mathcal{O}(\abs{he_{1}}^{2}) \right) \\ 
			&+ \frac18 \left(p(x,t) - p_{x} (x,t) he_{1} +\frac{p_{xx}}{2} (x,t) (he_{1})^{2} +\mathcal{O}(\abs{he_{1}}^{2}) \right) \\
			&+ \frac38\left(p(x,t) + p_{x} (x,t) he_{2} +\frac{p_{xx}}{2} (x,t) (he_{2})^{2}+\mathcal{O}(\abs{he_{2}}^{2}) \right) \\
			&+ \frac38\left(p(x,t) - p_{x} (x,t) he_{2} +\frac{p_{xx}}{2} (x,t) (he_{2})^{2}+\mathcal{O}(\abs{he_{2}}^{2}) \right) \\
			\therefore p_{t}(x,t) \tau + \mathcal{O} (\tau) &= \frac{p_{xx}}{8} (x,t) (he_{1})^{2} + \frac{3 p_{xx}}{8} (x,t) (he_{2})^{2} + \mathcal{O}(\abs{he_{1}}^{2}) + \mathcal{O}(\abs{he_{1}}^{2}) \tag{1}
			\end{align*}
			Let 
			$$ A(x)D^{2}p \equiv  \begin{pmatrix}
				\frac{1}{8} he_{1} & 0 \\ 0 & \frac{3}{8} he_{2}
			\end{pmatrix} p_{xx}(x,t),$$
			then dividing by $\tau$ on both sides of $(1)$ we have 
			\begin{align*}
				p_{t}(x,t) + \frac{\mathcal{O}}{\tau} - \text{tr}A(x)\frac{D^{2}p}{\tau}  &= \frac{\mathcal{O(\tau)}(\abs{he_{1}}^{2} + \abs{he_{2}}^{2})}{\tau} 
			\end{align*}
				Let us suppose $\lim\limits_{\tau \to 0} ((he_{1})^{2} + (he_{2})^{2}) /\tau \neq 0$ , then $\tau \to 0$ should return something non trivial. Suppose $((he_{1})^{2} + (he_{2})^{2}) /\tau = 2k > 0$. Then sending $\tau \to 0$ and $he_{i} \to 0$, which in return makes the remainder terms vanish results in 
				$$ p_{t} - \text{tr}(A(x)D^{2}p) = 0.$$
				We note there is no drift because the random walk system as defined is stable. Indeed, the probabilities in each direction are symmetric such that there's no preferred direction for a given orientation. 
	\end{document}