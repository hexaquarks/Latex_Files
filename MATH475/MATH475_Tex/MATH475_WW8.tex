\documentclass[12pt]{article}
\newcommand\hmmax{0}
\newcommand\bmmax{0}
\usepackage{xcolor}
\usepackage[dvipsnames]{xcolor}
\usepackage[many]{tcolorbox}
\usepackage{changepage}
\usepackage{titlesec}
\usepackage{caption}
\usepackage{mdframed, longtable}
\usepackage{mathtools, amssymb, amsfonts, amsthm, bm,amsmath} 
\usepackage{array, tabularx, booktabs}
\usepackage{graphicx,wrapfig, float, caption}
\usepackage{tikz,physics,cancel, siunitx, xfrac}
\usepackage{graphics, fancyhdr}
\usepackage{lipsum}
\usepackage{xparse}
\usepackage{thmtools}
\usepackage{mathrsfs}
\usepackage{undertilde}
\usepackage{tikz}
\usepackage{fullpage,enumitem}
\usepackage[labelfont=bf]{caption}
\newcommand{\td}{\text{dim}}
\newcommand{\tvw}{T : V\xrightarrow{} W }
\newcommand{\ttt}{\widetilde{T}}
\newcommand{\ex}{\textbf{Example}}
\newcommand{\aR}{\alpha \in \mathbb{R}}
\newcommand{\abR}{\alpha \beta \in \mathbb{R}}
\newcommand{\un}{u_1 , u_2 , \dots , n}
\newcommand{\an}{\alpha_1, \alpha_2, \dots, \alpha_2 }
\newcommand{\sS}{\text{Span}(\mathcal{S})}
\newcommand{\sSt}{($\mathcal{S}$)}
\newcommand{\la}{\langle}
\newcommand{\ra}{\rangle}
\newcommand{\Rn}{\mathbb{R}^{n}}
\newcommand{\R}{\mathbb{R}}
\newcommand{\Rm}{\mathbb{R}^{m}}
\usepackage{fullpage, fancyhdr}
\newcommand{\La}{\mathcal{L}}
\newcommand{\ep}{\epsilon}
\newcommand{\de}{\delta}
\usepackage[math]{cellspace}
\setlength{\cellspacetoplimit}{3pt}
\setlength{\cellspacebottomlimit}{3pt}
\newcommand\numberthis{\addtocounter{equation}{1}\tag{\theequation}}
\usepackage{newtxtext, newtxmath}
\usepackage{bbm}


\usepackage{mathtools}
\DeclarePairedDelimiter{\norm}{\lVert}{\rVert}
\newcommand{\vectorproj}[2][]{\textit{proj}_{\vect{#1}}\vect{#2}}
\newcommand{\vect}{\mathbf}
\newcommand{\uuuu}{\sum_{i=1}^{n}\frac{<u,u_i}{<u_i,u_i>} u_i}
\newcommand{\Ss}{\mathcal{S}}
\newcommand{\A}{\hat{A}}
\newcommand{\B}{\hat{B}}
\newcommand{\C}{\hat{C}}
\newcommand{\dr}{\mathrm{d}}
\allowdisplaybreaks
\usepackage{titling}
\newtheorem{theorem}{Theorem}[section]
\theoremstyle{definition}
\newtheorem{corollary}{Corollary}[theorem]
\theoremstyle{definition}
\newtheorem{lemma}[theorem]{Lemma}
\theoremstyle{definition}
\newtheorem{definition}{Definition}[section]
\theoremstyle{definition}
\newtheorem{Proposition}{Proposition}[section]
\theoremstyle{definition}
\newtheorem*{example}{Example}
\theoremstyle{example}
\newtheorem*{note}{Note}
\theoremstyle{note}
\newtheorem*{remark}{Remark}
\theoremstyle{remark}
\newtheorem*{example2}{External Example}
\theoremstyle{example}
\usepackage{bbold}
\title{MATH475 Weekly Work 8}
\titleformat*{\section}{\LARGE\normalfont\fontsize{14}{14}\bfseries}
\titleformat*{\subsection}{\Large\normalfont\fontsize{12}{15}\bfseries}
\author{Mihail Anghelici 260928404 }
\date{\today}

\relpenalty=9999
\binoppenalty=9999

\renewcommand{\sectionmark}[1]{%
	\markboth{\thesection\quad #1}{}}

\fancypagestyle{plain}{%
	\fancyhf{}
	\fancyhead[L]{\rule[0pt]{0pt}{0pt} Weekly Work 8} 
	\fancyhead[R]{\small Mihail Anghelici $260928404$} 
	\fancyfoot[C]{-- \thepage\ --}
	\renewcommand{\headrulewidth}{0.4pt}}
\pagestyle{plain}
\setlength{\headsep}{1cm}
\captionsetup{margin =1cm}
	\begin{document}
	\maketitle
		\section*{Question 1}
			\begin{gather*}
				 \pdv{}{x} u_{0}(x- u t) = u_{0}' \left(\pdv{}{x} \left(x  - ut \right)\right) = u_{0}' \left(1 - \pdv{}{x} \left(ut\right)\right) = u_{0}' (1- u_{x}t) 
				 \intertext{Rearranging the expression and setting the argument of $u_{0}'$ to $x_{0}$ we get }
				 u_{x}(x,t) = \frac{u_{0}'(x_{0})}{1 + u_{0}'(x_{0})t} 
			\end{gather*}
		\section*{Question 2}
			\begin{gather*}
				 \dv{x}{\tau} =q'(z) ; \quad\quad \dv{t}{\tau} =1 ; \quad\quad \dv{z}{t} = 0 \\
				\quad x(s,0) = s; \qquad t(s,0) = 0; \qquad z(s,0) = u_{0}(s)
			\end{gather*}
			Thus, we obtain
			$$ z = u_{0}(s) ; \qquad t = \tau ; \qquad x = q'(u)t + s.$$
			Therefore we convert to $u(x,t)$ 
			$$ s = x - q'(u)t  \implies u(x,t) = u_{0}(x - q'(u) t).$$
			As a double check, 
			$$ \pdv{}{x} u_{0}(x - q'(u) t) \implies u_{x}(x,t) = \frac{u_{0}'(x_{0})}{1 + q''(u) u_{0}'(x_{0})t }$$
		\section*{Question 3}
			A sufficient condition for $u(x,t)$ to remain smooth for all $t$ is to require $q''(u)u_{0}'(x) \ge 0 \ \forall \ x \in \mathbb{R}$. \\
			
			\noindent If this condition is violated, i.e., $\exists \ x_{0}$ such that $q''(u) u_{0}'(x_{0}) <0$ then since 
			$$ u_{x} \to -\infty \quad \text{as} \: t \to \frac{-1}{q''(u) u_{0}' (x_{0})},$$
			then 
			$$ t^{\ast} = \min\limits_{x \in \mathbb{R}} \Biggl\{\frac{-1}{q''(u) u_{0}' (x_{0})} \  \Bigg| \  q''(u)u_{0}'(x) < 0\Biggr\} = \frac{-1}{q''(u)u_{0}(x_{0})}.$$
	\end{document}