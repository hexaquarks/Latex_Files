\documentclass[
	12pt,
	]{article}
		\usepackage{xcolor}
			\usepackage[dvipsnames]{xcolor}
			\usepackage[many]{tcolorbox}
		\usepackage{changepage}
		\usepackage{titlesec}
		\usepackage{caption}
		\usepackage{mdframed, longtable}
		\usepackage{mathtools, amssymb, amsfonts, amsthm, bm,amsmath} 
		\usepackage{array, tabularx, booktabs}
		\usepackage{graphicx,wrapfig, float, caption}
		\usepackage{tikz,physics,cancel, siunitx, xfrac}
		\usepackage{graphics, fancyhdr}
		\usepackage{lipsum}
		\usepackage{xparse}
		\usepackage{thmtools}
		\usepackage{mathrsfs}
		\usepackage{undertilde}
		\usepackage{tikz}
		\usepackage{fullpage,enumitem}
		\usepackage[labelfont=bf]{caption}
	\newcommand{\td}{\text{dim}}
	\newcommand{\tvw}{T : V\xrightarrow{} W }
	\newcommand{\ttt}{\widetilde{T}}
	\newcommand{\ex}{\textbf{Example}}
	\newcommand{\aR}{\alpha \in \mathbb{R}}
	\newcommand{\abR}{\alpha \beta \in \mathbb{R}}
	\newcommand{\un}{u_1 , u_2 , \dots , n}
	\newcommand{\an}{\alpha_1, \alpha_2, \dots, \alpha_2 }
	\newcommand{\sS}{\text{Span}(\mathcal{S})}
	\newcommand{\sSt}{($\mathcal{S}$)}
	\newcommand{\la}{\langle}
	\newcommand{\ra}{\rangle}
	\newcommand{\Rn}{\mathbb{R}^{n}}
	\newcommand{\R}{\mathbb{R}}
	\newcommand{\Rm}{\mathbb{R}^{m}}
	\usepackage{fullpage, fancyhdr}
	\newcommand{\La}{\mathcal{L}}
	\newcommand{\ep}{\epsilon}
	\newcommand{\de}{\delta}
	\usepackage[math]{cellspace}
		\setlength{\cellspacetoplimit}{3pt}
		\setlength{\cellspacebottomlimit}{3pt}
	\newcommand\numberthis{\addtocounter{equation}{1}\tag{\theequation}}


	\usepackage{mathtools}
	\DeclarePairedDelimiter{\norm}{\lVert}{\rVert}
	\newcommand{\vectorproj}[2][]{\textit{proj}_{\vect{#1}}\vect{#2}}
	\newcommand{\vect}{\mathbf}
	\newcommand{\uuuu}{\sum_{i=1}^{n}\frac{<u,u_i}{<u_i,u_i>} u_i}
	\newcommand{\B}{\mathcal{B}}
	\newcommand{\Ss}{\mathcal{S}}
	
	\newtheorem{theorem}{Theorem}[section]
	\theoremstyle{definition}
	\newtheorem{corollary}{Corollary}[theorem]
	\theoremstyle{definition}
	\newtheorem{lemma}[theorem]{Lemma}
	\theoremstyle{definition}
	\newtheorem{definition}{Definition}[section]
	\theoremstyle{definition}
	\newtheorem{Proposition}{Proposition}[section]
	\theoremstyle{definition}
	\newtheorem*{example}{Example}
	\theoremstyle{example}
	\newtheorem*{note}{Note}
	\theoremstyle{note}
	\newtheorem*{remark}{Remark}
	\theoremstyle{remark}
	\newtheorem*{example2}{External Example}
	\theoremstyle{example}
	
	\title{MATH 475 Weekly Work 3}
	\titleformat*{\section}{\LARGE\normalfont\fontsize{12}{12}\bfseries}
	\titleformat*{\subsection}{\Large\normalfont\fontsize{10}{15}\bfseries}
	\author{Mihail Anghelici 260928404 }
	\date{\today}
	
	\relpenalty=9999
			\binoppenalty=9999
		
			\renewcommand{\sectionmark}[1]{%
			\markboth{\thesection\quad #1}{}}
			
			\fancypagestyle{plain}{%
			  \fancyhf{}
			  \fancyhead[L]{\rule[0pt]{0pt}{0pt} Weekly Work 3 } 
			  \fancyhead[R]{\small Mihail Anghelici $260928404$} 
			  \fancyfoot[C]{-- \thepage\ --}
			  \renewcommand{\headrulewidth}{0.4pt}}
			\pagestyle{plain}
			\setlength{\headsep}{1cm}
	\captionsetup{margin =1cm}
	\begin{document}
	\maketitle
		\section*{Question 1}
			\begin{align*}
				u(x,t) &= \int_{-\infty}^{\infty} \Gamma_{k}(x-y,t) g(y) \ dy \\
				&= \int_{-\infty}^{\infty} \frac{1}{\sqrt{4 \pi k t}} e^{\frac{-(x-y)^{2}}{4 kt}} e^{-y} \ dy 
			\end{align*}
			Next we complete the square in the exponential of $\exp(\frac{-(x-y)^{2}}{4kt} -y)$ with respect to the $y$ variable such that the denominator is a function of $4kt$ and there is a left-over term not involving $y$. From 
			\begin{align*}
				(y+2kt -x)^{2} &= \frac{y4kt +x^{2} -2xy +y^{2}}{4kt} -x+kt \\
				\implies -(y+2kt -x)^{2} &= \frac{-y4kt -x^{2} + 2xy -y^{2}}{4kt} -x +kt 
				\intertext{So we add $-x +kt$, essentially the square is completed by }
			\end{align*}
			\vspace{-0.2cm}
			$$ \frac{-(y+2kt -x)^{2}}{4kt} -x +kt.$$
			We continue evaluating the integral by setting $p = \frac{-(y+2kt -x)^{2}}{4kt} -x +kt \implies dp = \frac{dy}{\sqrt{4kt}}$
			\begin{align*}
				u(x,t) &= \int_{-\infty}^{\infty} \frac{1}{\sqrt{4\pi k t}}e^{-p^{2}} e^{kt -x} dp \sqrt{4kt} \\
				&= \frac{e^{kt -x }}{\sqrt{\pi}} \int_{-\infty}^{\infty} e^{-p^{2}} \ dp \\
				\intertext{This is the gaussian function solved commonly in polar coordinates}
				u(x,t) &= e^{kt -x}.
			\end{align*}
		\section*{Question 2}
			Let $\bar{u}(x,t) \equiv u(x,t) -1$ such that now the boundary condition is $0$. Then 
			
					$$g_{\text{odd}} = \begin{cases*}
							-1 \quad x &\ge 0 \\
							1 \quad x & $<$ 0
					\end{cases*}$$
			\begin{align*}
			\bar{u} (x,t) &= \int_{-\infty}^{\infty} \frac{1}{\sqrt{4 \pi k t}} e^{\frac{-(x-y)^{2}}{4kt}} g_{\text{odd}} \\
				&= \int_{-\infty}^{0}\frac{1}{\sqrt{4 \pi k t}} e^{\frac{-(x-y)^{2}}{4kt}}(1) - \int_{\infty}^{0}\frac{1}{\sqrt{4 \pi k t}} e^{\frac{-(x-y)^{2}}{4kt}}(1)
				\intertext{Let $-y=y'$ in the second integral ,then}
				&= \int_{-\infty }^{0} \frac{1}{\sqrt{4 \pi k t}} e^{\frac{-(x-y)^{2}}{4kt}}\ dy + \int_{-\infty}^{0} \frac{1}{\sqrt{4 \pi k t}} e^{\frac{-(x+y')^{2}}{4kt}} \ dy'
				\intertext{Let $r=x-y / \sqrt{4kt} \implies dr = dy/ \sqrt{4 k t}$ and $r = x+y'/\sqrt{4 k t} \implies dr = dy' / \sqrt{4kt}$} 
				&= \int_{-\infty }^{\sfrac{x}{\sqrt{4 k t}}} \frac{1}{\sqrt{\pi}} e^{-r^{2}} \ dr + \int_{-\infty }^{\sfrac{x}{\sqrt{4 k t}}}\frac{1}{\sqrt{\pi}}e^{-r^{2}} \ dr \\
				&= 2 F\left(\frac{x}{\sqrt{4kt}}\right).
			\end{align*}
			Now since $\bar{u}(x,t) = u(x,t) -1 \implies u(x,t) = \bar{u}(x,t) +1 \equiv F'(y)+1$. Then since $1- F(y)=F'(y) \implies 2 - F(y) =F'(y) +1$ we conclude 
			$$ u(x,t) = 2-2F\left(\frac{x}{\sqrt{4 k t}}\right).$$
		\section*{Question 3}
			\subsection*{a) }
			\begin{align*}
				f_{-} ' (0) = \lim\limits_{h \to 0^{-}}\frac{f(0+h) - f(0)}{h}  \\
				f_{+} ' (0) = \lim\limits_{h \to 0^{+}}\frac{f(0+h) - f(0)}{h}
			\end{align*}
			Letting the right derivative be $f(-h)$ since the function is even , equating we get 
			$$ f(-h) = f(h) \implies f'(h) = 0 \quad \text{as} \  h  \ \text{approaches } 0,$$
			hence $u_{x} (0,t) = 0 $.We chose $g(y)$ to be even instead of odd since the derivative of an even function is an odd function. Let 
			$$g_{\text{even}} = \begin{cases*}
			g(x) \quad x &\ge 0 \\
			g(-x) \quad x & $<$ 0
			\end{cases*}$$
			 Therefore,
			
			\begin{align*}
				u(x,t) &= \int_{-\infty }^{\infty} \frac{1}{\sqrt{4 \pi k t}} e^{\frac{-(x-y)^{2}}{4 k t}} g_{\text{even}} \ dy \\
					&= \int_{0}^{\infty } \frac{1}{\sqrt{4 \pi k t}} e^{\frac{-(x-y)^{2}}{4 k t}} g(y) \ dy +\int_{-\infty }^{0} \frac{1}{\sqrt{4 \pi k t}} e^{\frac{-(x-y)^{2}}{4 k t}} g(-y) \ dy
					\intertext{Let $y = -y$ in the second integral , we get }
					&=  \int_{0}^{\infty } \frac{1}{\sqrt{4 \pi k t}} e^{\frac{-(x-y)^{2}}{4 k t}} g(y) \ dy -\int_{\infty }^{0} \frac{1}{\sqrt{4 \pi k t}} e^{\frac{-(x+y')^{2}}{4 k t}} g(y') \ dy'\\
					&=\int_{0}^{\infty } \frac{1}{\sqrt{4 \pi k t}} e^{\frac{-(x-y)^{2}}{4 k t}} g(y) \ dy  +\int_{0}^{\infty } \frac{1}{\sqrt{4 \pi k t}} e^{\frac{-(x+y')^{2}}{4 k t}} g(y') \ dy'
					\intertext{Letting $y=y'$, we end up }                                                    
					u(x,t) &= \frac{1}{\sqrt{4 \pi k t}} \int_{0}^{\infty }\left(e^{\frac{-(x-y)^{2}}{4 k t}} + e^{\frac{-(x+y)^{2}}{4 k t}}\right) g(y) \ dy 
			\end{align*}
			\subsection*{b) }
				Since $g(x) =1 $ then we have 
				$$g_{\text{even}} = \begin{cases*}
				1 \quad x &\ge 0 \\
				1 \quad x & $<$ 0
				\end{cases*}$$
				Therefore, 
				\begin{align*}
					u(x,t) &= \int_{0}^{\infty }\frac{1}{\sqrt{4 \pi k t}} \left(e^{ \frac{-(x-y)^{2}}{4 k t}} + e^{\frac{-(x+y)^{2}}{4 k t}}\right) \ dy
					\intertext{Letting $r = x-y/ \sqrt{4 k t} \implies dr = -dy/\sqrt{4kt}$ in the left integral and $r = x+y/ \sqrt{4 k t} \implies dr = dy / \sqrt{4 k t}$ in the right integral we get}
					 &= - \int_{\sfrac{x}{\sqrt{4 k t}}}^{- \infty } \frac{1}{\sqrt{\pi}} e^{-r^{2}} \ dr + \int_{\sfrac{x}{\sqrt{4kt}}}^{\infty }\frac{1}{\sqrt{\pi }} e^{-r^{2}} \ dr
					 \intertext{Letting $r \to -r$ in the second integral , we conclude }
					 &= \int_{\sfrac{x}{\sqrt{4 k t}}}^{\infty } \frac{1}{\sqrt{\pi}}e^{-r^{2}} \ dr + \int_{\sfrac{x}{\sqrt{4 k t}}}^{\infty } \frac{1}{\sqrt{\pi}}e^{-r^{2}} \ dr \\
					 &= \frac{2}{\sqrt{\pi}} \int_{\sfrac{x}{\sqrt{4 k t}}}^{\infty } e^{-r^{2}} \ dr.
				\end{align*}
	\end{document}