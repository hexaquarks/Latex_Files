\documentclass[
	12pt,
	]{article}
		\usepackage{xcolor}
			\usepackage[dvipsnames]{xcolor}
			\usepackage[many]{tcolorbox}
		\usepackage{changepage}
		\usepackage{titlesec}
		\usepackage{caption}
		\usepackage{mdframed, longtable}
		\usepackage{mathtools, amssymb, amsfonts, amsthm, bm,amsmath} 
		\usepackage{array, tabularx, booktabs}
		\usepackage{graphicx,wrapfig, float, caption}
		\usepackage{tikz,physics,cancel, siunitx, xfrac}
		\usepackage{graphics, fancyhdr}
		\usepackage{lipsum}
		\usepackage{xparse}
		\usepackage{thmtools}
		\usepackage{mathrsfs}
		\usepackage{undertilde}
		\usepackage{tikz}
		\usepackage{fullpage,enumitem}
		\usepackage[labelfont=bf]{caption}
	\newcommand{\td}{\text{dim}}
	\newcommand{\tvw}{T : V\xrightarrow{} W }
	\newcommand{\ttt}{\widetilde{T}}
	\newcommand{\ex}{\textbf{Example}}
	\newcommand{\aR}{\alpha \in \mathbb{R}}
	\newcommand{\abR}{\alpha \beta \in \mathbb{R}}
	\newcommand{\un}{u_1 , u_2 , \dots , n}
	\newcommand{\an}{\alpha_1, \alpha_2, \dots, \alpha_2 }
	\newcommand{\sS}{\text{Span}(\mathcal{S})}
	\newcommand{\sSt}{($\mathcal{S}$)}
	\newcommand{\la}{\langle}
	\newcommand{\ra}{\rangle}
	\newcommand{\Rn}{\mathbb{R}^{n}}
	\newcommand{\R}{\mathbb{R}}
	\newcommand{\Rm}{\mathbb{R}^{m}}
	\usepackage{fullpage, fancyhdr}
	\newcommand{\La}{\mathcal{L}}
	\newcommand{\ep}{\epsilon}
	\newcommand{\de}{\delta}
	\usepackage[math]{cellspace}
		\setlength{\cellspacetoplimit}{3pt}
		\setlength{\cellspacebottomlimit}{3pt}
	\newcommand\numberthis{\addtocounter{equation}{1}\tag{\theequation}}


	\usepackage{mathtools}
	\DeclarePairedDelimiter{\norm}{\lVert}{\rVert}
	\newcommand{\vectorproj}[2][]{\textit{proj}_{\vect{#1}}\vect{#2}}
	\newcommand{\vect}{\mathbf}
	\newcommand{\uuuu}{\sum_{i=1}^{n}\frac{<u,u_i}{<u_i,u_i>} u_i}
	\newcommand{\B}{\mathcal{B}}
	\newcommand{\Ss}{\mathcal{S}}
	
	\newtheorem{theorem}{Theorem}[section]
	\theoremstyle{definition}
	\newtheorem{corollary}{Corollary}[theorem]
	\theoremstyle{definition}
	\newtheorem{lemma}[theorem]{Lemma}
	\theoremstyle{definition}
	\newtheorem{definition}{Definition}[section]
	\theoremstyle{definition}
	\newtheorem{Proposition}{Proposition}[section]
	\theoremstyle{definition}
	\newtheorem*{example}{Example}
	\theoremstyle{example}
	\newtheorem*{note}{Note}
	\theoremstyle{note}
	\newtheorem*{remark}{Remark}
	\theoremstyle{remark}
	\newtheorem*{example2}{External Example}
	\theoremstyle{example}
	
	\title{MATH 475 Weekly Work 4}
	\titleformat*{\section}{\LARGE\normalfont\fontsize{12}{12}\bfseries}
	\titleformat*{\subsection}{\Large\normalfont\fontsize{10}{15}\bfseries}
	\author{Mihail Anghelici 260928404 }
	\date{\today}
	
	\relpenalty=9999
			\binoppenalty=9999
		
			\renewcommand{\sectionmark}[1]{%
			\markboth{\thesection\quad #1}{}}
			
			\fancypagestyle{plain}{%
			  \fancyhf{}
			  \fancyhead[L]{\rule[0pt]{0pt}{0pt} Weekly Work 4} 
			  \fancyhead[R]{\small Mihail Anghelici $260928404$} 
			  \fancyfoot[C]{-- \thepage\ --}
			  \renewcommand{\headrulewidth}{0.4pt}}
			\pagestyle{plain}
			\setlength{\headsep}{1cm}
	\captionsetup{margin =1cm}
	\begin{document}
	\maketitle
		\section*{Question 1}
			By Duhamel method, let $v$ solve 
			$$ \begin{cases}
				v_{t} - kv_{xx} = 0 \qquad &\text{in } \ 1_{T}, \\
				v(0,t) = 0 \quad, v(1,t) = 0 \qquad &\text{in } \ (0,T], \\
				v(x,0;s) = s \sin(2 \pi x) \qquad &\text{in } \ (0,1).
			\end{cases}$$
			We use separation of variables
			\begin{align*}
				v(x,t) &= e^{-k \lambda^{2} t} (A \cos \lambda x + B \sin \lambda x) \\
				v(0,t) &= 0 \implies A = 0 \\
				v(1,t) &= 0 \implies \lambda =n \pi.
			\end{align*}
			We write the general solution 
			\begin{align*}
				v(x,t;s) &= \sum_{n=0}^{\infty} B_{n}e^{-k n^{2} \pi^{2} t} \sin(n \pi  x) \\
				v(x,0;s) &= s \sin (2\pi x) \implies B_{1} =0 \ \text{ and } \ B_{2} = s
			\end{align*}
			Finally, 
			$$ v(x,t-s;s) = s e^{-k 4 \pi^{2} (t-s)} \sin (2\pi x).$$
			Let $\alpha = 4k \pi^{2}$ then we proceed with the method 
			\begin{align*}
				u(x,t) &= \int_{0}^{t} s e^{-\alpha(t-s)}  sin(2\pi x) \ ds
				\intertext{Let $u = -\alpha t + \alpha s \implies du = \alpha ds$ and $s = (u+\alpha t / \alpha)$, thus } 
				&= \frac{\sin(2 \pi x)}{\alpha^{2}} \left(\int_{-\alpha t}^{0} ue^{u} \ du + \int_{-\alpha t}^{0} e^{u} \alpha t \ du \right) \\
				&=\frac{\sn (2 \pi x)}{\alpha^{2}} \Bigg[e^{u} u \Big|_{-\alpha t}^{0} - \int_{-\alpha t}^{0} e^{u} \ du + \int_{-\alpha t}^{0} e^{u} \alpha t \ du\Bigg] \\
				u(x,t) &= \frac{\sin(2 \pi x) (-1 + e^{-\alpha t} + \alpha t)}{\alpha^{2}},
			\end{align*}
			for $\alpha = 4 k \pi^{2}.$
		\section*{Question 2}
			Let $u(x,t) = v(x,t) + w(x,t)$ and let $w(x,t) = A(t) x + B(t)$ be a solution. The function that satisfies the given boundary conditions for $w(x,t) $ is $(t^{2} -1 )x + 1$. The $v(x,t) = u(x,t) - w(x,t)$ solves
			$$ \begin{cases}
				v_{t} -kv_{xx} = u_{x} - ku_{xx} - ((t^{2} -1)x +1)_{t} - ((t^{2} -1)x +1)_{xx} = t \sin (2 \pi x) \quad &\text{in } \ 1_{T},\\
				v(0,t) = v(1,t) = 1-1 =0 \qquad &\text{in } \ (0,T], \\ 
				v(x,0) = u(x,0) - w(x,0) = 1-x - (-x+1) = 0 \qquad &\text{in } \ (0,1).
			\end{cases}$$
			Then we apply Duhamel's method for $v'(x,t;s)$, the result of this particular situation is already computed in Question 1 : 
			\begin{align*}
				v'(x,t-s;s) &= s e^{-k 4 \pi^{2} (t-s)} \sin (2\pi x) \\
				\implies v(x,t) &= \int_{0}^{t} s e^{-k 4 \pi^{2} (t-s)} \sin (2\pi x) \ ds \\
				&= \frac{\sin(2 \pi x)(-1+ e^{-\alpha t} + \alpha t )}{\alpha^{2}},
			\end{align*} 
			And so finally, 
			$$ u(x,t) = v(x,t) + w(x,t) = \frac{\sin(2 \pi x)(-1 e^{-\alpha t} + \alpha t )}{\alpha^{2}} + (t^{2} -1 )x + 1.$$
		\section*{Question 3}
			Let $u(x,t) = A(t)x + B(t)$, then from the BVP's boundary condition, $A(t) = t^{2}$. We then can write $B(t) = u(x,t) - t^{2}(x)$. Then $B(t)$ solves 
			$$ \begin{cases}
				B_{t} - B_{xx} = u_{t} - u_{xx} - (t^{2} x)_{t} -(t^{2} x)_{xx} = -2tx \qquad &\text{in } \ (0,\infty) \cross (0,T], \\
				B(x,0) =0 \qquad \ &\text{in } \ (0,\infty ), \\
				B_{x}(0,t)  = t^{2} - t^{2} = 0 \qquad &\text{in } \ (0,T].
			\end{cases}$$
			By Duhamel's method, we let $v$ solve
				$$ \begin{cases}
			v_{t} - v_{xx} = 0 \qquad &\text{in } \ (0,\infty) \cross (0,T], \\
			v(x,0;s) =-2sx \qquad \ &\text{in } \ (0,\infty ), \\
			v_{x}(0,t) = 0 \qquad &\text{in } \ (0,T].
			\end{cases}$$
			We can solve this for the half-line using the reflection method. Neumann boundaries so we take $g_{\text{even}}$. Let 
			$$ g_{\text{even}}(x)  = \begin{cases}
				-2sx \qquad , &x \ge 0  \\
				2sx \qquad , &x \le 0
			\end{cases}$$
			As it was shown in Weekly worksheet 3, for even function 
			$$ u(x,t) = \frac{1}{\sqrt{4 \pi k t}} \int_{0}^{\infty} \left(e^{\frac{-(x-y)^{2}}{4kt}} + e^{\frac{-(x+y)^{2}}{4 kt}}\right)g(y) \ dy,$$
			therefore in our case ,
			\begin{gather*}
				v(x,t-s;s) = \frac{-2s}{\sqrt{4 \pi k(t-s)}} \int_{0}^{\infty} \left(e^{\frac{-(x-y)^{2}}{4kt}} + e^{\frac{-(x+y)^{2}}{4 kt}}\right)y \ dy,
				\intertext{Since $u(x,t) = B(t) + t^{2}(x)$ , it follows that }
				u(x,t) =\int_{0}^{t} \frac{-2s}{\sqrt{4 \pi k(t-s)}} \int_{0}^{\infty} \left(e^{\frac{-(x-y)^{2}}{4kt}} + e^{\frac{-(x+y)^{2}}{4 kt}}\right)y \ dy \ ds + t^{2}x.
			\end{gather*}
	\end{document}