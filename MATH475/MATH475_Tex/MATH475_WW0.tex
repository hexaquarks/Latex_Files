\documentclass[
	12pt,
	]{article}
		\usepackage{xcolor}
			\usepackage[dvipsnames]{xcolor}
			\usepackage[many]{tcolorbox}
		\usepackage{changepage}
		\usepackage{titlesec}
		\usepackage{caption}
		\usepackage{mdframed, longtable}
		\usepackage{mathtools, amssymb, amsfonts, amsthm, bm,amsmath} 
		\usepackage{array, tabularx, booktabs}
		\usepackage{graphicx,wrapfig, float, caption}
		\usepackage{tikz,physics,cancel, siunitx, xfrac}
		\usepackage{graphics, fancyhdr}
		\usepackage{lipsum}
		\usepackage{xparse}
		\usepackage{thmtools}
		\usepackage{mathrsfs}
		\usepackage{undertilde}
		\usepackage{tikz}
		\usepackage{fullpage,enumitem}
		\usepackage[labelfont=bf]{caption}
	\newcommand{\td}{\text{dim}}
	\newcommand{\tvw}{T : V\xrightarrow{} W }
	\newcommand{\ttt}{\widetilde{T}}
	\newcommand{\ex}{\textbf{Example}}
	\newcommand{\aR}{\alpha \in \mathbb{R}}
	\newcommand{\abR}{\alpha \beta \in \mathbb{R}}
	\newcommand{\un}{u_1 , u_2 , \dots , n}
	\newcommand{\an}{\alpha_1, \alpha_2, \dots, \alpha_2 }
	\newcommand{\sS}{\text{Span}(\mathcal{S})}
	\newcommand{\sSt}{($\mathcal{S}$)}
	\newcommand{\la}{\langle}
	\newcommand{\ra}{\rangle}
	\newcommand{\Rn}{\mathbb{R}^{n}}
	\newcommand{\R}{\mathbb{R}}
	\newcommand{\Rm}{\mathbb{R}^{m}}
	\usepackage{fullpage, fancyhdr}
	\newcommand{\La}{\mathcal{L}}
	\usepackage[math]{cellspace}
		\setlength{\cellspacetoplimit}{3pt}
		\setlength{\cellspacebottomlimit}{3pt}


	\usepackage{mathtools}
	\DeclarePairedDelimiter{\norm}{\lVert}{\rVert}
	\newcommand{\vectorproj}[2][]{\textit{proj}_{\vect{#1}}\vect{#2}}
	\newcommand{\vect}{\mathbf}
	\newcommand{\uuuu}{\sum_{i=1}^{n}\frac{<u,u_i}{<u_i,u_i>} u_i}
	\newcommand{\B}{\mathcal{B}}
	\newcommand{\Ss}{\mathcal{S}}
	
	\newtheorem{theorem}{Theorem}[section]
	\theoremstyle{definition}
	\newtheorem{corollary}{Corollary}[theorem]
	\theoremstyle{definition}
	\newtheorem{lemma}[theorem]{Lemma}
	\theoremstyle{definition}
	\newtheorem{definition}{Definition}[section]
	\theoremstyle{definition}
	\newtheorem{Proposition}{Proposition}[section]
	\theoremstyle{definition}
	\newtheorem*{example}{Example}
	\theoremstyle{example}
	\newtheorem*{note}{Note}
	\theoremstyle{note}
	\newtheorem*{remark}{Remark}
	\theoremstyle{remark}
	\newtheorem*{example2}{External Example}
	\theoremstyle{example}
	
	\title{MATH 475 Weekly Work 0}
	\titleformat*{\section}{\LARGE\normalfont\fontsize{12}{12}\bfseries}
	\titleformat*{\subsection}{\Large\normalfont\fontsize{10}{15}\bfseries}
	\author{Mihail Anghelici 260928404 }
	\date{\today}
	
	\relpenalty=9999
			\binoppenalty=9999
		
			\renewcommand{\sectionmark}[1]{%
			\markboth{\thesection\quad #1}{}}
			
			\fancypagestyle{plain}{%
			  \fancyhf{}
			  \fancyhead[L]{\rule[0pt]{0pt}{0pt} Weekly Work 0 } 
			  \fancyhead[R]{\small Mihail Anghelici $260928404$} 
			  \fancyfoot[C]{-- \thepage\ --}
			  \renewcommand{\headrulewidth}{0.4pt}}
			\pagestyle{plain}
			\setlength{\headsep}{1cm}
	\captionsetup{margin =1cm}
	\begin{document}
	\maketitle
		\section*{Question 1 }
			\subsection*{(i)}
				\begin{align*}
					-\Delta u = - (u_{x_{1}x_{1}} + \dots + u_{x_{n}x_{n}}) = -\text{tr}(D^{2}u).
				\end{align*}
				Since $L[u] = -\text{tr}(A(x)D^{2}u)$, and $G(Du, u ,x) = 0$ we have 
				$$ -\text{tr}(A(x)D^{2}u) = 0.$$ From which it follows that 
				$$ -\text{tr}(D^{2}u) = -\text{tr}(A(x)D^{2}) \implies A(x) = I_{n}.$$ 
			\subsection*{(ii)}
				\begin{gather*}
					u_{tt} - \Delta u = u_{tt} - u_{x_{1}x_{1}} - \dots - u_{x_{n}x_{n}} = -\text{tr}(A(x,t)D^{2}_{x,t} u) \\
					\implies  u_{x_{1}x_{1}} + \dots u_{x_{n}x_{n}} - u_{tt} = \text{tr}(A(x,t)D^{2}_{x,t}u).
				\end{gather*}
				It is evident that the matrix $A(x,t)$ corresponds to 
				$$ A(x,t) = 
				\begin{bmatrix}
				1 & 0 & \dots & 0 \\
				0 & 1 & & \\
				\vdots & & \ddots & \vdots \\
				0 & & & -1	
				\end{bmatrix}$$
			\subsection*{(iii)}
				Since $$ F(D^{2}u , Du, u ,x) = L[u] + G(Du, u ,x),$$
				we set $G(Du,u,x) = u_t$. Therefore, 
				\begin{gather*}
					u_t - \Delta u = u_{t} - (u_{x_{1}x_{1}} + \dots + u_{x_{n}x_{n}}) = 0 \\
					\implies -\text{tr}(A(x,t)D^{2}_{x,t}u) + u_t = u_{t} +  (u_{x_{1}x_{1}} + \dots + u_{x_{n}x_{n}}).
				\end{gather*}
				It follows evidently, that the matrix $A(x,t)$ corresponds to 
				$$ A(x,t) = 
				\begin{bmatrix}
				1 & 0 & \dots & 0 \\
				0 & 1 & & \\
				\vdots & & \ddots & \vdots \\
				0 & & & 0	
				\end{bmatrix}$$	
			\section*{Question 2}
				\subsection*{(i)}
				From $L[u] = -\text{tr}(A(x,y)D^{2}u)$ we deduce $A(x,y)$ to be ,
				\begin{align*} au_{xx} + 2bu_{xy} + cu_{yy} &= \text{tr}\left( \begin{bmatrix}a & b \\ b& c \end{bmatrix}\begin{bmatrix} u_{xx} & u_{xy} \\ u_{yx} & u_{yy}\end{bmatrix}\right). \\
				\end{align*}
				We then look for the eigenvalues of $A$
				\begin{align*}
				 (a-\lambda)(c-\lambda) - b^2 = 0 &\implies ac - \lambda(a+c) + \lambda^2 -b^2 =0 \\
				&\implies \lambda^2 -\lambda(c+a) + ac-b^2 =0 \\
				&\implies \lambda = \frac{(c+a) \pm \sqrt{(c+a)^2 +4(b^2 -ac)}}{2}
				\end{align*}
				
				\noindent Assuming $\abs{4(b^{2} - ac)} < (c+a)^{2}$ to preserve a real part for $\lambda_{i}$, then if $(b^2 - ac)<0 \implies 4(b^{2} - ac)<0$. We find that both solutions for $\lambda$ are non-zero and of positive sign since $c+a > \Delta \ \forall a,b,c \in \mathbb{R}$.
				\subsection*{(ii)}
					Similarly as in $(\text{i})$ if $(b^{2} - ac) > 0 \implies 4(b^{2} - ac)>0 $. Thus $\Delta > 0  \ \forall a,b,c \in \mathbb{R}$. Moreover, $\Delta > (c+a) \ \forall a,b,c \in \mathbb{R}$ such that $\lambda_{1} >0$ and $\lambda_{2} < 0$, both being nonzero since $\Delta > 0$.
				\subsection*{(iii)}
					Similarly as in $\text{(i)}$, if $(b^{2}- ac) = 0 \implies 4(b^{2}-ac) = 0$. Therefore, $\Delta  = (c+a)$ from which it is evident that one eigenvalue is positive or negative while the other is zero.
			\section*{Question 3}
				We assume $A(x,y)$ is not symmetric, thus $a_{12} \neq a_{21}$. Conversely, $\widetilde{A}(x,y)$ is symmetric therefore $\widetilde{a}_{12} = \widetilde{a}_{21}$. $D^{2}u$ is known so if we expand the trace of the matrix product for both equations we find 
				\begin{gather*}
					a_{11}u_{xx} + a_{12}u_{xy} + a_{21}u_{yx} + a_{22}u_{yy} = \widetilde{a}_{11}u_{xx} + 2\widetilde{a}_{12}u_{xy} + \widetilde{a}_{22}u_{yy}. \tag{1}
				\end{gather*}
				Since the hessian matrix is symmetric ($u \in C^{2}$) then $u_{xy} = u_{yx}$ such that $a_{12}u_{xy} + a_{21}u_{yx} = (a_{12} + a_{21})u_{xy}$. Thus, matching the coefficients of $(1)$ we get an expression for $\widetilde{A}(x,y)$ : 
				$$ \widetilde{A}(x,y) = \begin{bmatrix}
					a_{11} & \frac{a_{12} + a_{21}}{2} \\
					\frac{a_{12} + a_{21}}{2}  & a_{22}
				\end{bmatrix}.$$ 
				$$\vec{\nabla} \cdot \vec{F} = \left(\left( \frac{1}{r^2 \sin \theta}\right)\frac{\partial(F_{r} r^2)}{\partial r} 
				 + \left( \frac{1}{r \sin \theta}\right)\frac{\partial(f_{\theta)\sin \theta}{\partial \theta} + \left( \frac{1}{r\sin \theta}\right)\frac{\partial f_{\phi}}{\partial \phi}\right) dr d\theta d\phi$$
			
	\end{document}
			