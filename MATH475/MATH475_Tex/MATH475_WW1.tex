\documentclass[
	12pt,
	]{article}
		\usepackage{xcolor}
			\usepackage[dvipsnames]{xcolor}
			\usepackage[many]{tcolorbox}
		\usepackage{changepage}
		\usepackage{titlesec}
		\usepackage{caption}
		\usepackage{mdframed, longtable}
		\usepackage{mathtools, amssymb, amsfonts, amsthm, bm,amsmath} 
		\usepackage{array, tabularx, booktabs}
		\usepackage{graphicx,wrapfig, float, caption}
		\usepackage{tikz,physics,cancel, siunitx, xfrac}
		\usepackage{graphics, fancyhdr}
		\usepackage{lipsum}
		\usepackage{xparse}
		\usepackage{thmtools}
		\usepackage{mathrsfs}
		\usepackage{undertilde}
		\usepackage{tikz}
		\usepackage{fullpage,enumitem}
		\usepackage[labelfont=bf]{caption}
	\newcommand{\td}{\text{dim}}
	\newcommand{\tvw}{T : V\xrightarrow{} W }
	\newcommand{\ttt}{\widetilde{T}}
	\newcommand{\ex}{\textbf{Example}}
	\newcommand{\aR}{\alpha \in \mathbb{R}}
	\newcommand{\abR}{\alpha \beta \in \mathbb{R}}
	\newcommand{\un}{u_1 , u_2 , \dots , n}
	\newcommand{\an}{\alpha_1, \alpha_2, \dots, \alpha_2 }
	\newcommand{\sS}{\text{Span}(\mathcal{S})}
	\newcommand{\sSt}{($\mathcal{S}$)}
	\newcommand{\la}{\langle}
	\newcommand{\ra}{\rangle}
	\newcommand{\Rn}{\mathbb{R}^{n}}
	\newcommand{\R}{\mathbb{R}}
	\newcommand{\Rm}{\mathbb{R}^{m}}
	\usepackage{fullpage, fancyhdr}
	\newcommand{\La}{\mathcal{L}}
	\usepackage[math]{cellspace}
		\setlength{\cellspacetoplimit}{3pt}
		\setlength{\cellspacebottomlimit}{3pt}


	\usepackage{mathtools}
	\DeclarePairedDelimiter{\norm}{\lVert}{\rVert}
	\newcommand{\vectorproj}[2][]{\textit{proj}_{\vect{#1}}\vect{#2}}
	\newcommand{\vect}{\mathbf}
	\newcommand{\uuuu}{\sum_{i=1}^{n}\frac{<u,u_i}{<u_i,u_i>} u_i}
	\newcommand{\B}{\mathcal{B}}
	\newcommand{\Ss}{\mathcal{S}}
	
	\newtheorem{theorem}{Theorem}[section]
	\theoremstyle{definition}
	\newtheorem{corollary}{Corollary}[theorem]
	\theoremstyle{definition}
	\newtheorem{lemma}[theorem]{Lemma}
	\theoremstyle{definition}
	\newtheorem{definition}{Definition}[section]
	\theoremstyle{definition}
	\newtheorem{Proposition}{Proposition}[section]
	\theoremstyle{definition}
	\newtheorem*{example}{Example}
	\theoremstyle{example}
	\newtheorem*{note}{Note}
	\theoremstyle{note}
	\newtheorem*{remark}{Remark}
	\theoremstyle{remark}
	\newtheorem*{example2}{External Example}
	\theoremstyle{example}
	
	\title{MATH 475 Weekly Work 0}
	\titleformat*{\section}{\LARGE\normalfont\fontsize{12}{12}\bfseries}
	\titleformat*{\subsection}{\Large\normalfont\fontsize{10}{15}\bfseries}
	\author{Mihail Anghelici 260928404 }
	\date{\today}
	
	\relpenalty=9999
			\binoppenalty=9999
		
			\renewcommand{\sectionmark}[1]{%
			\markboth{\thesection\quad #1}{}}
			
			\fancypagestyle{plain}{%
			  \fancyhf{}
			  \fancyhead[L]{\rule[0pt]{0pt}{0pt} Weekly Work 0 } 
			  \fancyhead[R]{\small Mihail Anghelici $260928404$} 
			  \fancyfoot[C]{-- \thepage\ --}
			  \renewcommand{\headrulewidth}{0.4pt}}
			\pagestyle{plain}
			\setlength{\headsep}{1cm}
	\captionsetup{margin =1cm}
	\begin{document}
	\maketitle
		\section*{Question 1 }
		Let 
		\begin{equation} 
		u(x,t) = X(x)T(t) 
		\end{equation}
			\subsection*{a)}
				$$ T'(t) = 0 \implies T(t) = \int 0 \ dt = c_{1}.$$
				$$ X''(x) = 0 \implies X'(x) = \int 0 \ dx = c_{2} \implies X(x) = \int c_{2} \ dx = c_{2}x + c_{3}.$$
				Replacing in $(1)$ we have 
				$$ u(x,t)=c_{1}(c_{2}x+c_{3}).$$
				Since $c_{1},c_{2},c_{3}$ are all arbitrary constants this is equivalent to $u(x,t) = Ax+B$ for $A,B \in \mathbb{R}$.
			\subsection*{b)}
				\begin{gather*}
					T'(t) = -k \lambda^2 T(t) \implies \frac{T'(t)}{T(t)} = -k \lambda^2 \\
					\implies \int \frac{T'(t)}{T(t)} \ dt = \int -k \lambda^2 \ dt \implies \ln \abs{T(t)} = - k \lambda^2 + c_{1} \\ 
					\implies T(t) = e^{-k t\lambda^2 + c_{1}}
					\intertext{We set $e^{c_{1}} \equiv C$}
					\therefore T(t) = Ce^{-k\lambda^2 t}.
				\end{gather*}
				Next, 
				\begin{gather*}
					X''(x) = -\lambda^2 X(x) \implies \frac{X''(x)}{X(x)} = -\lambda^2 
					\intertext{Solving the caracteristic equation yields}
					\implies X''(x) + \lambda^2 X(x) = 0 \implies X(x) = \pm i \lambda
					\intertext{Complex roots therefore we use the general solution :}
					X(x) = A\cos\lambda x + B\sin \lambda x
				\end{gather*}
				Finally, 
				\begin{align*}
					u(x,t) &= X(x)T(t)  \\
					&= Ce^{-k\lambda^2 t}(A\cos \lambda x + B\sin \lambda x)
					\intertext{$C$ is an arbitrary constant which is absorbed by $A$ and $B$, getting}
					u(x,t) &= e^{-k\lambda^2 t} (A\cos \lambda x + B\sin \lambda x).
				\end{align*}
				\subsection*{c) }
					First,
					\begin{gather*}
					T'(t) = -\lambda^2 k T(t) 
					\implies \ln \lvert T(t)\rvert = -\lambda^2 k t + c_{1}
					\implies T(t) = Ce^{-\lambda^2kt}
					\end{gather*} 
					Then,
					\begin{gather*}
					X''(x) = -\lambda^2 X(x) 
					\intertext{Since $\pm i\lambda = \pm \abs{\lambda}$ for $\lambda$ immaginary we have}
					X(x) = \pm \abs{\lambda} 
					\end{gather*}
					Therefore the solution to the characteristic equation is 
					\begin{align*}
					X(x) = Ae^{\abs{\lambda} x} + Be^{-\abs{\lambda} x} 
					\end{align*}
					Substituting everything back in $(1)$ we get
					$$ u(x,t)= e^{-\abs{\lambda}^2 kt}(Ae^{\abs{\lambda} x} + Be^{-\abs{\lambda} x}).$$
			\newpage		
			\section*{Question 2}
				\subsection*{a)}
					The only solution that can satisfy this PDE is evidently $u(x,t) = Ax+B$ since it's the only linear solution, hence the only solution that can satisfy the initial condition $u(x,0) = 3x$ on $(0,L)$. Solving : 
					\begin{align*}
						u(0,t) = 0 \implies B=0 \qquad , u(x,0) = Ax + B = 3x \implies A=3.
					\end{align*}
					Therefore the final solution is $u(x,t) = 3x$ since it also satisfies $u(L,t)=3L$ by symmetry of the initial condition.
					
				\subsection*{b) }
					The initial condition is of exponential nature so we dismiss the $\lambda^2 = 0$ linear case. Let us attempt the $\lambda^2 < 0$ case.
					\begin{align*}
						u(x,0) &= Ae^{\abs{\lambda}x} + B^{-\abs{\lambda}x}= 2e^{3x}+2e^{-3x}
						\intertext{The only solutions to this equality is $A=B=2$ and $\abs{\lambda} =3$. We proceed further,}
						u_{x} &= \abs{\lambda}e^{2\abs{\lambda}^{2}t} (Ae^{\abs{\lambda}x} - Be^{-\abs{\lambda}x}) \\
						u_{x}(L,t) &= \abs{\lambda} e^{2\abs{\lambda}^{2}t}(Ae^{\abs{\lambda}L}-Be^{-\abs{\lambda}L}) \\
						&= 3 e^{2\abs{3}^{2}t}(2e^{\abs{3}L} - 2e^{-\abs{3}L})\\
						&= 6e^{18t}(e^{3L} - e^{-3L})
					\end{align*}
					We conclude the $\lambda^{2} < 0$ case satisfies the given BVP since it also satisfies the boundary $u_{x}(0,t) = 0$ given that $(A=B)$, and the $\lambda^{2} >0$ case is dismissed since only one case can satisfy the BVP.
				\section*{Question 3}
					\subsection*{a) }
						Since only the $\lambda^{2} > 0$ satisfies the BVP, we check the boundary
						$$ u(0,t)=0 \implies e^{-3 \lambda^{2}t}A = 0,$$
						since $e^{-3\lambda^{2}t} \neq  0 \ \forall \ t\in \mathcal{I}$ we conclude that $A=0$.
						
						\noindent Now verifying the right end boundary condition, 
						\begin{align*}
							u(1,t) &= 0 \implies e^{-3 \lambda^{2}t}B\sin(\lambda (1)) = 0
							\intertext{Expressing $\lambda = n \pi / 1$ we get }
							u(1,t) &= e^{-3 n^{2}\pi^{2}t}B\sin(n\pi (1))
							\intertext{Extending the spatial domain to $x$ we obtain}
							u(x,t) &= e^{-3 \pi^{2}n^{2}t}B\sin(n\pi x).
						\end{align*}
					\subsection*{b) }
						First, we note that the PDE as defined is linear (transport equation), therefore a linear combination of solutions is also a solution.$ u(x,t) = Be^{-3n^2 \pi^2 t} \sin(n \pi x)$ suggests there exists an infinite number of solutions since $n \in \mathbb{Z}$ , thus 
						\begin{equation}
							u(x,t) := \sum_{n=-N}^{N}B_{n}e^{-3n^2 \pi^2 t} \sin(n \pi x)
						\end{equation}
						 is a solution to the given PDE.
						
						We can truncate the sum from $n=-N \to 0$ since the $B_{n<0}$ are absorbed in the terms $B_{n>0}$. For illustrative purposes, let us consider the $n=-1 \to 1$ case : 
						\begin{align*}
							\sum_{n=-1}^{1} &= B_{-1}e^{\alpha}(\sin (-\pi x)) + B_{0}0 + B_{1}e^{\alpha}\sin(\pi x)
							\intertext{Using the identity $\sin(-\pi x) = -\sin(\pi x)$, this is equivalent to}
							&=(B_{1} - B_{-1})e^{\alpha}\sin(\pi x)
						\end{align*}
						Since $B_{n}$ are arbitrary numbers $(2)$ can indeed be expressed as
						$$ u(x,t) := \sum_{n=0}^{N}B_{n}e^{-3n^2 \pi^2 t} \sin(n \pi x).$$
					\subsection*{c) }
						Using the general solution found in $3 \ \text{b})$ , 
						\begin{gather*}
							u(x,0) = \sum_{n=0}^{N} B_{n}\sin(n\pi x) = 5\sin (2\pi x) - 30 \sin(n \pi x) \implies B_{1} =0 \ , B_{2} = 5 \, B_{3} = -30.
							\intertext{Therefore we may express the general solution to this particular PDE as}
							u(x,t) = 5e^{-3(\pi^{2})2^{2}t}\sin(2\pi x) - 30e^{-3\pi^{2}3^{2}t}\sin(3 \pi x).
						\end{gather*}
						
		
	\end{document}