\documentclass[12pt]{article}
\newcommand\hmmax{0}
\newcommand\bmmax{0}
\usepackage{xcolor}
\usepackage[dvipsnames]{xcolor}
\usepackage[many]{tcolorbox}
\usepackage{changepage}
\usepackage{titlesec}
\usepackage{caption}
\usepackage{mdframed, longtable}
\usepackage{mathtools, amssymb, amsfonts, amsthm, bm,amsmath} 
\usepackage{array, tabularx, booktabs}
\usepackage{graphicx,wrapfig, float, caption}
\usepackage{tikz,physics,cancel, siunitx, xfrac}
\usepackage{graphics, fancyhdr}
\usepackage{lipsum}
\usepackage{xparse}
\usepackage{thmtools}
\usepackage{mathrsfs}
\usepackage{undertilde}
\usepackage{tikz}
\usepackage{fullpage,enumitem}
\usepackage[labelfont=bf]{caption}
\newcommand{\td}{\text{dim}}
\newcommand{\tvw}{T : V\xrightarrow{} W }
\newcommand{\ttt}{\widetilde{T}}
\newcommand{\ex}{\textbf{Example}}
\newcommand{\aR}{\alpha \in \mathbb{R}}
\newcommand{\abR}{\alpha \beta \in \mathbb{R}}
\newcommand{\un}{u_1 , u_2 , \dots , n}
\newcommand{\an}{\alpha_1, \alpha_2, \dots, \alpha_2 }
\newcommand{\sS}{\text{Span}(\mathcal{S})}
\newcommand{\sSt}{($\mathcal{S}$)}
\newcommand{\la}{\langle}
\newcommand{\ra}{\rangle}
\newcommand{\Rn}{\mathbb{R}^{n}}
\newcommand{\R}{\mathbb{R}}
\newcommand{\Rm}{\mathbb{R}^{m}}
\usepackage{fullpage, fancyhdr}
\newcommand{\La}{\mathcal{L}}
\newcommand{\ep}{\epsilon}
\newcommand{\de}{\delta}
\usepackage[math]{cellspace}
\setlength{\cellspacetoplimit}{3pt}
\setlength{\cellspacebottomlimit}{3pt}
\newcommand\numberthis{\addtocounter{equation}{1}\tag{\theequation}}
\usepackage{newtxtext, newtxmath}


\usepackage{mathtools}
\DeclarePairedDelimiter{\norm}{\lVert}{\rVert}
\newcommand{\vectorproj}[2][]{\textit{proj}_{\vect{#1}}\vect{#2}}
\newcommand{\vect}{\mathbf}
\newcommand{\uuuu}{\sum_{i=1}^{n}\frac{<u,u_i}{<u_i,u_i>} u_i}
\newcommand{\Ss}{\mathcal{S}}
\newcommand{\A}{\hat{A}}
\newcommand{\B}{\hat{B}}
\newcommand{\C}{\hat{C}}
\newcommand{\dr}{\mathrm{d}}
\allowdisplaybreaks
\usepackage{titling}
\newtheorem{theorem}{Theorem}[section]
\theoremstyle{definition}
\newtheorem{corollary}{Corollary}[theorem]
\theoremstyle{definition}
\newtheorem{lemma}[theorem]{Lemma}
\theoremstyle{definition}
\newtheorem{definition}{Definition}[section]
\theoremstyle{definition}
\newtheorem{Proposition}{Proposition}[section]
\theoremstyle{definition}
\newtheorem*{example}{Example}
\theoremstyle{example}
\newtheorem*{note}{Note}
\theoremstyle{note}
\newtheorem*{remark}{Remark}
\theoremstyle{remark}
\newtheorem*{example2}{External Example}
\theoremstyle{example}
\usepackage{bbold}
\title{PHYS350 Assignment 6}
\titleformat*{\section}{\LARGE\normalfont\fontsize{14}{14}\bfseries}
\titleformat*{\subsection}{\Large\normalfont\fontsize{12}{15}\bfseries}
\author{Mihail Anghelici 260928404 }
\date{\today}

\relpenalty=9999
\binoppenalty=9999

\renewcommand{\sectionmark}[1]{%
	\markboth{\thesection\quad #1}{}}

\fancypagestyle{plain}{%
	\fancyhf{}
	\fancyhead[L]{\rule[0pt]{0pt}{0pt} Assignment 6} 
	\fancyhead[R]{\small Mihail Anghelici $260928404$} 
	\fancyfoot[C]{-- \thepage\ --}
	\renewcommand{\headrulewidth}{0.4pt}}
\pagestyle{plain}
\setlength{\headsep}{1cm}
\captionsetup{margin =1cm}
	\begin{document}
	\maketitle
		\section*{Question 1}
		\subsection*{a)}
 			The charge inside the ${C_{60}}^{-}$ ion is negative so we use the method of images with $q' + q'' = q$. We then proceed with
			\begin{align*}
				F = qE \implies f &= \frac{q}{4 \pi \ep_{0}}\left(\frac{q'' }{a^{2}} + \frac{q' }{(a-b)^{2}}\right) 
				\intertext{Since $q'' = q - q'$, }
				&= \frac{q}{4 \pi \ep_{0}} \left(\frac{q}{a^{2}} - \frac{q'}{a^{2}} + \frac{q' }{(a-b)^{2}}\right) \\
				&= \frac{q^{2}}{4 \pi \ep_{0}} + \underbrace{\frac{qq'}{4 \pi \ep_{0}} \left(\frac{-1}{a^{2}} + \frac{1}{(a-b)^{2}}\right)}_{\text{HW } \ 4} \\
				&= \frac{q^{2}}{4 \pi \ep_{0}a^{2}} + \frac{q^{2}}{4\pi \ep _{0}} \left(\frac{R}{a}\right)^{3} \frac{(2a^{2} -R^{2})}{(a^{2}-R^{2})^{2}}\\
				&= \frac{q^{2}}{4 \pi \ep_{0}} \left(a - R^{3} \frac{(2a^{2} -R^{2})}{(a^{2} -R^{2})^{2}}\right).
			\end{align*}
			We look for $r$ at $F=0$. At that force, $a(a^{2} - R^{2})^{2} = R^{3} (2a^{2} -R^{2})$. Solving numerically we get 
			$$ R_{+} = \frac12 (\sqrt{5} -1)a ; \qquad R_{-} -\frac12 (1+\sqrt{5})a.$$
			We chose the positive root since the distance is a positive quantity in this setting. 
			$$ \implies R = \frac12  (\sqrt{5} -1)a  \implies a -\frac{2}{\sqrt{5} -1} = \frac12 (1+\sqrt{5}) \implies a \approxeq 5.663 \ \si{\angstrom}.$$
		\subsection*{b)}
			Let $a = \gamma R$. By definition 
			$$ W = - \int_{\infty}^{a} F\ \dr a  = \underbrace{\int_{\gamma}^{\infty} \frac{1}{4 \pi \ep_{0} R } \left(\frac{1}{\gamma^{2}}\right) \dr \gamma}_{:= I_{1}} - \underbrace{\int_{\gamma}^{\infty} \frac{2 (\gamma^{2} -1)}{\gamma^{3} (\gamma^{2} -1)^{2}} \dr \gamma}_{:= I_{2}}.$$
			We solve the two integrals ,$I_{1}$ is trivial. For $I_{2},$ let $u = \gamma^{2}$ , 
			\begin{gather*}
				I_{2} = -\frac12 \int \frac{2u-1}{(u-1)^{2} u^{2}} \dr u = \frac12 \int \frac{1}{u^{2}} \dr u - \frac12 \int \frac{1}{(u-1)^{2}} \dr u = -\frac{1}{2u} - \frac12 \int \frac{1}{s^{2}} \dr s = \frac{1}{2(u-1)u}.\\
				\therefore I_{2} = \frac{1}{2\gamma^{2} (\gamma^{2} -1)}.
			\end{gather*}
			We conclude 
			$$ W = \frac{q^{2}}{4 \pi \ep_{0}R}\left( - \frac{1}{\gamma } + \frac{1}{2\gamma^{2} (\gamma^{2} -1)}\right) = \frac{q^{2}}{4\pi \ep_{0} R} \frac{1-2\gamma^{3} + 2\gamma}{2\gamma^{2} (\gamma^{2} -1)}.$$
			Substituting $\gamma = (1+\sqrt{5} / 2)$ and replacing with the appropriate constants, 
			$$ W = \frac{q^{2}}{8 \pi \ep_{0} R} = \frac{(1.6 \times 10^{-19})^{2} }{8 \pi (8.85 \times 10^{-12}) (5.66 \times 10^{-10})^{2}} = 2.03 \times 10^{-14} \ \si{\joule} = 1.27 \ \si{\electronvolt}.$$
		\section*{Question 2}
			We show that $3(\vec{p} \cdot \hat{r}) \hat{r} - \vec{p} = 2 \cos\theta \hat{r} + \sin \theta \hat{\theta}$ then we are done. In spherical coordinates with $\varphi \equiv 0$ and $\vec{p } \parallel \hat{z}$ , we have that 
			$$ (\vec{p} \cdot \hat{r} )\hat{r} = p \cos\theta \qquad \text{ and } \ \ \ (\vec{p} \cdot \hat{r}) \hat{\theta} = - p \sin\theta,$$
			so then 
			$$ \vec{p}(r,\theta) = p\cos \theta - p\sin \theta \implies 3(\vec{p } \cdot \hat{r})\hat{r} - \vec{p} = 3p \cos\theta - p\cos\theta + p\sin \theta = 2p\cos\theta + p\sin \theta \ \ \checkmark. $$
		\section*{Question 3}
			\subsection*{a) }
				The density volume charge is $\rho(r) = Ar$. Then by Gauss's law, 
				$$ E(4 \pi r^{2}) = \frac{Q_{\text{enc}}}{\ep_{0}} = \oint E \cdot \dr a.$$
				We find $Q_{\text{enc}}$ ; 
				$$ Q_{\text{enc}} = \int_{V} \rho(r) \dr \tau = \int_{V} Ar \dr \tau = \int_{0}^{r} Ar (4\pi r^{2}) \dr r \implies \frac{4 Ar^{4} \pi}{4 \ep_{0}} = 4\pi r^{2} E \implies E = \frac{Ar^{2} }{4 \ep_{0}}.$$
				Since the dipole moment is $p = ed$, we need to find $d$ . Let $d\equiv r$, then 
				$$ E = \frac{Ad^{2}}{4 \ep_{0}} \implies d = \sqrt{\frac{4 \ep_{0} E }{A}} =\implies p = 2e \sqrt{\frac{E \ep_{0}}{A}} \ \therefore p \propto \sqrt{E}.$$
			\subsection*{b) }
				Since $E \propto r$ and $E(0) \neq 0 \implies \rho(r) \propto E \implies \rho(r) \propto r $ and $\rho(0) \neq 0$.
		\section*{Question 4}
			By symmetry, each dipole induces an electric field of the same strength. So we let $\vec{p} = \vec{p_{1}}$ and let the electric field be due to the second dipole. Then, 
			\begin{align*}
			E_{\text{dip} ,2} = \frac{1}{4\pi \ep_{0} r^{3}} (3 (\vec{p_{2}} \cdot \hat{r}) \hat{r} - \vec{p_{1}}) \implies -\vec{p_{1}} \cdot E_{2} &= - \vec{p_{1}} \cdot (3 (\vec{p_{2}} \cdot \hat{r}) ) + \vec{p_{1}} \cdot \vec{p_{2}} \\
			&= (- \vec{p_{1} } \cdot\hat{r}) 3(\vec{p_{2}} \cdot \hat{r}) + \vec{p_{1}} \cdot \vec{p_{2}} \\
				&= -3 (\vec{p_{1}} \cdot \hat{r}) (\vec{p_{2}} \cdot \hat{r}) + \vec{p_{1}} \cdot \vec{p_{2}}. 
			\end{align*}
		\section*{Question 5}
			Given $\sigma_{b} = P \cdot \hat{n}$ and $\rho_{b} = -\nabla \cdot P $, the total charge on the dielectric is 
			\begin{align*}
				 Q_{\text{tot}} = \oint\limits_{S} \sigma_{b} \dr a + \int\limits_{V} \rho_{b} \dr \tau &= \oint\limits_{S} (P \cdot \hat{n} ) \dr a - \int\limits_{V} (\nabla \cdot P ) \dr \tau \\
				 &= \oint\limits_{S} P \cdot \dr \boldsymbol{a} - \int\limits_{V} \nabla \cdot P \dr \tau,
			\end{align*}
			by the divergence theorem the last two integrals are equal so $Q_{\text{tot}} = 0$; the total bound charge vanishes. 
		\section*{Question 6}
			We use 
			\begin{align*}
				E_{\text{dip}} (\vec{r} , \theta) &= \frac{p}{4 \pi \ep_{0} r^{3} } (2 \cos \theta \hat{r} + \sin \theta \hat{\theta}) 
				\intertext{Sicne the distance is fixed we set $\hat{r} = 0$ giving us }
				&= \frac{p}{4 \pi \ep_{0} r^{3}} (\sin \theta \hat{\theta}) 
			\end{align*}
 			Since $0 \le  \sin \theta  \le 1$ , it follows that 
			$$ \max E_{\text{dip}}(\vec{r},\theta) = \frac{p }{4 \pi \ep_{0} r^{3}}.$$
			Since the dipole moment is on the $\hat{z}$ axis and $\sin(0) =1$ we conclude that this $E_{\text{max}}$ is parallel transposed on the $\vec{P} \implies$ same orientation .
		\subsection*{b) }
			We have that 
			\begin{align*}
				U &= \frac{1}{4\pi \ep_{0} r^{3}} (\vec{p_{1}} \cdot \vec{p_{2}} - 3(\vec{p_{1}} \cdot \hat{r}) (\vec{p_{2}} \cdot \hat{r})) \\
				&= \frac{1}{4 \pi \ep_{0} r^{3}} (\abs{p_{1}} \abs{p_{2}} \cos \theta_{p_{1}p_{2}} - \cancelto{0 , \ \text{since no } r \text{dependence}}{3(\vec{p_{1}} \cdot \hat{r}) (\vec{p_{2}} \cdot \hat{r}))}
				\intertext{It follows that when $\cos\theta_{p_{1} p_{2}} = 0$ or $1$ we have the minmimum and maximum interaction energy respectively} 
				& \qquad \implies \max U = \frac{p_{1}p_{2}}{4 \pi \ep_{0} r^{3}} ; \qquad \min U = 0,
			\end{align*}
			these values physically represent two dipoles oriented along the same axis and oriented perpendicularly ,respectively. 
		\subsection*{c) }
			Given that one mole of proton is $\sim 1 \ \si{\gram}$, then $m_{p} = 1/ N_{A} = 1\ 6 \cross 10^{23}.$ Then we have that for $18 \ \text{amu},$ $18/(6 \times 10^{23}) = 3 \cross 10^{-26} \ \si{\kilogram}.$ We then compute the normal vector
			$$ \hat{n} = \frac{\rho}{m_{H_{2}O}}= \frac{1000 \ \si{\kilogram\per\meter\cubed}}{3 \times 10^{26} \ \si{\kilogram}} = 3 \times 10^{28} \ \si{\meter\cubed}$$. 
			Then we use 
			$$ U \sim \vec{p} \cdot \vec{E} = \frac{p^{2}}{4 \pi \ep_{0} d^{3}} = \frac{p^{2}}{4 \pi \ep_{0} (n^{(-1/3)})^{3}} = \frac{(1.85 \times 3.33 \times 10^{-30} )^{2}}{4 \pi \times 8.85\times 10^{-12} (3.22 \times 10^{-10})^{3})} \approxeq 1.01 \times 10^{-20} \ \si{\joule}. $$
			For steam, we use $pV = NkT$ for which 
			$$ pV = NkT \implies (10^{5} \ \si{\pascal})d^{3} = (1)k (373.15 \ \si{\kelvin}) \implies d = 3.62 \times 10^{-9},$$
			we then use the same interaction energy formula ; 
			$$ U = \frac{p^{2} }{4 \pi \ep_{0} r^{3}} = \frac{(1.85 \times 3.33 \times 10^{-30} )^{2}}{4 \pi \times 8.85\times 10^{-12} (3.62 \times 10^{-9})^{3}} \approxeq 7.2 \times 10^{-24} \ \si{\joule}.$$
			Finally, we know that $kT$ actually represents the energy in $\si{\joule}$ such that for liquid water 
			$$ kT = \text{Energy} = 1.01 \times 10^(-20) \xrightarrow{k = 1.38 \times 10^{-23}} T = 732.6 \ \si{\kelvin}.$$
			And for steam, 
			$$ kT = \text{Energy} = 7.2 \times 10^{-24} \xrightarrow{k = 1.38 \times 10^{-23}} T = 0.51 \ \si{\kelvin}.$$
			the temperatures should be larger than these values to exceed the dipole interaction energy, respectively.
	\end{document}